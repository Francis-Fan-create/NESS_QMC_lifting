\documentclass[11pt,a4paper]{scrartcl}
\usepackage[utf8]{inputenc}
\usepackage[T1]{fontenc}
\usepackage[british]{babel}
\usepackage{csquotes}
\usepackage{amsmath}
\usepackage{amsfonts}
\usepackage{amssymb}
\usepackage{makeidx}
\usepackage{graphicx}
\usepackage{hyperref}
\usepackage{mathtools}
\usepackage{color}
\usepackage{amsthm}
\usepackage{mathrsfs}
\usepackage{float}
\usepackage{enumerate}
\usepackage{comment}
\usepackage{caption}
\usepackage{subcaption}
\usepackage{mdframed}
%\usepackage{ulem}

\bibliographystyle{alpha}

\hyphenation{semi-group}






\usepackage{todonotes}
\newcommand{\jl}[1]{\todo[color=blue!35]{JL: #1}}
\newcommand{\jlinline}[1]{\todo[inline,color=blue!35]{JL: #1}}


\hypersetup{
	colorlinks=true,
	linkcolor=blue,
	citecolor=blue,
	filecolor=blue,      
	urlcolor=blue
}

\usepackage{tikz, pgf}

\title{
Stability of Optimal Acceleration in Non-Equilibrium Lifting: A Perturbative Analysis}
\author{
Zexi Fan\thanks{School of Mathematics, Peking University. E-Mail: \href{mailto:2200010816@stu.pku.edu.cn}{2200010816@stu.pku.edu.cn}}\qquad 
Bowen Li\thanks{Department of Mathematics, City University of Hong Kong. E-Mail: \href{mailto:boweli4@cityu.edu.hk}{boweli4@cityu.edu.hk}}\qquad 
Jianfeng Lu\thanks{Department of Mathematics, Department of Physics, Department of Chemistry, Duke University. E-Mail: \href{mailto:jianfeng@math.duke.edu}{jianfeng@math.duke.edu}}
}
\date{\today}

\newcommand{\Lhat}{{\widehat{\mathcal L}}}
\newcommand{\mL}{{\mathcal L}}
\newcommand{\E}{\mathcal{E}}
\newcommand{\Ld}{{\widehat{\mathcal L}_\mathrm{tr}}}
\newcommand{\Lv}{{\widehat{\mathcal L}_v}}

\newcommand{\zexi}[1]{{\color{red}{ [\textbf{Zexi:}} #1]}}
\newcommand{\bowen}[1]{{\color{magenta}{ [\textsf{\textbf{Bowen:}} #1}]}}
\newcommand{\jianfeng}[1]{{\color{blue}{ [\textbf{Jianfeng:}} #1]}}

\newcommand{\myAngle}[1]{\left< #1 \right>_{L^2(\lambda \otimes \mu)}}
\newcommand{\myAngleHat}[1]{\left< #1 \right>_{L^2(\lambda \otimes \widehat\mu)}}
\newcommand{\myVert}[1]{\left\Vert #1 \right\Vert_{L^2(\lambda \otimes \mu)}}
\newcommand{\myVertHat}[1]{\left\Vert #1 \right\Vert_{L^2(\lambda \otimes \widehat\mu)}}
\newcommand{\myIndent}{&\quad\quad\quad}
\newcommand{\LvPoincareConstant}{{\frac1{m_v}}}
\newcommand{\LvInvPoincareConstant}{m_v}
\newcommand{\auxConstantIntLtrTransposeLtrL}{{C^*_1}}
\newcommand{\auxConstantIntLtrTransposeLtrE}{{C^\dagger_1}}
\newcommand{\auxConstantIntLtrTransposeLv}{{C^*_2}}
\newcommand{\Shat}  {{\widehat{\mathcal S}}}
\renewcommand{\S}   {\mathcal{S}}
\newcommand{\V}     {\mathcal{V}}
\newcommand{\diff}  {\mathop{}\!\mathrm{d}}
\newcommand{\comp}  {\mathrm{c}}
\newcommand{\R}     {\mathbb{R}}
\newcommand{\D}     {\mathrm{D}}

\newcommand{\eff}{\textrm{eff}}
\newcommand{\ham}{\textrm{ham}}
\newcommand{\pert}{\textrm{pert}}

\DeclareMathOperator{\dom}  {Dom}
\DeclareMathOperator{\Gap}  {Gap}
\let\Re\relax
\let\Im\relax
\DeclareMathOperator{\Re}   {Re}
\DeclareMathOperator{\Im}   {Im}
\DeclareMathOperator*{\argmax}{arg\,max}
\DeclareMathOperator{\Var}  {Var}
\DeclareMathOperator{\Cov}  {Cov}
\DeclareMathOperator{\tr}   {tr}
\DeclareMathOperator{\spn}  {span}

\numberwithin{equation}{section}
\theoremstyle{plain}
\newtheorem{theorem}{Theorem}[section]
\newtheorem{lemma}{Lemma}[section]
\newtheorem{proposition}{Proposition}[section]
\newtheorem{corollary}{Corollary}[section]

\theoremstyle{definition}
\newtheorem{definition}{Definition}[section]
\newtheorem{remark}{Remark}[section]
\newtheorem{example}{Example}[section]
\newtheorem{notation}{Notation}[section]
\newtheorem{assumption}{Assumption}[section]

\usepackage{autonum}

\begin{document}

\maketitle

\begin{abstract}
    Lifting techniques have proven highly effective for accelerating convergence to equilibrium in reversible Markov processes, achieving optimal diffusive-to-ballistic speedup with convergence rate $\nu(\mathcal{L}) = \Theta(\sqrt{s(\mathcal{L}_O)})$, where $s(\mathcal{L}_O)$ is the spectral gap of the slow dynamics. However, nonequilibrium steady states (NESS) break detailed balance, introducing non-normality in the generator that can destabilize spectral properties and potentially invalidate standard hypocoercive analysis. We address the fundamental question: \textit{Is the optimal lifting acceleration structurally stable under non-equilibrium perturbations?}
    
    We establish rigorous perturbative bounds under which the answer is affirmative. For systems with a \textit{near-equilibrium lifting structure}---characterized by a weak non-conservative perturbation of strength $\eta$ to an otherwise reversible slow dynamics---we prove that the optimal quadratic speedup persists provided $\eta \ll s(\mathcal{L}_O)^2$. The key mathematical tool is an \textit{Approximate Quadratic Form Condition} that quantifies the deviation from reversibility as an explicitly bounded error proportional to $\eta$. This enables adaptation of the Flow Poincar\'e inequality framework to the non-equilibrium regime, yielding explicit bounds: $\nu(\mathcal{L}) = \Theta(\sqrt{s(\mathcal{L}_O) + \eta/s(\mathcal{L}_O)})$. Our analysis demonstrates that hypocoercive acceleration is robust against sufficiently small symmetry-breaking perturbations, with the critical threshold $\eta_{\text{crit}} = \Theta(s(\mathcal{L}_O)^2)$ emerging naturally from spectral stability requirements. We validate this stability result through numerical experiments on classical Langevin dynamics and quantum Zeno spin chains.
    \par\vspace\baselineskip
    \noindent\textbf{Keywords:} Hypocoercivity; Lifting; Nonequilibrium steady states; Spectral stability; Pseudospectrum; Adiabatic elimination.\par
\end{abstract}

\section{Introduction}

Recent advances in hypocoercivity theory have established that lifting techniques can achieve optimal acceleration of convergence to equilibrium in reversible Markov processes. For a slow dynamics with spectral gap $\lambda_O$, classical results show that direct simulation converges at a diffusive rate $\nu \sim \lambda_O$. However, by embedding the state space into a higher-dimensional space and introducing carefully designed coupling between auxiliary variables and strong dissipation, one can construct a lifted generator $\mathcal{L}$ that achieves ballistic convergence $\nu(\mathcal{L}) = \Theta(\sqrt{\lambda_O})$ \cite{Li2025,Eberle2024a}. This quadratic speedup, derived via Flow Poincar\'e inequalities \cite{Eberle2025,Li2024}, represents the theoretical maximum for Markovian lifts and has been rigorously established for reversible systems satisfying detailed balance.

However, many physical systems of interest relax not to thermal equilibrium but to nonequilibrium steady states (NESS). Examples include quantum transport with boundary reservoirs at different temperatures \cite{Landi2022}, driven-dissipative quantum phases \cite{Lang2015}, and reservoir-engineered quantum states \cite{Sellem2024,Robin2024,Naseem2025}. These systems exhibit currents, break detailed balance ($\mathcal{L}^\dagger \sigma_{NESS} = 0$ but $\mathcal{L}$ non-reversible), and display complex relaxation phenomena such as Zeno localization \cite{Maimbourg2021,Popkov2025}. The breaking of detailed balance introduces fundamental mathematical challenges: generators become non-normal operators, eigenvalues acquire imaginary parts, and spectral gaps can be highly sensitive to perturbations due to pseudospectral effects \cite{Trefethen2005}. These pathologies raise a critical question that has remained unresolved:

\vspace{0.3em}
\begin{center}
\textit{Is the optimal lifting acceleration structurally stable under non-equilibrium perturbations?}
\end{center}
\vspace{0.3em}

Standard hypocoercive analysis relies crucially on self-adjointness (or reversibility) of the slow generator $\mathcal{L}_O$ with respect to the steady-state inner product. This structure ensures real spectra, orthogonal eigenmodes, and stable quadratic form estimates. In the non-equilibrium regime, none of these properties are guaranteed. While numerical experiments suggest that acceleration persists in certain NESS systems \cite{Monmarche2022,LeRegent2023}, the absence of rigorous bounds leaves open the possibility of catastrophic collapse: even small symmetry-breaking perturbations could, in principle, destroy hypocoercivity through non-normal amplification mechanisms.

\paragraph{Main Result: Perturbative Stability Bounds.}
In this work, we provide the first rigorous stability analysis of optimal lifting acceleration in the non-equilibrium regime. We do not propose new construction algorithms; rather, we establish quantitative perturbative bounds under which the known optimal rates remain valid. Our analysis applies to systems with a \textbf{near-equilibrium lifting structure}---a class of generators $\mathcal{L} = \gamma \mathcal{R} + \mathcal{V}$ arising naturally from adiabatic elimination, where the effective slow dynamics $\mathcal{L}_O$ is a weak perturbation of a reversible baseline. Specifically, we decompose $\mathcal{V} = \mathcal{L}_{\ham} + \eta \mathcal{L}_{\pert}$, where $\mathcal{L}_{\ham}$ is skew-adjoint (Hamiltonian) and $\eta \mathcal{L}_{\pert}$ is a non-conservative perturbation of strength $\eta > 0$ that breaks detailed balance.

Our central theoretical contribution (Corollary \ref{cor:optimal_time_prefactor_ae}) proves that the quadratic speedup $\nu(\mathcal{L}) = \Theta(\sqrt{s(\mathcal{L}_O)})$ persists provided the perturbation satisfies the critical threshold:
\begin{equation}
    \eta \ll s(\mathcal{L}_O)^2, \label{eq:intro_threshold}
\end{equation}
where $s(\mathcal{L}_O)$ is the singular value gap of the effective generator. This bound is sharp: at $\eta = \Theta(s(\mathcal{L}_O)^2)$, the non-equilibrium correction $\eta/s(\mathcal{L}_O)$ becomes comparable to the gap itself, and the perturbative regime breaks down. The threshold \eqref{eq:intro_threshold} emerges from two independent requirements: (i) spectral stability of the effective generator under perturbation (Proposition \ref{prop:quad_form_condition_ae_approx}), and (ii) positivity of the Flow Poincar\'e constant (Theorem \ref{thm:flow_poincare_ae}).

\paragraph{Key Mathematical Tool: Approximate Quadratic Form.}
The technical core of our analysis is an \textit{Approximate Quadratic Form Condition} (Proposition \ref{prop:quad_form_condition_ae_approx}), which quantifies the deviation of the lifted generator from the ideal reversible case. For reversible systems, the quadratic form identity $\langle \mathcal{L} X, S \mathcal{L} Y \rangle = \langle X, |\mathcal{L}_O| Y \rangle$ holds exactly, enabling direct application of hypocoercive estimates. In our setting, this identity fails due to the non-conservative perturbation, but we establish an explicit error bound:
\begin{equation}
    \left| \langle \mathcal{L} X, S \mathcal{L} Y \rangle_{\mathcal{H}} - \langle X, |\mathcal{L}_O| Y \rangle_{\mathcal{H}_S} \right| \le \eta C_{AQF} \|X\|_{\mathcal{H}}\|Y\|_{\mathcal{H}},
\end{equation}
where $C_{AQF} = \Theta(s(\mathcal{L}_O)^{-1})$ is derived from operator perturbation theory. This approximate quadratic form allows us to extend the Flow Poincar\'e inequality framework \cite{Eberle2025,Li2024} to the non-equilibrium setting, treating the symmetry-breaking effects as controlled perturbations rather than fundamental obstructions.

\paragraph{Relation to Adiabatic Elimination.}
Our framework connects to the rich literature on adiabatic elimination (AE) \cite{Tokieda2025,Azouit2016}, Schrieffer-Wolff transforms \cite{Malekakhlagh2022}, and Zeno dynamics \cite{Popkov2018,Popkov2021}. We employ the Generalized Schrieffer-Wolff formalism \cite{Kessler2012} to rigorously derive the effective generator $\mathcal{L}_O$ from the full system $\mathcal{L} = \gamma\mathcal{R} + \mathcal{V}$ in the limit $\gamma \to \infty$. This reveals a duality: while AE extracts slow dynamics from a complex system, lifting reverses this process by constructing an accelerated dynamics from a known target. Our analysis demonstrates that this duality extends to the non-equilibrium regime, provided the perturbation remains within the stability threshold. Notably, AE routinely produces non-equilibrium effective generators \cite{LeRegent2023,Monmarche2022}, making our stability results directly applicable to physically derived systems.

\paragraph{Organization.}
Section \ref{sec:preliminaries} reviews quantum Markov semigroups, the KMS inner product, and hypocoercivity. Section \ref{sec:adiabatic_elimination} develops the Generalized Schrieffer-Wolff formalism for adiabatic elimination. Section \ref{sec:ae_to_lifting} introduces the near-equilibrium lifting structure and the approximate quadratic form condition. Section \ref{sec:convergence_bounds} proves our main stability results: upper bounds (Theorem \ref{thm:upper_bound_convergence_ae}), lower bounds via Flow Poincar\'e inequalities (Theorem \ref{thm:flow_poincare_ae}), and the stability criterion $\eta < C_{corr}^{-1}$. Section \ref{sec:asymptotic_analysis} formalizes the near-equilibrium scaling limit $\eta = o(s^2)$ and proves that optimal quadratic acceleration $\Theta(\sqrt{s})$ is structurally stable in this regime (Corollary \ref{cor:optimal_time_prefactor_ae}). Section \ref{sec:examples} validates the theory through classical Langevin dynamics and quantum Zeno spin chains, including numerical verification. Section \ref{sec:conclusion} discusses implications and open problems. Proofs are in Appendices \ref{app:prelim_proof}--\ref{app:lower_proof}.

\section{Preliminaries}
\label{sec:preliminaries}
In this section, we introduce the foundational concepts and properties required throughout the paper. We provide a concise overview of Quantum Markov Semigroups, ergodicity, and the KMS inner product, with a particular focus on the role of hypocoercivity.

%\section{Notations and General Definitions}

Throughout, we will use the following notations. Let $\mathcal{H}$ be a finite-dimensional Hilbert space and $\mathcal{B}(\mathcal{H})$ be the associated algebra of linear operators. We denote by $\mathbf{1}\in\mathcal{B}(\mathcal{H})$ the identity element and by $\mathrm{id}$ the identity map on $\mathcal{B}(\mathcal{H})$. The adjoint of an operator $X\in\mathcal{B}(\mathcal{H})$ is $X^{*}$. The standard \emph{Hilbert-Schmidt (HS) inner product} on $\mathcal{B}(\mathcal{H})$ is given by $\langle X,Y\rangle=\operatorname{tr}(X^{*}Y)$, with the induced norm $\|X\|=\sqrt{\langle X,X\rangle}$. A quantum state (or density operator) is an element $\rho\in\mathcal{B}(\mathcal{H})$ satisfying $\rho\succeq 0$ and $\operatorname{tr}(\rho)=1$. The set of all quantum states is $\mathcal{D}(\mathcal{H})$, and the subset of full-rank states is $\mathcal{D}^{+}(\mathcal{H})$. The adjoint of a superoperator $\Phi:\mathcal{B}(\mathcal{H})\to\mathcal{B}(\mathcal{H})$ with respect to the HS inner product is written as $\Phi^{\dagger}$. We use standard asymptotic notations $O(\cdot)$, $\Omega(\cdot)$, and $\Theta(\cdot)$.
%\end{notation}

\subsection{Quantum Markov Semigroups and Ergodicity}
\label{sec:qms_ergodicity}

A \textbf{Quantum Markov Semigroup (QMS)} $(\mathcal{P}_t)_{t\geq 0}$ is a semigroup of unital and completely positive (UCP) maps on $\mathcal{B}(\mathcal{H})$, corresponding to the Heisenberg picture of quantum dynamics. Its generator, the \textbf{Lindbladian}, is defined by $\mathcal{L}:=\lim_{t\downarrow 0}\frac{\mathcal{P}_t (X)-X}{t}$ and has the Gorini-Kossakowski-Sudarshan-Lindblad (GKSL) form:
\begin{align}
    \mathcal{L}(X)=i[H,X]+\sum_{\alpha}\left(L_\alpha^* X L_\alpha-\frac{1}{2}\{L_\alpha^* L_\alpha,X\}\right). \label{eq:lindbladian_heisenberg}
\end{align}
Here, $H$ is the self-adjoint system Hamiltonian, and $\{L_\alpha\}$ are the jump operators describing environmental interaction.

To characterize the long-time behavior of $\mathcal{P}_t=\exp(t\mathcal{L})$, we introduce two key subspaces: the \textbf{fixed point space} $\mathcal{F}(\mathcal{L}) = \ker(\mathcal{L})$ and the \textbf{decoherence-free sub-algebra} $\mathcal{N}(\mathcal{L}):=\{X\in\mathcal{B}(\mathcal{H}) \mid \mathcal{P}_{t}(X^{*}X)=\mathcal{P}_{t}(X)^{*}\mathcal{P}_{t}(X), \forall t\geq0\}$.
Since $\mathcal{F}(\mathcal{L})$ is a von Neumann subalgebra, there exists a unique completely positive and trace-preserving (CPTP) projection onto it, denoted by $\mathbb{E}_\mathcal{F}: \mathcal{B}(\mathcal{H}) \to \mathcal{F}(\mathcal{L})$, which is called the \textbf{conditional expectation}. Here, trace-preserving means that $\operatorname{tr}(\mathbb{E}_\mathcal{F}(X)) = \operatorname{tr}(X)$ for all $X \in \mathcal{B}(\mathcal{H})$, ensuring that the projection preserves both complete positivity and the trace functional.

\begin{lemma}[Ergodicity Criterion]
\label{lem:ergodicity_criterion}
    Let $\mathcal{P}_t$ be a QMS with an invariant state $\sigma\in\mathcal{D}^{+}(\mathcal{H})$. The semigroup converges to the conditional expectation onto the fixed points, $\lim_{t\to\infty}\mathcal{P}_t=\mathbb{E}_\mathcal{F}$, if and only if the decoherence-free subalgebra coincides with the fixed point space, i.e., $\mathcal{N}(\mathcal{L})=\mathcal{F}(\mathcal{L})$.
\end{lemma}
The proof can be found in \cite[Theorem 3.3 and 3.4]{Frigerio1982}. A QMS is called \textbf{ergodic} if this condition holds and \textbf{primitive} if it has a unique invariant state $\sigma$. For a primitive semigroup, the limit is $\mathbb{E}_\mathcal{F}(X)=\operatorname{tr}(\sigma X)\mathbf{1}$.

\begin{lemma}[Ergodic Spectrum]
\label{lem:ergodic_spectrum}
    An ergodic QMS is characterized by a generator $\mathcal{L}$ that has no purely imaginary eigenvalues, i.e.,  every non-zero eigenvalue has a strict negative real part.
\end{lemma}
This result, derived from \cite[Theorem 29 and Proposition 31]{Carbone2015}, ensures asymptotic stability in the sense that any initial state converges to the steady state subspace.

\subsection{Convergence Metrics and KMS Structure}
\label{sec:convergence_kms}

For a primitive QMS with a full-rank invariant state $\sigma\in\mathcal{D}^{+}(\mathcal{H})$, we define a family of weighted inner products on $\mathcal{B}(\mathcal{H})$ parameterized by $s\in[0,1]$:
\begin{align}
    \langle X,Y\rangle_{\sigma,s}:=\operatorname{tr}(\sigma^sX^*\sigma^{1-s}Y).
\end{align}
The case $s=1/2$ is known as the \textbf{Kubo-Martin-Schwinger (KMS)} inner product. The norm induced by the KMS inner product is written as $\|X\|_{2,\sigma} := \sqrt{\langle X, X \rangle_{\sigma, 1/2}}$. We denote the orthogonal complement of the fixed-point space $\mathcal{F}(\mathcal{L})$ with respect to the KMS inner product as $\mathcal{F}(\mathcal{L})^\perp$.

The convergence rate is determined by the ``gaps'' of the generator $\mathcal{L}$, as defined below.
\begin{definition}[Spectral and Singular Value Gaps]
\label{def:gaps}
    Let $\mathcal{L}$ be an ergodic generator with invariant state $\sigma$.
    \begin{enumerate}
        \item The \textbf{spectral gap} is the smallest real part of non-zero eigenvalues of $-\mathcal{L}$:
        \begin{align}
            \lambda(\mathcal{L}):=\inf\left\{\Re(\lambda) \mid \lambda\in\operatorname{Spec}(-\mathcal{L})\setminus\{0\}\right\}.
        \end{align}
        \item The \textbf{singular value gap} is the smallest non-zero singular value of $\mathcal{L}$ with respect to the KMS inner product:
        \begin{align}
            s(\mathcal{L}) := \inf \left\{ \| \mathcal{L}(X) \|_{2,\sigma} \mid X \in \mathcal{F}(\mathcal{L})^\perp, \|X\|_{2,\sigma} = 1 \right\}.
        \end{align}
    \end{enumerate}
\end{definition}

The spectral gap $\lambda(\mathcal{L})$ determines the sharp \textit{asymptotic} exponential rate of convergence to the steady state. The singular value gap $s(\mathcal{L})$, on the other hand, characterizes the instantaneous rate of dissipation for observables orthogonal to the fixed-point space. For a generator $\mathcal{L}$ that is self-adjoint with respect to the KMS inner product, these two gaps coincide: $\lambda(\mathcal{L}) = s(\mathcal{L})$. However, for non-self-adjoint generators, we only have the inequality $s(\mathcal{L}) \ge \lambda(\mathcal{L})$. The relation between $\lambda$ and $s$ is also discussed in \cite[Section 1.4]{Chatterjee2025} in the context of discrete Markov chains. We quantify the convergence speed using the following metrics:
\begin{definition}[Mixing and Relaxation Time]
\label{def:mixing_times}
    The $L^1$-\textbf{mixing time} is defined using the trace norm on states:
    \begin{align}
        t_{\mathrm{mix}}(\mathcal{L}):=\operatorname*{inf}\left\{t\geq0 \mid \sup_{\rho\in\mathcal{D}(\mathcal{H})} \|\mathcal{P}_{t}^{\dagger}(\rho)-\sigma\|_{\mathrm{tr}}\leq e^{-1} \right\},
    \end{align}
    and the $L^2$-\textbf{relaxation time} is defined using the KMS norm on observables:
    \begin{align}
        t_{\mathrm{rel}}(\mathcal{L}):=\inf\left\{t\geq0 \mid \sup_{X \in \mathcal{F}(\mathcal{L})^\perp} \frac{\|\mathcal{P}_{t}(X)\|_{2,\sigma}}{\|X\|_{2,\sigma}}\leq e^{-1} \right\}.
    \end{align}
\end{definition}
These two times are related by $t_{\mathrm{mix}}(\mathcal{L})\leq C(\sigma) t_{\mathrm{rel}}(\mathcal{L})$, where $C(\sigma)$ depends on the invariant state $\sigma$ \cite[Appendix B]{Li2025}. In this work, we will focus on analyzing the relaxation time $t_{\mathrm{rel}}$.

\subsection{Hypocoercivity}
\label{sec:hypocoercivity}

For a general ergodic QMS, the convergence to steady state is described by:
\begin{align}
    \|\mathcal{P}_{t}(X)-\mathbb{E}_{\mathcal{F}}(X)\|_{2,\sigma}\leq Ce^{-\nu t}\|X-\mathbb{E}_{\mathcal{F}}(X)\|_{2,\sigma}, \label{eq:hypo_def}
\end{align}
where $C\geq 1$ and $\nu>0$. The sharp asymptotic rate is $\nu = \lambda(\mathcal{L})$ \cite{Engel2000}.

\begin{definition}[Coercivity and Hypocoercivity]
\label{def:hypocoercivity}
    An ergodic QMS is \textbf{coercive} if \eqref{eq:hypo_def} holds with $C=1$. It is \textbf{hypocoercive} if it holds for $C \geq 1$, and \textbf{strictly hypocoercive} if $C>1$ is necessary.
\end{definition}

Coercivity implies $\lambda(\mathcal{L}) = s(\mathcal{L})$ and that all modes relax at a uniform exponential rate. Strict hypocoercivity arises when some modes are not directly dissipated ($s(\mathcal{L})$ might be large) but must be mixed by the oscillatory part of the dynamics into dissipative channels, resulting in a slower asymptotic rate $\lambda(\mathcal{L})$.

To characterize this mechanism, we define the adjoint of the Lindbladian $\mathcal{L}^\dagger$ with respect to the KMS inner product:
\begin{align}
    \langle X, \mathcal{L}Y \rangle_{\sigma, 1/2} = \langle \mathcal{L}^\dagger X, Y \rangle_{\sigma, 1/2}, \quad \forall X,Y \in \mathcal{B}(\mathcal{H}).
\end{align}
We then decompose $\mathcal{L}$ into its self-adjoint (dissipative) and skew-adjoint (oscillatory) parts relative to this KMS structure:
\begin{align}
    \mathcal{S} = \frac{\mathcal{L}+\mathcal{L}^\dagger}{2}, \quad \mathcal{A} = \frac{\mathcal{L}-\mathcal{L}^\dagger}{2}.
\end{align}
In a strictly hypocoercive system, there exist observables in $\ker(\mathcal{S})$ that are not steady states (i.e., not in $\mathcal{F}(\mathcal{L})$). The following lemma makes this precise. 

\begin{lemma}[Criterion for Strict Hypocoercivity]
\label{lem:kernel_collapse_hypo}
    For an ergodic QMS, the fixed-point space is a subspace of the dissipative kernel, $\mathcal{F}(\mathcal{L})\subset \ker(\mathcal{S})$. The semigroup is strictly hypocoercive if and only if this inclusion is strict, meaning there exists an element in the dissipative kernel that is not a fixed point:
    \begin{align}
        \ker(\mathcal{S}) \supsetneq \mathcal{F}(\mathcal{L}).
    \end{align}
    Equivalently, the codimension of $\mathcal{F}(\mathcal{L})$ in $\ker(\mathcal{S})$ is positive. This characterization applies to both finite and infinite-dimensional systems.
\end{lemma}
The proof follows the standard spectral analysis of the KMS decomposition; see Appendix~\ref{app:kernel_collapse_proof} for details.

Finally, the relaxation time and decay rate are fundamentally linked to the singular value gap, $s(\mathcal{L})$.
\begin{lemma}[Singular Value Gap Bounds]
\label{lem:singular_gap_bounds}
Let $s(\mathcal{L})$ be the singular value gap of an ergodic generator $\mathcal{L}$. The constants in \eqref{eq:hypo_def} satisfy:
\begin{align}
    \nu\leq(1+\log C)s(\mathcal{L}) \quad \text{and} \quad t_{\mathrm{rel}}\geq\frac{1}{2s(\mathcal{L})}.
\end{align}
\end{lemma}
A rigorous proof of this relationship can be found in \cite[Lemma 2.5]{Li2024}.

\subsection{Abstract Framework for Convergence Analysis}
\label{sec:abstract_semigroups}

For the convergence analysis in Section \ref{sec:ae_to_lifting}, we work in a more general abstract setting. We model the system's evolution using a \textbf{contractive strongly continuous semigroup} (or $C_0$-semigroup), denoted by $\{P_t\}_{t\ge 0}$, acting on a Hilbert space $\mathcal{H}$ with inner product $\langle\cdot,\cdot\rangle_\mathcal{H}$ and induced norm $\|\cdot\|_\mathcal{H}$. This family of operators satisfies the standard properties: (i) $P_0 = \operatorname{id}$, (ii) $P_s P_t = P_{s+t}$ for all $s, t \ge 0$, (iii) for every $X \in \mathcal{H}$, the map $t \mapsto P_t X$ is continuous, and (iv) $\|P_t\| \le 1$ for all $t \ge 0$.

The dynamics are governed by the \textbf{generator} $(L, \mathrm{Dom}(L))$, defined by $L X := \lim_{t\downarrow 0} (P_t X - X)/t$ for $X$ in its domain. Key properties include: (a) the generator is dissipative, $\Re\langle X, L X \rangle \le 0$; and (b) the set of fixed points $\mathcal{F} := \ker(L)$ forms a closed subspace, with the orthogonal complement $\mathcal{F}^\perp$ invariant under the dynamics. These are standard results in semigroup theory \cite{Engel2000}.

\paragraph{Convergence Modes.}
Assuming the semigroup converges to the steady-state projection $\mathbb{E}_{\mathcal{F}}$, we distinguish two modes:
\begin{enumerate}
    \item \textbf{Coercivity:} There exists $\nu > 0$ such that $\|P_t X - \mathbb{E}_{\mathcal{F}} X\|_{\mathcal{H}} \le e^{-\nu t} \|X - \mathbb{E}_{\mathcal{F}} X\|_{\mathcal{H}}$.
    \item \textbf{Hypocoercivity:} There exist $\nu > 0$ and $C \ge 1$ such that $\|P_t X - \mathbb{E}_{\mathcal{F}} X\|_{\mathcal{H}} \le C e^{-\nu t} \|X - \mathbb{E}_{\mathcal{F}} X\|_{\mathcal{H}}$.
\end{enumerate}
Strict hypocoercivity ($C > 1$) arises when the generator $L = \mathcal{S} + \mathcal{A}$ has dissipative ($\mathcal{S}$) and conservative ($\mathcal{A}$) parts with $\ker(\mathcal{S}) \supsetneq \ker(L)$. Convergence then relies on $\mathcal{A}$ mixing undamped modes into damped ones \cite{Villani2009}.

\paragraph{The Singular Value Gap.}
A key quantity for analyzing hypocoercive convergence is the singular value gap:
\begin{align}
    s(L) := \inf \left\{ \|L X\|_{\mathcal{H}} \mid X \in \mathrm{Dom}(L) \cap \mathcal{F}^{\perp}, \|X\|_{\mathcal{H}} = 1 \right\}.
\end{align}
This measures the minimum action of $L$ on states orthogonal to its kernel. For a hypocoercive semigroup with optimal rate $\nu$ and prefactor $C$, the decay rate is bounded by $\nu \le (1 + \log C) s(L)$ (see Appendix~\ref{Singular value gap bounds decay exponent--proof}).

\section{Mathematical Setup and Structural Assumptions}
\label{sec:mathematical_setup}

We now establish the mathematical framework for our stability analysis. Rather than deriving effective dynamics from first principles, we adopt a direct axiomatic approach: we specify the structure of the full generator and state the algebraic properties that enable our perturbative bounds. This section defines the operators, introduces the reference slow generator, and formalizes the geometric assumptions that control spectral stability under non-equilibrium perturbations.

\subsection{The Perturbed Generator}
\label{sec:perturbed_generator}

We consider a quantum Markov semigroup (QMS) with generator $\mathcal{L}$ acting on $\mathcal{B}(\mathcal{H})$, where $\mathcal{H}$ is a finite-dimensional Hilbert space. The generator exhibits a clear separation of time scales and admits the decomposition:
\begin{align}
 \mathcal{L} = \gamma \mathcal{R} + \mathcal{V}, \label{eq:gen_decomp}
\end{align}
where $\gamma \gg 1$ is a dimensionless parameter governing the time-scale separation. The fast generator $\mathcal{R}$ is a self-adjoint, dissipative superoperator satisfying $\ker(\mathcal{R}) = \mathcal{H}_S$ (the \textbf{slow subspace}) and possessing a strictly positive spectral gap $\lambda_R > 0$ on the orthogonal complement $\mathcal{H}_F := \mathcal{H}_S^\perp$ (the \textbf{fast subspace}). We denote the orthogonal projections onto these subspaces by $P_S$ and $P_F := \mathbf{1} - P_S$, respectively.

The coupling generator $\mathcal{V}$ contains the remaining dynamics and is further decomposed as:
\begin{align}
\mathcal{V} = \mathcal{L}_{\ham} + \eta \mathcal{L}_{\pert}, \label{eq:V_decomp}
\end{align}
where $\mathcal{L}_{\ham}$ is a skew-adjoint Hamiltonian contribution and $\mathcal{L}_{\pert}$ is a \textbf{non-conservative perturbation}---a bounded operator that breaks the detailed balance of the unperturbed generator $\gamma\mathcal{R} + \mathcal{L}_{\ham}$. The parameter $\eta > 0$ quantifies the strength of this symmetry-breaking perturbation. Throughout this work, $\eta$ is treated as a \textbf{fixed small parameter}, and our bounds will be explicit in their dependence on $\eta$. The regime where optimal acceleration persists will be characterized by the scaling $\eta \ll s(L_O)^2$ (see Section \ref{sec:convergence_bounds}).

\begin{remark}[Physical Interpretation]
\label{rem:physical_interpretation}
The decomposition \eqref{eq:gen_decomp}--\eqref{eq:V_decomp} arises naturally in systems with timescale separation. In the overdamped limit of Langevin dynamics \cite{Monmarche2022,Iacobucci2017}, $\mathcal{R}$ represents momentum dissipation ($\gamma \to \infty$), while $\mathcal{V}$ encodes position-dependent forces. In the quantum Zeno limit \cite{Popkov2018,Popkov2021}, strong dissipation ($\gamma \to \infty$) confines the system to a decoherence-free subspace $\mathcal{H}_S = \ker(\mathcal{R})$, with $\mathcal{V}$ describing coherent tunneling and weak symmetry-breaking baths. The non-conservative perturbation $\eta\mathcal{L}_{\pert}$ models external driving fields or boundary reservoirs at different chemical potentials, producing nonequilibrium steady states (NESS).
\end{remark}

\subsection{The Reference Slow Generator}
\label{sec:reference_generator}

To analyze the convergence of the full dynamics $e^{t\mathcal{L}}$, we require a reference generator describing the effective slow dynamics within $\mathcal{H}_S$. This reference is provided by the Generalized Schrieffer-Wolff (GSW) formalism \cite{Kessler2012}, a systematic perturbative framework for deriving effective generators in systems with timescale separation. The GSW method constructs a similarity transformation $e^{\mathcal{S}}$ that block-diagonalizes the rescaled generator $\mathcal{L}' = \mathcal{R} + (1/\gamma)\mathcal{V}$, yielding an effective generator on $\mathcal{H}_S$ accurate to all orders in $1/\gamma$.

For our purposes, we work directly with the \textbf{second-order truncation}, which captures the dominant slow dynamics when the first-order term $P_S \mathcal{V} P_S$ vanishes (a common scenario enforced by symmetry). Following the GSW expansion (see \cite{Kessler2012} or Appendix \ref{app:proof_thm_sw_expansion} for details), the effective generator takes the form:
\begin{align}
    L_O := -P_S \mathcal{V} \mathcal{R}^+ \mathcal{V} P_S, \label{eq:L_O_def}
\end{align}
where $\mathcal{R}^+ := (P_F \mathcal{R} P_F)^{-1}$ denotes the pseudo-inverse of $\mathcal{R}$ restricted to the fast subspace $\mathcal{H}_F$. This operator $L_O$ acts on $\mathcal{H}_S$ and inherits an $\eta$-dependence through $\mathcal{V}$ (see \eqref{eq:V_decomp}).

\begin{remark}[GSW Formalism and GKSL Form]
\label{rem:gsw_gksl}
The GSW transformation $e^{\mathcal{S}}$ is a similarity transformation but not unitary. Consequently, $L_O$ does not generally preserve the Gorini-Kossakowski-Sudarshan-Lindblad (GKSL) form \cite{Gorini1976}. However, in many physical systems (e.g., the examples in Section \ref{sec:examples}), $L_O$ does generate a valid QMS. For our abstract framework, we assume when necessary that $L_O$ is a legitimate Lindbladian. Rigorous error bounds for the GSW approximation are discussed in \cite{Maimbourg2021,Salzmann2024}.
\end{remark}

The generator $L_O$ serves as the \textbf{reference slow generator} against which we measure the convergence of the full dynamics. The central question of this work is: Under what conditions on $\eta$ does the full generator $\mathcal{L}$ achieve the same optimal convergence rate as $L_O$, despite the breaking of detailed balance?

\subsection{Geometric Assumptions: The Near-Equilibrium Structure}
\label{sec:geometric_assumptions}

We now formalize the algebraic structure that enables our perturbative stability analysis. This structure, which we call the \textbf{near-equilibrium lifting structure}, consists of three key assumptions. These are not derived consequences but rather the \textit{axiomatic foundation} for our convergence bounds.

\paragraph{Inner Product Convention.}
For the analysis, we fix a Hilbert space structure independent of $\gamma$. Let $\sigma_O$ denote the (fixed, $\eta$-dependent) NESS of the reference generator $L_O$, and let $\sigma_{\mathcal{R}}$ be a fixed invariant state of the fast dynamics (e.g., the Maxwell-Boltzmann distribution). We define the reference inner product on $\mathcal{H} = \mathcal{H}_S \oplus \mathcal{H}_F$ using the product state $\sigma_\infty := \sigma_O \otimes \sigma_{\mathcal{R}}$:
\begin{align}
    \langle X, Y \rangle_{\mathcal{H}} := \langle X, Y \rangle_{\sigma_\infty, 1/2} = \operatorname{tr}(\sigma_\infty^{1/2} X^* \sigma_\infty^{1/2} Y).
\end{align}
This is the KMS inner product with respect to $\sigma_\infty$. For observables $X, Y \in \mathcal{H}_S$, we use the analogous inner product on the slow subspace:
\begin{align}
    \langle X, Y \rangle_{\mathcal{H}_S} := \langle X, Y \rangle_{\sigma_O, 1/2}.
\end{align}
The product structure $\sigma_\infty = \sigma_O \otimes \sigma_{\mathcal{R}}$ holds in the adiabatic regime $\gamma \to \infty$ by timescale separation \cite{Carlen2024,Monmarche2022}. We denote the induced norm by $\|\cdot\|$.

\begin{assumption}[Orthogonality Condition]
\label{assump:orthogonality}
The coupling $\mathcal{V}$ maps the slow subspace into the fast subspace, i.e., $P_S \mathcal{V} P_S = 0$. Equivalently, for all $X, Y \in \mathcal{H}_S$:
\begin{align}
    \langle X, \mathcal{V} Y \rangle_{\mathcal{H}} = 0.
\end{align}
\end{assumption}
This condition is the algebraic signature of vanishing first-order terms in the GSW expansion. Physically, it reflects symmetry constraints (e.g., parity) that eliminate direct slow-to-slow transitions.

\begin{assumption}[Structural Prerequisites]
\label{assump:structural_primitives}
The operators $\mathcal{R}$, $\mathcal{V}$, and $L_O$ satisfy the following properties:
\begin{enumerate}[(a)]
    \item \textbf{Fast Dissipation:} The operator $\mathcal{R}$ is self-adjoint with $\ker(\mathcal{R}) = \mathcal{H}_S$. Its restriction to $\mathcal{H}_F$ is strictly dissipative: $-\mathcal{R}|_{\mathcal{H}_F} \ge \lambda_R P_F$ for some $\lambda_R > 0$. The pseudo-inverse $S := (-\mathcal{R}|_{\mathcal{H}_F})^{-1}$ satisfies $\|S\| = \lambda_R^{-1}$.
    \item \textbf{Coupling Decomposition:} The coupling admits $\mathcal{V} = \mathcal{L}_{\ham} + \eta \mathcal{L}_{\pert}$, where $\mathcal{L}_{\ham}$ is skew-adjoint and $\mathcal{L}_{\pert}$ is bounded.
    \item \textbf{Reference Spectral Gap:} The reference generator $L_O = -P_S \mathcal{V} S \mathcal{V} P_S$ admits a non-trivial singular value gap $s(L_O) > 0$ on $\ker(L_O)^\perp \cap \mathcal{H}_S$.
    \item \textbf{Kernel Consistency:} Denote $Q_O := -P_S \mathcal{L}_{\ham} S \mathcal{L}_{\ham} P_S$ (the unperturbed reference generator at $\eta = 0$). We require $\dim\ker(L_O) = \dim\ker(Q_O)$, ensuring the perturbation does not alter the ergodicity class.
\end{enumerate}
\end{assumption}

The following assumption is the \textbf{central mathematical tool} for controlling non-equilibrium perturbations. It bounds the deviation of the quadratic form associated with the full generator from that of the reference generator, quantifying the ``non-normality'' introduced by $\eta$.

\begin{assumption}[Approximate Quadratic Form (AQF)]
\label{assump:AQF}
Under Assumptions \ref{assump:orthogonality} and \ref{assump:structural_primitives}, there exists a constant $C_{AQF} \ge 0$ such that for all $X, Y \in \mathcal{H}_S$:
\begin{align}
    \left| \langle \mathcal{L} X, S \mathcal{L} Y \rangle_{\mathcal{H}} - \langle X, |L_O| Y \rangle_{\mathcal{H}_S} \right| \le \eta C_{AQF} \|X\| \|Y\|, \label{eq:AQF_condition}
\end{align}
where $|L_O| := \sqrt{L_O^\dagger L_O}$ is the operator magnitude of the reference generator.
\end{assumption}

\begin{remark}[Interpretation of AQF]
\label{rem:AQF_interpretation}
The Approximate Quadratic Form condition \eqref{eq:AQF_condition} is the perturbative analogue of the exact quadratic form identity satisfied by reversible lifts \cite{Eberle2024a,Li2025}. In the equilibrium case ($\eta = 0$), the identity $\langle \mathcal{L} X, S \mathcal{L} Y \rangle = \langle X, |Q_O| Y \rangle$ holds exactly, enabling direct application of hypocoercive estimates. The AQF quantifies the $O(\eta)$ deviation caused by symmetry breaking. The constant $C_{AQF}$ depends on operator norms and the reference gap $s(L_O)$; its explicit form is derived in Proposition \ref{prop:quad_form_condition_ae_approx} (see Section \ref{sec:ae_to_lifting} for the detailed proof).
\end{remark}

Together, Assumptions \ref{assump:orthogonality}--\ref{assump:AQF} define the \textbf{near-equilibrium lifting structure}. This structure abstracts the essential geometric properties arising from adiabatic elimination and provides the foundation for our Flow Poincaré inequality and convergence bounds (Section \ref{sec:convergence_bounds}).

\begin{remark}[Explicit Form of $C_{AQF}$]
\label{rem:explicit_AQF}
The constant $C_{AQF}$ in Assumption \ref{assump:AQF} can be computed explicitly from the operator norms and spectral gaps. Following the perturbation theory developed in Proposition \ref{prop:quad_form_condition_ae_approx} (proved in Section \ref{sec:ae_to_lifting}), we have:
\begin{align}
    C_{AQF} = K_{lin} + \eta K_{quad} + \frac{K_{sq}}{\sqrt{2} s(L_O)},
\end{align}
where the dimensionless perturbation constants are defined by:
\begin{align}
    K_{lin} &:= 2 \|\mathcal{L}_{\ham}\| \lambda_R^{-1} \|\mathcal{L}_{\pert}\|, \\
    K_{quad} &:= \|\mathcal{L}_{\pert}\|^2 \lambda_R^{-1}, \\
    K_{sq} &:= 2 \|Q_O\| (K_{lin} + K_{quad}) + (K_{lin} + K_{quad})^2,
\end{align}
with $Q_O = -P_S \mathcal{L}_{\ham} S \mathcal{L}_{\ham} P_S$ the unperturbed reference generator. This explicit form shows that $C_{AQF} = O(s(L_O)^{-1})$ in the regime $\eta \ll s(L_O)^2$, which is the critical scaling for optimal acceleration.
\end{remark}

\section{From Adiabatic Elimination to Lifting}
\label{sec:ae_to_lifting}

Having established the structural assumptions in Section \ref{sec:mathematical_setup}, we now develop the connection between the full generator $\mathcal{L}$ and the reference generator $L_O$. This section proceeds in two stages. First, we prove that the Approximate Quadratic Form condition (Assumption \ref{assump:AQF}) holds under the structural prerequisites, providing an explicit formula for the constant $C_{AQF}$. Second, we formalize the notion of a \textbf{near-equilibrium lifting structure}, which synthesizes all the geometric assumptions into a unified framework for convergence analysis.
\subsection{Proof of the Approximate Quadratic Form}
\label{sec:proof_AQF}

We now establish that the Approximate Quadratic Form condition (Assumption \ref{assump:AQF}) holds under the structural prerequisites (Assumption \ref{assump:structural_primitives}), providing an explicit formula for the constant $C_{AQF}$. This result quantifies the deviation from the exact quadratic form identity satisfied by reversible systems, treating the non-equilibrium perturbation as a controlled error proportional to $\eta$.

\begin{proposition}[Approximate Quadratic Form]
\label{prop:quad_form_condition_ae_approx}
Let the operators satisfy Assumptions \ref{assump:orthogonality} and \ref{assump:structural_primitives}. Define the dimensionless perturbation constants:
\begin{align}
    K_{lin} &\coloneqq 2 \|\mathcal{L}_{\ham}\| \lambda_R^{-1} \|\mathcal{L}_{\pert}\|, \\
    K_{quad} &\coloneqq \|\mathcal{L}_{\pert}\|^2 \lambda_R^{-1}.
\end{align}
Let $Q_0 \coloneqq -P_S \mathcal{L}_{\ham} S \mathcal{L}_{\ham} P_S$. Define the operator square deviation constant:
\begin{equation}
    K_{sq} \coloneqq 2 \|Q_0\| (K_{lin} + K_{quad}) + (K_{lin} + K_{quad})^2.
\end{equation}
Then, for all vectors $X, Y \in \mathcal{H}_S$:
\begin{equation}
    \left| \langle \mathcal{L} X, S \mathcal{L} Y \rangle_{\mathcal{H}} - \langle X, |L_O| Y \rangle_{\mathcal{H}_S} \right| \le \eta\, C_{AQF} \|X\|_{\mathcal{H}} \|Y\|_{\mathcal{H}},
\end{equation}
where the constant $C_{AQF}$ is given explicitly by:
\begin{equation}
    C_{AQF} \coloneqq K_{lin} + \eta K_{quad} + \frac{K_{sq}}{\sqrt{2} s(L_O)}.
\end{equation}
\end{proposition}

\begin{proof}
The proof proceeds by bounding the deviation of the bilinear form on the full space and then utilizing the spectral stability of the reference generator to bound the operator square root.

Let $X, Y \in \mathcal{H}_S$. Since $\mathcal{H}_S = \ker(\mathcal{R})$, we have $\mathcal{L} X = \mathcal{V}X$. Expanding the coupling term yields:
\begin{equation}
    \langle \mathcal{L} X, S \mathcal{L} Y \rangle_{\mathcal{H}} = \langle (\mathcal{L}_{\ham} + \eta \mathcal{L}_{\pert})X, S(\mathcal{L}_{\ham} + \eta \mathcal{L}_{\pert})Y \rangle_{\mathcal{H}}.
\end{equation}
We separate this into the unperturbed term and an error term $E_\eta(X,Y)$:
\begin{equation}
    \langle \mathcal{L} X, S \mathcal{L} Y \rangle_{\mathcal{H}} = \langle \mathcal{L}_{\ham}X, S\mathcal{L}_{\ham}Y \rangle_{\mathcal{H}} + E_\eta(X,Y),
\end{equation}
where $E_\eta(X,Y)$ collects all terms dependent on $\eta$. Using the Cauchy-Schwarz inequality and that $\|S\|=\lambda_R^{-1}$:
\begin{align}
    |E_\eta(X,Y)| &\le \eta \left( 2 \|\mathcal{L}_{\ham}\| \lambda_R^{-1} \|\mathcal{L}_{\pert}\| + \eta \|\mathcal{L}_{\pert}\|^2 \lambda_R^{-1} \right) \|X\|_{\mathcal{H}} \|Y\|_{\mathcal{H}} \\
    &= \eta (K_{lin} + \eta K_{quad}) \|X\|_{\mathcal{H}} \|Y\|_{\mathcal{H}}.
\end{align}
The unperturbed term corresponds to the quadratic form of the \textit{unperturbed reference generator} $Q_0 := -P_S \mathcal{L}_{\ham} S \mathcal{L}_{\ham} P_S$. Since $\mathcal{L}_{\ham}$ is skew-adjoint and $S$ is self-adjoint positive:
\begin{equation}
    \langle \mathcal{L}_{\ham}X, S\mathcal{L}_{\ham}Y \rangle_{\mathcal{H}} = \langle X, (-P_S \mathcal{L}_{\ham} S \mathcal{L}_{\ham} P_S) Y \rangle_{\mathcal{H}_S} = \langle X, Q_0 Y \rangle_{\mathcal{H}_S}.
\end{equation}
Since $Q_0$ is positive semi-definite, $Q_0 = |Q_0|$. Thus:
\begin{equation}
    \label{eq:bilinear_bound_final}
    \left| \langle \mathcal{L} X, S \mathcal{L} Y \rangle_{\mathcal{H}} - \langle X, |Q_0| Y \rangle_{\mathcal{H}_S} \right| \le \eta (K_{lin} + \eta K_{quad}) \|X\|_{\mathcal{H}} \|Y\|_{\mathcal{H}}.
\end{equation}

We define positive operators $A \coloneqq L_O^\dagger L_O$ and $B \coloneqq Q_0^2$. The target operator is $|L_O| = \sqrt{A}$, and the unperturbed approximation is $|Q_0| = \sqrt{B}$.
Note that when $\eta = 0$, we have $L_O = Q_0$ (both equal to $-P_S \mathcal{L}_{\ham} S \mathcal{L}_{\ham} P_S$). The difference in squared operators, $A - B$, arises from the perturbation $\Delta_\eta := L_O - Q_0$. A straightforward calculation shows that 
\begin{equation}
    \|A - B\| \le \eta \bigl( 2  \|Q_0\| (K_{lin} + K_{quad}) + (K_{lin} + K_{quad})^2 \bigr) = \eta K_{sq}.
\end{equation}
We now analyze the spectra to apply the Lipschitz continuity of the square root. By Assumption \ref{assump:structural_primitives}(c), the non-zero spectrum of $A$ is bounded below by $s(L_O)^2$:
\begin{equation}
    \sigma(A) \setminus \{0\} \subseteq [s(L_O)^2, \infty).
\end{equation}
Using Weyl's inequality and the Kernel Consistency assumption, the non-zero spectrum of the unperturbed operator $B$ satisfies:
\begin{equation}
    \min(\sigma(B) \setminus \{0\}) \ge \min(\sigma(A) \setminus \{0\}) - \|A - B\| \ge s(L_O)^2 - \eta K_{sq}.
\end{equation}
Given the condition $\eta \le \frac{s(L_O)^2}{2 K_{sq}}$, we have $\eta K_{sq} \le \frac{1}{2} s(L_O)^2$. Therefore:
\begin{equation}
    \min(\sigma(B) \setminus \{0\}) \ge \frac{1}{2} s(L_O)^2.
\end{equation}

 Starting from the representation $\sqrt{A} = \frac{1}{\pi} \int_0^\infty t^{-1/2} A (t+A)^{-1} \, dt$ and the identity $A(t+A)^{-1} = I - t(t+A)^{-1}$, we express the difference as
 \begin{equation}
    \begin{aligned}
     \sqrt{A} - \sqrt{B} & = \frac{1}{\pi} \int_0^\infty \sqrt{t} \left[ (B+t)^{-1} - (A+t)^{-1} \right] \, dt \\
     & = \frac{1}{\pi} \int_0^\infty \sqrt{t} \, (B+t)^{-1} (A-B) (A+t)^{-1} \, dt,
     \end{aligned}
 \end{equation}
 where the last line used the resolvent identity $(B+t)^{-1} - (A+t)^{-1} = (B+t)^{-1} (A-B) (A+t)^{-1}$.
Assuming the spectra of $A$ and $B$ (excluding zero) are contained in $[\mu, \infty)$ with $\mu > 0$, the resolvent norms satisfy $\|(A+t)^{-1}\|, \|(B+t)^{-1}\| \le (t+\mu)^{-1}$. Taking the norm of the integral implies $\|\sqrt{A} - \sqrt{B}\| \le \frac{\|A-B\|}{\pi} \int_0^\infty \frac{\sqrt{t}}{(t+\mu)^2} \, dt$. Evaluating the definite integral as $\int_0^\infty \frac{\sqrt{t}}{(t+\mu)^2} \, dt = \frac{\pi}{2\sqrt{\mu}}$, we obtain the final Lipschitz bound $\|\sqrt{A} - \sqrt{B}\| \le \frac{1}{2\sqrt{\mu}} \|A-B\|$.

 This argument, based on operator monotone function theory (see, e.g., \cite[Chapter V]{Bhatia1997}), is valid in both finite and infinite dimensions provided the spectral gap condition $\mu > 0$ holds. Thus, we have the estimate 
\begin{equation}
    \bigl\lVert |L_O| - |Q_0| \bigr\rVert = \| \sqrt{A} - \sqrt{B} \| \le \frac{1}{\sqrt{2} s(L_O)} \|A - B\| \le \eta \frac{K_{sq}}{\sqrt{2} s(L_O)}.
\end{equation}

We combine the bilinear bound \eqref{eq:bilinear_bound_final} and the operator root bound via the triangle inequality:
\begin{align}
    \left| \langle \mathcal{L} X, S \mathcal{L} Y \rangle_{\mathcal{H}} - \langle X, |L_O| Y \rangle_{\mathcal{H}_S} \right| 
    &\le \left| \langle \mathcal{L} X, S \mathcal{L} Y \rangle_{\mathcal{H}} - \langle X, |Q_0| Y \rangle_{\mathcal{H}_S} \right| + \left| \langle X, (|Q_0| - |L_O|) Y \rangle_{\mathcal{H}_S} \right| \\
    &\le \eta (K_{lin} + \eta K_{quad}) \|X\|_{\mathcal{H}} \|Y\|_{\mathcal{H}} + \eta \frac{K_{sq}}{\sqrt{2} s(L_O)} \|X\|_{\mathcal{H}} \|Y\|_{\mathcal{H}} \\
    &= \eta \left( K_{lin} + \eta K_{quad} + \frac{K_{sq}}{\sqrt{2} s(L_O)} \right) \|X\|_{\mathcal{H}} \|Y\|_{\mathcal{H}}.
\end{align}
We finish the proof by defining the term in parentheses as $C_{AQF}$.
\end{proof}


\begin{remark}[Extension to Infinite Dimensions]
\label{rem:infinite_dim}
The framework extends naturally to infinite-dimensional Hilbert spaces under appropriate spectral conditions. The key requirement is a uniform spectral gap: both $\mathcal{R}_F = P_F \mathcal{R} P_F$ and $L_O$ must have spectra (excluding zero) bounded away from the origin. This ensures: (i) the pseudo-inverse $\mathcal{R}^+$ exists as a bounded operator, (ii) the perturbative expansion converges, and (iii) the Lipschitz estimate for the operator square root holds via operator monotone function theory \cite{Bhatia1997}. In Section \ref{sec:example_langevin}, we apply this to classical Langevin dynamics in $\mathbb{R}^d$, where $\mathcal{H} = L^2(\mathbb{R}^{2d}, d\widehat{\mu})$ with unbounded operators satisfying the requisite spectral conditions.
\end{remark}

\subsection{The Near-Equilibrium Lifting Structure}
\label{sec:ad_embed_structure}

We now synthesize the geometric assumptions (Section \ref{sec:geometric_assumptions}) and the Approximate Quadratic Form (Proposition \ref{prop:quad_form_condition_ae_approx}) into a unified framework for analyzing the full generator $\mathcal{L} = \gamma\mathcal{R} + \mathcal{V}$ as a lift of the reference generator $L_O$.

\begin{definition}[Near-Equilibrium Lifting Structure]
\label{def:ae_lift_structure}
Let $\mathcal{L} = \gamma\mathcal{R} + \mathcal{V}$ be a generator on a Hilbert space $\mathcal{H}$ (finite or infinite-dimensional). We say that $\mathcal{L}$ exhibits a \textbf{near-equilibrium lifting structure} if it satisfies Assumptions \ref{assump:orthogonality}--\ref{assump:AQF}, namely:
\begin{enumerate}
    \item \textbf{Orthogonality:} The coupling $\mathcal{V}$ maps $\mathcal{H}_S = \ker(\mathcal{R})$ into $\mathcal{H}_F = \mathcal{H}_S^\perp$, i.e., $\langle X, \mathcal{V} Y \rangle_\mathcal{H} = 0$ for all $X, Y \in \mathcal{H}_S$.
    \item \textbf{Structural Prerequisites:} The operators satisfy:
        \begin{enumerate}[(a)]
            \item Fast dissipation: $-\mathcal{R}|_{\mathcal{H}_F} \ge \lambda_R P_F$ for some $\lambda_R > 0$.
            \item Coupling decomposition: $\mathcal{V} = \mathcal{L}_{\ham} + \eta \mathcal{L}_{\pert}$ with $\mathcal{L}_{\ham}$ skew-adjoint.
            \item Reference spectral gap: $s(L_O) > 0$ for $L_O = -P_S \mathcal{V} S \mathcal{V} P_S$ (with $S = (-\mathcal{R}|_{\mathcal{H}_F})^{-1}$).
            \item Kernel consistency: $\dim\ker(L_O) = \dim\ker(Q_O)$ (where $Q_O = -P_S \mathcal{L}_{\ham} S \mathcal{L}_{\ham} P_S$).
        \end{enumerate}
    \item \textbf{Approximate Quadratic Form:} There exists $C_{AQF} \ge 0$ (given explicitly in Proposition \ref{prop:quad_form_condition_ae_approx}) such that for all $X, Y \in \mathcal{H}_S$:
    \begin{align}
        \left| \langle \mathcal{L} X, S \mathcal{L} Y \rangle_{\mathcal{H}} - \langle X, |L_O| Y \rangle_{\mathcal{H}_S} \right| \le \eta C_{AQF} \|X\| \|Y\|.
    \end{align}
\end{enumerate}
Additionally, we require that the steady states coincide, $\ker(\mathcal{L}) = \ker(L_O) \subset \mathcal{H}_S$, and that the reference generator admits hypocoercive decay: $\|e^{t L_O} X\| \le C_O e^{-\nu_O t} \|X\|$ for $X \in \ker(L_O)^\perp$.
\end{definition}

This definition synthesizes the algebraic structure arising from adiabatic elimination into a minimal set of assumptions. The Orthogonality and Structural Prerequisites ensure the basic geometry, while the Approximate Quadratic Form quantifies the controlled deviation from the reversible case. Together, these conditions enable the Flow Poincaré inequality analysis in Section \ref{sec:convergence_bounds}, yielding explicit bounds on the convergence rate $\nu(\mathcal{L})$ in terms of the reference gap $s(L_O)$ and the perturbation strength $\eta$.



\section{Perturbative Stability of Convergence Rates}
\label{sec:convergence_bounds}

We now establish the central stability results of this work. Given a generator $\mathcal{L} = \gamma\mathcal{R} + \mathcal{V}$ with the near-equilibrium lifting structure (Definition \ref{def:ae_lift_structure}), we prove that the convergence rate remains robustly close to the optimal value $\Theta(\sqrt{s(L_O)})$ provided the non-conservative perturbation $\eta$ remains below a critical threshold. The analysis proceeds in two stages: first, we demonstrate spectral continuity via an upper bound (Section \ref{sec:upper_bound_stability}), showing that the full generator's gap does not collapse under small perturbations; second, we establish the Flow Poincaré inequality (Section \ref{sec:flow_poincare_stability}), rigorously quantifying how the Approximate Quadratic Form controls the error induced by symmetry breaking and deriving the explicit stability criterion $\eta < C_{corr}^{-1}$.

\subsection{Spectral Continuity: Upper Bound on the Convergence Rate}
\label{sec:upper_bound_stability}

The first component of our stability analysis addresses the question: \textit{Does the spectral gap of the full generator $\mathcal{L}$ remain bounded away from zero under non-equilibrium perturbations?} The following theorem establishes an affirmative answer by proving continuity of the singular value gap with respect to $\eta$.

The exponential convergence rate $\nu(\mathcal{L})$ governs the asymptotic decay of the full semigroup. By Lemma \ref{lem:singular_gap_bounds}, this rate is bounded by the singular value gap $s(\mathcal{L})$ via the hypocoercivity constant $C(\mathcal{L})$ as $\nu(\mathcal{L}) \le (1 + \log C(\mathcal{L})) s(\mathcal{L})$. Our strategy is to bound $s(\mathcal{L})$ in terms of the reference generator $L_O$ and the perturbation strength $\eta$, using the Approximate Quadratic Form condition.

\begin{theorem}[Spectral Continuity Under Perturbation]
\label{thm:upper_bound_convergence_ae}
    Let $\{P_t\}_{t\geq 0}$ be a hypocoercive $C_0$-semigroup on $\mathcal{H}$ with generator $\mathcal{L} = \gamma\mathcal{R} + \mathcal{V}$ satisfying the near-equilibrium lifting structure (Definition \ref{def:ae_lift_structure}). Let $L_O$ be the reference generator acting on $\mathcal{H}_S = \ker(\mathcal{R})$ with singular value gap $s(L_O) > 0$. Let $S = (-\mathcal{R}|_{\mathcal{H}_F})^{-1}$ and define $\Pi_1$ as the orthogonal projection onto $\overline{\operatorname{Ran}(\mathcal{V}|_{\mathcal{H}_S})} \subseteq \mathcal{H}_F$.
    
    Then, the singular value gap of the full generator satisfies the explicit bound:
    \begin{align}
        s(\mathcal{L}) \le \sqrt{\frac{s(L_O) + \eta C_{AQF}}{s(\Pi_1 S \Pi_1)}},
    \end{align}
    where $C_{AQF}$ is the constant from the Approximate Quadratic Form (Assumption \ref{assump:AQF}), and $s(\Pi_1 S \Pi_1)$ is the smallest eigenvalue of the positive operator $\Pi_1 S \Pi_1$ restricted to $\operatorname{Ran}(\Pi_1)$.
    
    Consequently, the exponential convergence rate is bounded by:
    \begin{align}
        \nu(\mathcal{L}) \leq (1+\log C(\mathcal{L})) \sqrt{\frac{s(L_O) + \eta C_{AQF}}{s(\Pi_1 S \Pi_1)}}.
    \end{align}
\end{theorem}
\begin{proof}
    The proof involves bounding the singular value gap $s(\mathcal{L})$ by restricting the infimum defining it to the slow subspace $\mathcal{H}_S$. The Approximate Quadratic Form condition (Assumption \ref{assump:AQF}) relates the quadratic form $\langle \mathcal{L} X, S \mathcal{L} Y \rangle_{\mathcal{H}}$ to $\langle X, |L_O| Y \rangle_{\mathcal{H}_S}$ with an explicit error term bounded by $\eta C_{AQF}$. By applying this bound and the spectral properties of $S$, we obtain the stated inequality. The detailed derivation is provided in Appendix \ref{app:upper_bound_nu_proof}.
\end{proof}

\begin{remark}[Interpretation: Spectral Stability]
\label{remark:spectral_stability}
This theorem establishes \textbf{spectral continuity}: the singular value gap $s(\mathcal{L})$ of the perturbed system remains within a controlled neighborhood of $\sqrt{s(L_O)}$, with the deviation explicitly bounded by the perturbation strength $\eta$. The key insight is that the Approximate Quadratic Form condition bounds the "damage" to the quadratic form identity caused by breaking detailed balance.

Using the scaling $C_{AQF} = \Theta(s(L_O)^{-1})$ (Proposition \ref{prop:quad_form_condition_ae_approx}), the effective spectral parameter becomes:
\begin{equation}
    s(\mathcal{L}) \lesssim \sqrt{s(L_O) + \eta C_{AQF}} = \sqrt{s(L_O) + O(\eta s(L_O)^{-1})}.
\end{equation}
For the perturbative regime $\eta \ll s(L_O)^2$, the correction term $\eta s(L_O)^{-1} = o(s(L_O))$ is negligible, yielding:
\begin{equation}
    s(\mathcal{L}) = \Theta(\sqrt{s(L_O)}) \quad \text{as } \eta/s(L_O)^2 \to 0.
\end{equation}
This demonstrates that the optimal diffusive-to-ballistic acceleration ($s \sim 1/\gamma \rightsquigarrow s(\mathcal{L}) \sim \gamma^{-1/2}$) is \textit{structurally stable} under small symmetry-breaking perturbations. The critical threshold $\eta_{\text{crit}} = \Theta(s(L_O)^2)$ emerges naturally: beyond this scale, the perturbation can close the spectral gap, invalidating the lifting framework.
\end{remark}

\subsection{Flow Poincaré Inequality and the Stability Criterion}
\label{sec:flow_poincare_stability}

The upper bound established spectral continuity but does not guarantee that the convergence rate achieves the optimal scaling. To prove optimality, we must demonstrate that the full generator satisfies a \textbf{Flow Poincaré inequality} with a positive constant that scales as $\Theta(\sqrt{s(L_O)})$. This section derives the inequality and identifies the precise \textbf{stability criterion} under which it holds.

\subsubsection{Functional Framework and the Flow Operator}

The Flow Poincaré inequality provides a variational characterization of hypocoercive decay by comparing time-averaged dissipation to time-averaged state norms. We work in the Bochner space $L^2([0,T];\mathcal{H})$ of square-integrable paths $X_t: [0,T] \to \mathcal{H}$, equipped with the normalized inner product:
\begin{align}
    \langle X_t, Y_t \rangle_{T,\mathcal{H}} := \frac{1}{T}\int_{0}^{T} \langle X_t, Y_t \rangle_{\mathcal{H}} dt.
\end{align}
We denote the closed subspace of paths valued in $\mathcal{H}_S = \ker(\mathcal{R})$ by $L^2([0,T];\mathcal{H}_S)$.

For a closed, densely defined operator $A$ on $\mathcal{H}$, we define the associated quadratic form:
\begin{align}
    \mathcal{E}_A(X,Y) := -\langle X, A Y \rangle_{\mathcal{H}}, \quad \mathcal{E}_A(X) := \mathcal{E}_A(X,X),
\end{align}
with time-averaged versions $\mathcal{E}_{T,A}(X_t, Y_t) := \frac{1}{T}\int_{0}^{T} \mathcal{E}_{A}(X_t, Y_t) dt$.

The \textbf{flow operator} $\mathcal{A}$ encodes the coupling between time evolution and the generator $\mathcal{V}$:
\begin{align}
    \mathcal{A} X_t := -\partial_t X_t + \mathcal{V} X_t,
\end{align}
with domain $\mathrm{Dom}(\mathcal{A}) := \{ X_t \in H^{1}([0,T];\mathcal{H}) \mid X_t \in \mathrm{Dom}(\mathcal{V}) \text{ a.e., } \mathcal{V}X_t \in L^{2}([0,T];\mathcal{H}) \}$.

The key technical step is to relate the action of $\mathcal{A}$ to the reference generator $L_O$, while explicitly tracking the error induced by the non-conservative perturbation.

\begin{lemma}[Inner Product Reduction with Perturbative Error]
\label{lem:inner_product_reduction_ae}
    Assume the near-equilibrium lifting structure (Definition \ref{def:ae_lift_structure}) with Approximate Quadratic Form constant $C_{AQF}$. For sufficiently regular paths $X_t, Y_t, Z_t$ valued in $\mathcal{H}_S$:
    \begin{align}
        \langle \mathcal{A} X_t, Z_t + S \mathcal{V} Y_t \rangle_{T,\mathcal{H}} = -\langle \partial_t X_t, Z_t \rangle_{T,\mathcal{H}_S} + \mathcal{E}_{T,-|L_O|}(X_t, Y_t) + R_{T, \eta}(X, Y),
    \end{align}
    where the \textbf{perturbative remainder} satisfies:
    \begin{align}
        |R_{T, \eta}(X, Y)| \le \eta C_{AQF} \frac{1}{T} \int_0^T \|X_t\|_{\mathcal{H}} \|Y_t\|_{\mathcal{H}} dt.
    \end{align}
\end{lemma}
\begin{proof}
The identity follows from expanding $\langle \mathcal{A} X_t, Z_t + S \mathcal{V} Y_t \rangle_{T,\mathcal{H}}$, applying the Orthogonality condition (Assumption \ref{assump:orthogonality}) to eliminate cross-terms, and invoking the Approximate Quadratic Form (Assumption \ref{assump:AQF}):
\begin{equation}
    \langle \mathcal{V} X_t, S \mathcal{V} Y_t \rangle_{\mathcal{H}} = \langle X_t, |L_O| Y_t \rangle_{\mathcal{H}_S} + O(\eta C_{AQF} \|X_t\| \|Y_t\|).
\end{equation}
Time-averaging yields the stated bound on $R_{T,\eta}$. Details are in Appendix \ref{app:inner_product_reduction_ae_proof}.
\end{proof}

\begin{remark}[Role of the Approximate Quadratic Form]
\label{rem:AQF_error_control}
The remainder term $R_{T,\eta}$ quantifies the \textit{deviation from the reversible case}. In equilibrium lifts \cite{Li2025}, the quadratic form identity holds exactly ($R_{T,\eta} \equiv 0$), enabling direct application of the Flow Poincaré framework. Here, the Approximate Quadratic Form (Assumption \ref{assump:AQF}) ensures that this deviation is \textit{explicitly bounded} and \textit{proportional to $\eta$}. This is the mathematical mechanism by which we control the "damage" caused by breaking detailed balance.
\end{remark}

\subsubsection{The Stability Criterion and Main Flow Poincaré Inequality}

Armed with the perturbative error bound from Lemma \ref{lem:inner_product_reduction_ae}, we now derive the Flow Poincaré inequality. The central challenge is to ensure that the error term $R_{T,\eta}$ does not destroy the positivity of the Poincaré constant. This leads to an explicit \textbf{stability criterion} on the perturbation strength $\eta$.

We require auxiliary solutions to a divergence equation involving $|L_O|$. Define the subspace orthogonal to the kernel:
\begin{align}
    L_\perp^2([0,T];\mathcal{H}_S) = \{ X_t \in L^2([0,T];\mathcal{H}_S) \mid X_t \perp \ker(L_O) \text{ for a.e. } t \}.
\end{align}

\begin{theorem}[Abstract Divergence Lemma]
\label{thm:existence_coupled_solution_ae}
    Under Assumption~\ref{assump:regularity_fpi_ae}, for any $T>0$ and $X_t \in L_\perp^2([0,T];\mathcal{H}_S)$, there exists $(Z_t, Y_t)$ with $Z_t, Y_t$ valued in $\mathrm{Dom}(|L_O|)$ solving:
    \begin{align}
        \partial_t Z_t + |L_O| Y_t = X_t,
    \end{align}
    and satisfying energy estimates with constants:
    \begin{align}
        c_1 = \Theta(1), \quad c_2 = \Theta(1), \quad c_3 = \Theta\left(T + \frac{1}{\sqrt{s(L_O)}}\right), \quad c_4 = \Theta\left(1 + \frac{1}{T \sqrt{s(L_O)}}\right).
    \end{align}
\end{theorem}
\begin{proof}
Standard spectral method \cite[Lemma 3.12]{Li2025} using the eigenbasis of the self-adjoint operator $|L_O|$.
\end{proof}
Before stating the regularity assumptions, we establish the coupling strength estimates needed to absorb the perturbative error.

\begin{lemma}[Coupling Strength Estimates]
\label{lem:coupling_strength_ae}
    Under Assumption~\ref{assump:regularity_fpi_ae}, for sufficiently regular paths $X_t \in \mathrm{Dom}(\mathcal{R})$ and $Y_t \in \mathrm{Dom}(L_O)$:
    \begin{align}
        |\langle \mathcal{R} X_t, S \mathcal{V} Y_t \rangle_{\mathcal{H}}| &\le \sqrt{\mathcal{E}_{\mathcal{R}}(X_t) (\mathcal{E}_{-|L_O|}(Y_t) + \eta C_{AQF}\|Y_t\|_{\mathcal{H}}^2)}, \\
        |\langle X_t-P_SX_t, \mathcal{V} Y_t \rangle_{\mathcal{H}}| &\le \sqrt{\mathcal{E}_{\mathcal{R}}(X_t) (\mathcal{E}_{-|L_O|}(Y_t) + \eta C_{AQF}\|Y_t\|_{\mathcal{H}}^2)}, \\
        |\langle \mathcal{V}(X_t-P_SX_t), S\mathcal{V} Y_t \rangle_{\mathcal{H}}| &\le \|X_t-P_SX_t\|_{\mathcal{H}} (K_2 \||L_O|Y_t\|_{\mathcal{H}_S} + K_3 \sqrt{\mathcal{E}_{-|L_O|}(Y_t)}).
    \end{align}
\end{lemma}
\begin{proof}
Apply Cauchy-Schwarz, properties of $S$, the Approximate Quadratic Form, and bounds from Assumption~\ref{assump:regularity_fpi_ae}. See Appendix~\ref{app:coupling_strength_ae_proof}.
\end{proof}

\begin{assumption}[Regularity Conditions for the Flow Poincaré Inequality]
\label{assump:regularity_fpi_ae}
Let $\mathcal{L}=\gamma\mathcal{R}+\mathcal{V}$ satisfy Definition~\ref{def:ae_lift_structure}. We assume:
\begin{enumerate}
    \item \textbf{Path Space Density:} There exists a dense subspace $\mathcal{C}_0 \subset \mathcal{F}^\perp \cap \mathrm{Dom}(\mathcal{L})$ such that trajectories $P_t \mathcal{C}_0$ lie in the domains required by Theorem \ref{thm:existence_coupled_solution_ae} and $\mathcal{A}$.
    \item \textbf{Bounded Perturbation:} For some $K_1 > 0$ and all $X \in \mathrm{Dom}(\mathcal{R}) \cap \mathrm{Dom}(\mathcal{V})$ with $X \in \mathcal{H}_F$:
    \begin{align}
         \| S^{1/2} \mathcal{V} X \|_{\mathcal{H}} \le K_1 \sqrt{\mathcal{E}_{\mathcal{R}}(X)}.
    \end{align}
    \item \textbf{Intertwining Control:} For constants $K_2, K_3 > 0$ and all $Y \in \mathrm{Dom}(L_O)$:
    \begin{align}
        \| (\mathbf{1}-P_S)\mathcal{V}^* S \mathcal{V} Y \|_{\mathcal{H}} \le K_2 \||L_O|Y\|_{\mathcal{H}_S} + K_3 \sqrt{\mathcal{E}_{-|L_O|}(Y)}.
    \end{align}
\end{enumerate}
\end{assumption}

We now state the main theorem, which establishes the Flow Poincaré inequality and derives the explicit stability criterion.

\begin{theorem}[Flow Poincaré Inequality and Stability Criterion]
\label{thm:flow_poincare_ae}
    Let $C_{AQF}$ be the Approximate Quadratic Form constant (Assumption \ref{assump:AQF}). Define the \textbf{structural correction constant}:
    \begin{align}
        C_{corr} := C_{AQF} s(L_O)^{-1} c_1(T).
    \end{align}
    
    \textbf{Stability Criterion:} Assume the non-conservative perturbation satisfies:
    \begin{align}
        \eta < C_{corr}^{-1} = \frac{s(L_O)}{C_{AQF} c_1(T)}. \label{eq:stability_criterion}
    \end{align}
    
    Then, under Assumption \ref{assump:regularity_fpi_ae}, for any $T>0$ and $X_0 \in \mathcal{F}^\perp \cap \mathrm{Dom}(\mathcal{L})$, the trajectory $X_t = P_t X_0$ satisfies:
    \begin{align}
        \alpha_T(\eta) \|X_t\|_{T,\mathcal{H}}^2 \le \mathcal{E}_{T,\mathcal{R}}(X_t),
    \end{align}
    where the \textbf{$\eta$-dependent Poincaré constant} is:
    \begin{align}
        \alpha_T(\eta) = \left[ \frac{(\gamma \tilde{A}_1(T,\eta) + \tilde{A}_2(T,\eta))^2}{(1 - \eta C_{corr})^2} + \frac{1}{\lambda_R} \right]^{-1} > 0,
    \end{align}
    with $\eta$-corrected coefficients:
    \begin{align}
        \tilde{A}_1(T,\eta) &= c_3(T) + \sqrt{\eta C_{AQF}} s(L_O)^{-1} c_1(T), \\
        \tilde{A}_2(T,\eta) &= K_1 c_2(T) + \lambda_R^{-1/2}(\|S\|^{1/2} c_4(T) + K_2 c_1(T) + K_3 c_2(T)) + \sqrt{\eta C_{AQF}} K_1 s(L_O)^{-1/2} c_2(T).
    \end{align}
\end{theorem}
\begin{proof}
    The proof adapts the variational argument from \cite{Li2025}. It involves decomposing the trajectory norm, using the solution $(Z_t, Y_t)$ to the abstract divergence equation to handle the projected part $P_SX_t$, and applying the explicit coupling estimates (Lemma \ref{lem:coupling_strength_ae}) derived from the near-equilibrium lifting structure. The $\eta$-dependent error terms from the Approximate Quadratic Form condition are rigorously tracked and absorbed into the coefficients $\tilde{A}_i(T,\eta)$ and $C_{corr}$, resulting in a strict inequality without constant error tails. A full, self-contained proof is provided in Appendix \ref{app:flow_poincare_ae_proof}.
\end{proof}

\begin{remark}[Interpretation: The Stability Criterion]
The condition $\eta < C_{corr}^{-1}$ is the \textbf{stability criterion} for the Flow Poincaré inequality. Its significance:
\begin{itemize}
    \item \textbf{Denominator Singularity:} As $\eta \to C_{corr}^{-1}$ from below, the denominator $(1 - \eta C_{corr})^2 \to 0^+$, causing $\alpha_T(\eta) \to 0^+$. This indicates \emph{loss of coercivity}: the Poincaré constant degrades to the point where dissipation no longer controls the state norm effectively.
    
    \item \textbf{Breakdown of Stability:} When $\eta \ge C_{corr}^{-1}$, the inequality fails because the perturbative error $R_{T,\eta}$ (bounded by the AQF) grows large enough to \emph{overwhelm} the dissipative control from $\mathcal{R}$. The lifting structure breaks down.
    
    \item \textbf{Structural Origin:} Since $C_{corr} = C_{AQF} s(L_O)^{-1} c_1(T)$, the threshold is determined by:
    \begin{itemize}
        \item $C_{AQF}$: How much the non-conservative perturbation deviates from reversibility (quantified by the AQF).
        \item $s(L_O)$: The spectral gap of the reference slow generator. \emph{Larger gaps $\Rightarrow$ more tolerance for perturbation}.
        \item $c_1(T)$: A path-integral constant scaling the cumulative effect of the slow dynamics over time $T$.
    \end{itemize}
    
    \item \textbf{Consistency with Upper Bound:} Combining the AQF scaling $C_{AQF} = \Theta(s(L_O)^{-1})$ (Proposition \ref{prop:quad_form_condition_ae_approx}) and $c_1(T) = \Theta(1)$ (Theorem \ref{thm:existence_coupled_solution_ae}), we obtain $C_{corr} = \Theta(s(L_O)^{-2})$, yielding the stability criterion $\eta \ll s(L_O)^2$. This coincides \emph{exactly} with the perturbative regime identified in the spectral continuity theorem (Theorem \ref{thm:upper_bound_ae_v2}).
\end{itemize}
\end{remark}

\subsubsection{Lower Bound Estimation from Flow Poincaré Inequality}

The Flow Poincaré inequality (Theorem~\ref{thm:flow_poincare_ae}) establishes coercivity of dissipation over time-averaged norms, with a Poincaré constant $\alpha_T(\eta)$ that \emph{remains strictly positive} for $\eta < C_{corr}^{-1}$. This coercivity translates directly to exponential convergence via a Grönwall-type argument. The key question is: \textbf{How does the convergence rate depend on $\eta$, and how does it scale with system parameters in the stable regime?}

\begin{theorem}[$T$-Average Convergence Lower Bound]
\label{thm:T_avg_bound}
    Let the generator $L=\gamma\mathcal{R}+\mathcal{V}$ satisfy the assumptions of Theorem \ref{thm:flow_poincare_ae}. Assume $\eta < C_{corr}^{-1}$ to ensure $\alpha_T(\eta) > 0$. For any observation period $T > 0$, provided the effective rate $\nu_{\eff}$ defined below is positive, any initial state $X_0 \in \mathcal{F}^\perp \cap \mathrm{Dom}(L)$ exhibits time-averaged strict exponential decay bounded by:
    \begin{align}
        \frac{1}{T}\int_t^{t+T}\|P_s X_0\|_{\mathcal{H}}^{2}ds \le e^{- 2\nu_{\eff} t}\|X_{0}\|_{\mathcal{H}}^{2},
    \end{align}
    where the decay rate parameter $\nu_{\eff}$ depends on $\gamma, \eta, T$ as
    \begin{align}
        \nu_{\eff} = \nu_{\eff}(\gamma, \eta, T) := \gamma \alpha_T(\eta) - \eta \|L_{\pert}\|.
    \end{align}
\end{theorem}
\begin{proof}
    Define the time-averaged energy functional $E_T(t) = \frac{1}{T}\int_t^{t+T}\|P_s X_0\|_{\mathcal{H}}^2 ds$. Differentiate w.r.t. $t$, apply the Flow Poincaré inequality (Theorem \ref{thm:flow_poincare_ae}) to bound dissipation, and control the perturbation term using $\|L_{\pert}\|$. This yields $\frac{d}{dt}E_T(t) \le -2\nu_{\eff} E_T(t)$. Solve via Grönwall's lemma. Full derivation in Appendix \ref{app:T_avg_bound}.
\end{proof}

\begin{remark}[Interpretation: Stability Under Perturbation]
The rate $\nu_{\eff}(\gamma, \eta, T) = \gamma \alpha_T(\eta) - \eta \|L_{\pert}\|$ reveals two competing effects:
\begin{itemize}
    \item \textbf{Enhancement via Fast Dissipation:} The term $\gamma \alpha_T(\eta)$ shows that scaling $\gamma$ accelerates convergence (as in equilibrium lifting).
    \item \textbf{Degradation via Non-Conservative Perturbation:} The term $-\eta \|L_{\pert}\|$ represents the rate penalty from breaking detailed balance.
    \item \textbf{Stability of Optimal Scaling:} As long as $\eta < C_{corr}^{-1}$, the Poincaré constant $\alpha_T(\eta)$ remains $\Theta(1)$, so the optimal rate $\nu_{\eff} = \Theta(\gamma)$ scales correctly despite the perturbation. The degradation is \emph{controlled} and does not destroy the $\gamma$-scaling.
\end{itemize}
\end{remark}

The time-averaged decay described in Theorem \ref{thm:T_avg_bound} yields an explicit lower bound on the convergence rate. This result can be refined to provide a pointwise exponential decay bound for the trajectory $\|X_t\|_{\mathcal{H}}$, along with an explicit expression for the near-optimal fast dissipation scale $\gamma$ that maximizes this lower bound for a fixed averaging time $T$.

\begin{corollary}[Near-optimal Selection of $\gamma$]
\label{cor:near_optimal_gamma}
    Assume $\eta < C_{corr}^{-1}$ to ensure $\alpha_T(\eta) > 0$. Under the assumptions of Theorem \ref{thm:T_avg_bound}, for any $T>0$ and initial state $X_0 \in \mathcal{F}^\perp \cap \mathrm{Dom}(L)$, the trajectory $X_t = P_t X_0$ satisfies the pointwise decay bound:
    \begin{align}
        \|X_{t}\|_{\mathcal{H}} \le C_T e^{-\nu_{\eff} t} \|X_0\|_{\mathcal{H}},
    \end{align}
    where the decay rate is $\nu_{\eff} = \gamma \alpha_T(\eta) - \eta \|L_{\pert}\|$ and the prefactor is $C_T=e^{\nu_{\eff} T}$.

    Furthermore, for a fixed $T$, the lower bound on the decay rate $\nu_{\eff}$ is maximized by choosing $\gamma$ near-optimally as
    \begin{align}
        \gamma_{opt}(T) = \frac{1}{\tilde{A}_1(T,\eta)}\sqrt{\tilde{A}_2(T,\eta)^2 + \frac{(1 - \eta C_{corr})^2}{2\lambda_R}},
    \end{align}
    where $\tilde{A}_1, \tilde{A}_2, C_{corr}$ are the $\eta$-dependent coefficients defined in Theorem \ref{thm:flow_poincare_ae}. This choice yields the corresponding maximal rate lower bound $\nu_{opt}(T)$ explicitly derived in Appendix \ref{app:near_optimal_gamma}.
\end{corollary}
\begin{proof}
    The hypocoercive-type estimate follows from the time-averaged bound (Theorem \ref{thm:T_avg_bound}) using semigroup contractivity. The explicit expressions for $\gamma_{opt}(T)$ and $\nu_{opt}(T)$ are obtained by maximizing the rate expression $\nu_{\eff}$ with respect to $\gamma$. The detailed derivation is provided in Appendix \ref{app:near_optimal_gamma}.
\end{proof}

This corollary provides an explicit lower bound $\nu_{opt}(T)$ on the exponential convergence rate of the full generator $L$ for a given $T$. It demonstrates how the interplay between the fast dissipation scale $\gamma$, the properties of the effective dynamics $L_O$, and the non-conservative perturbation $\eta L_{\pert}$ determines the convergence speed. 

\section{Asymptotic Analysis and Physical Regimes}
\label{sec:asymptotic_analysis}

The bounds established in Section~\ref{sec:perturbative_stability} are valid for any $\eta < C_{corr}^{-1}$, ensuring structural stability. However, to achieve the characteristic \textbf{quadratic acceleration} $\nu(\mathcal{L}) = \Theta(\sqrt{s(L_O)})$—the hallmark of optimal lifting—we must understand the \emph{asymptotic regime} where perturbative corrections become negligible. This section formalizes the \textbf{scaling limit} $\eta = o(s(L_O)^2)$, proves that under this condition the near-equilibrium lifting achieves the same optimal scaling as reversible lifts, and provides physical interpretation of this regime.

\subsection{The Scaling Limit: Formalizing Near-Equilibrium}

The upper bound (Theorem~\ref{thm:upper_bound_ae_v2}) and stability criterion (Theorem~\ref{thm:flow_poincare_ae}) both involve the threshold $\eta \sim s(L_O)^2$. To achieve asymptotic optimality, we require a stronger condition.

\begin{definition}[Near-Equilibrium Scaling Limit]
\label{def:near_eq_limit}
We say a family of generators $\{\mathcal{L}_s\}_{s \to 0^+}$ indexed by the spectral gap $s = s(L_O)$ satisfies the \textbf{near-equilibrium scaling limit} if the non-conservative perturbation strength satisfies:
\begin{equation}
    \eta = o(s^2) \quad \text{as } s \to 0. \label{eq:near_eq_scaling}
\end{equation}
Equivalently, for any $\delta > 0$, there exists $s_0 > 0$ such that $\eta < \delta s^2$ for all $s < s_0$.
\end{definition}

This condition is \emph{strictly stronger} than the stability threshold $\eta = O(s^2)$ required for positive Poincaré constant. Its role is to ensure that perturbative error terms vanish \emph{faster} than the lifting gain.

\begin{theorem}[Vanishing Perturbative Corrections in the Scaling Limit]
\label{thm:scaling_limit_analysis}
Assume the near-equilibrium scaling limit $\eta = o(s^2)$. Let $C_{AQF} = \Theta(s^{-1})$ (as in Proposition~\ref{prop:quad_form_condition_ae_approx}). Then:
\begin{enumerate}
    \item \textbf{Stability Correction Vanishes:} The structural correction in the Flow Poincaré inequality satisfies:
    \begin{equation}
        \eta C_{corr} = \eta \cdot C_{AQF} s^{-1} c_1(T) = o(1) \quad \text{as } s \to 0.
    \end{equation}
    Consequently, the denominator $(1 - \eta C_{corr})^2 \to 1$ and the Poincaré constant $\alpha_T(\eta) \to \alpha_T(0)$ (the reversible limit).
    
    \item \textbf{Perturbative Decay Penalty Vanishes:} In the effective rate $\nu_{\eff} = \gamma \alpha_T(\eta) - \eta \|L_{\pert}\|$, the perturbation term satisfies:
    \begin{equation}
        \frac{\eta \|L_{\pert}\|}{\gamma \alpha_T(\eta)} = o(1) \quad \text{for } \gamma = \Theta(\sqrt{s}).
    \end{equation}
    Thus the non-conservative correction is subdominant to the lifting gain.
    
    \item \textbf{Preservation of Ballistic Scaling:} The optimal convergence rate satisfies:
    \begin{equation}
        \nu_{opt} = \Theta(\sqrt{s}) + o(\sqrt{s}) = \Theta(\sqrt{s}),
    \end{equation}
    achieving the same asymptotic scaling as reversible optimal lifts.
\end{enumerate}
\end{theorem}

\begin{proof}
(1) From $C_{AQF} = \Theta(s^{-1})$, $c_1(T) = \Theta(1)$ (Theorem~\ref{thm:existence_coupled_solution_ae}), and $\eta = o(s^2)$:
\begin{equation}
    \eta C_{corr} = \eta \cdot \Theta(s^{-1}) \cdot s^{-1} \cdot \Theta(1) = \Theta(\eta s^{-2}) = o(1).
\end{equation}

(2) For $\gamma = \Theta(\sqrt{s})$ and $\alpha_T(\eta) = \Theta(1)$ (bounded away from zero for $\eta < C_{corr}^{-1}$):
\begin{equation}
    \gamma \alpha_T(\eta) = \Theta(\sqrt{s}), \quad \eta \|L_{\pert}\| = O(\eta) = o(s^2).
\end{equation}
Since $s \to 0$, we have $\sqrt{s} \gg s^2$ for small $s$, thus $\eta \|L_{\pert}\| / (\gamma \alpha_T(\eta)) = o(1)$.

(3) Combine (1) and (2): the rate $\nu_{opt} = \gamma_{opt} \alpha_T(\eta) - \eta \|L_{\pert}\| + o(\sqrt{s})$ with $\gamma_{opt} = \Theta(\sqrt{s})$ yields $\nu_{opt} = \Theta(\sqrt{s})(1 + o(1)) = \Theta(\sqrt{s})$.
\end{proof}

\subsection{Physical Interpretation: Why $\eta \ll s^2$?}

The scaling limit $\eta = o(s^2)$ has a clear physical meaning rooted in the \textbf{timescale hierarchy} of the system.

\subsubsection{Three Intrinsic Timescales}

A near-equilibrium lifted system exhibits three natural timescales:

\begin{enumerate}
    \item \textbf{Fast Relaxation Time:} $\tau_{\text{fast}} = \lambda_R^{-1}$
    
    The time for the fast mode $\mathcal{R}$ to equilibrate. By assumption (timescale separation), $\lambda_R \gg s(L_O)$, so $\tau_{\text{fast}} \ll \tau_{\text{slow}}$.
    
    \item \textbf{Slow Relaxation Time (Unlifted):} $\tau_{\text{slow}} = s(L_O)^{-1}$
    
    The intrinsic convergence time of the effective slow generator $L_O$. This sets the baseline (diffusive) timescale.
    
    \item \textbf{Perturbation Timescale:} $\tau_{\text{pert}} = \eta^{-1}$
    
    The timescale over which the non-conservative perturbation $\eta \mathcal{L}_{\pert}$ accumulates significant deviation from reversibility.
\end{enumerate}

\subsubsection{The Dominance Condition}

The condition $\eta \ll s^2$ can be rewritten as:
\begin{equation}
    \frac{1}{\tau_{\text{pert}}} \ll \frac{1}{\tau_{\text{slow}}^2} \quad \Longleftrightarrow \quad \tau_{\text{pert}} \gg \tau_{\text{slow}}^2.
\end{equation}

\textbf{Physical Meaning:} The non-equilibrium drive acts on a timescale \emph{much longer} than the square of the slow relaxation time. Equivalently, within the slow timescale $\tau_{\text{slow}}$, the system experiences only a \emph{fractional perturbation} of order $\eta \tau_{\text{slow}} = o(s)$, which is negligible compared to the intrinsic slow dynamics.

\subsubsection{Regime Classification}

\begin{itemize}
    \item \textbf{Near-Equilibrium Regime:} $\eta \ll s^2$
    
    The non-conservative perturbation is subdominant. The system behaves \emph{almost reversibly} on the slow timescale. Optimal lifting achieves ballistic acceleration $\Theta(\sqrt{s})$.
    
    \item \textbf{Marginal Stability Regime:} $\eta = \Theta(s^2)$
    
    Perturbative corrections are comparable to the lifting gain. The spectral gap remains positive ($\eta < C_{corr}^{-1} = \Theta(s^2)$), but acceleration is degraded by $O(1)$ factors. The system is \emph{stable but not optimal}.
    
    \item \textbf{Breakdown Regime:} $\eta \geq \Theta(s^2)$ (specifically $\eta \geq C_{corr}^{-1}$)
    
    The Flow Poincaré constant degrades to zero. Dissipation no longer controls the norm effectively. The lifting structure \emph{fails}.
\end{itemize}

\subsection{Optimal Acceleration in the Near-Equilibrium Regime}

With the scaling limit formalized, we can now state the main asymptotic result rigorously.

\begin{corollary}[Optimal Rate Scaling and Quadratic Speedup]
\label{cor:optimal_time_prefactor_ae}
    Let $s \coloneqq s(L_O)$ denote the singular value gap of the target effective generator. Under the assumptions of Theorem \ref{thm:flow_poincare_ae}, suppose the structural constants satisfy the scaling conditions $K_1, K_2, \lambda_R, \|S\| = \Theta(1)$ and $K_3 = \Theta(\sqrt{s})$.
    Assume further that the Approximate Quadratic Form constant scales as $C_{AQF} = \Theta(s^{-1})$ and that the non-conservative perturbation strength $\eta$ satisfies the \textbf{near-equilibrium scaling limit} (Definition~\ref{def:near_eq_limit}):
    \begin{equation}
        \eta = o(s^2) \quad \text{as } s \to 0.
    \end{equation}
    Then, by choosing the observation time optimally as $T_{opt} = \Theta(s^{-1/2})$, the maximal convergence rate lower bound exhibits \textbf{quadratic scaling}:
    \begin{align}
        \nu_{opt}(T_{opt}) = \Theta(\sqrt{s}).
    \end{align}
    Moreover, the prefactor $C_{T_{opt}} = e^{\nu_{opt}(T_{opt}) T_{opt}}$ associated with the pointwise decay bound remains asymptotically bounded, i.e., $C_{T_{opt}} = \Theta(1)$.
\end{corollary}

\begin{proof}
    The proof relies on the asymptotic analysis of the rate expression $\nu_{opt}(T)$ derived in Corollary \ref{cor:near_optimal_gamma}. Substituting the scalings of the energy estimate constants $c_i(T)$ with $T = \Theta(s^{-1/2})$ and $K_3 = \Theta(\sqrt{s})$, the dominant behavior of the lifting term is identified as $\Theta(\sqrt{s})$. Crucially, Theorem~\ref{thm:scaling_limit_analysis} shows that the condition $\eta = o(s^2)$ ensures: (i) the spectral stability correction term $\eta C_{corr} = o(1)$ vanishes asymptotically, preserving the strict positivity of the Flow Poincaré constant, and (ii) the perturbative decay rate reduction $-\eta \|L_{\pert}\|$ remains negligible compared to the lifting gain. Finally, the exponent $\nu_{opt}(T_{opt}) T_{opt}$ is shown to be $\Theta(1)$, ensuring a bounded prefactor. The detailed rigorous calculation is provided in Appendix \ref{app:optimal_time_prefactor_ae_strict}.
\end{proof}

\subsection{Main Physical Conclusion}

\begin{mdframed}[linewidth=1.5pt, linecolor=black, backgroundcolor=gray!5]
\textbf{Conclusion: Structural Stability of Optimal Acceleration}

\vspace{0.5em}

The quadratic acceleration $\nu(\mathcal{L}) = \Theta(\sqrt{s(L_O)})$ characteristic of optimal equilibrium lifting is \textbf{structurally stable} under non-equilibrium perturbations in the near-equilibrium regime $\eta = o(s(L_O)^2)$.

\vspace{0.5em}

Specifically:
\begin{itemize}
    \item The \textbf{ballistic scaling} $\Theta(\sqrt{s})$ persists despite breaking detailed balance.
    \item The \textbf{optimal dissipation scale} $\gamma_{opt} = \Theta(\sqrt{s})$ is preserved.
    \item The \textbf{mechanism}: The Approximate Quadratic Form controls perturbative deviations, ensuring error terms $R_{T,\eta} = O(\eta)$ remain subdominant to the $\Theta(\sqrt{s})$ lifting gain when $\eta \ll s^2$.
\end{itemize}

\vspace{0.5em}

This establishes that the non-normality of NESS generators does \emph{not} obstruct optimal acceleration, provided the non-equilibrium drive remains weaker than the intrinsic slow dynamics on the appropriate timescale.
\end{mdframed}

\vspace{1em}

\begin{remark}[Robustness vs. Optimality]
We distinguish two notions of stability:

\begin{enumerate}
    \item \textbf{Stability (Robustness):} The spectral gap remains positive and acceleration persists for $\eta < C_{corr}^{-1} = \Theta(s^2)$.
    
    This is the \emph{structural stability} regime proven in Section~\ref{sec:perturbative_stability}. The lifting works, but may experience $O(1)$ degradation.
    
    \item \textbf{Asymptotic Optimality:} The rate achieves the pure scaling $\Theta(\sqrt{s})$ without degradation for $\eta = o(s^2)$.
    
    This is the \emph{near-equilibrium regime} analyzed in this section. All perturbative corrections vanish asymptotically.
\end{enumerate}

The broader regime $\eta = O(s^2)$ (but $\eta < C_{corr}^{-1}$) represents a \textbf{stable-but-suboptimal} intermediate zone where the mechanism is robust but asymptotic purity is lost.
\end{remark}

\begin{remark}[Physical Consistency of Scaling Assumptions]
The structural scaling assumptions employed in Corollary~\ref{cor:optimal_time_prefactor_ae} are not arbitrary:

\begin{itemize}
    \item $\|S\| = \lambda_R^{-1} = \Theta(1)$: Follows from timescale separation $\lambda_R \gg s(L_O)$.
    \item $C_{AQF} = \Theta(s^{-1})$: Spectral geometry of $\sqrt{L_O}$ near zero (Proposition~\ref{prop:quad_form_condition_ae_approx}).
    \item $K_3 = \Theta(\sqrt{s})$: Intertwining scaling matching the optimal coupling identified for equilibrium lifts in \cite[Eq.~(2.42)]{Li2025}.
\end{itemize}

These scalings are geometrically natural and coincide with the known optimal structure for reversible systems, suggesting that the near-equilibrium lifting framework successfully generalizes equilibrium lifting to NESS.
\end{remark}

\section{Numerical Verification of Robustness}
\label{sec:examples}

Having established the theoretical framework for near-equilibrium lifting and the asymptotic scaling regime $\eta = o(s^2)$, we now provide \textbf{numerical validation of the robustness} of optimal acceleration under non-equilibrium perturbations. We apply the framework to two canonical systems: classical nonequilibrium Langevin dynamics \cite{Monmarche2022,Iacobucci2017} and a boundary-driven open quantum spin chain \cite{Popkov2020,Landi2022}. 

For both examples, we rigorously verify that the generators satisfy the key structural conditions (Orthogonality and Approximate Quadratic Form). The numerical experiments are designed to test three central predictions:
\begin{enumerate}
    \item \textbf{Structural Stability}: Optimal acceleration persists despite breaking detailed balance (finite $\eta > 0$).
    \item \textbf{Optimal Dissipation Scale}: There exists a finite $\gamma_{opt}$ maximizing the convergence rate.
    \item \textbf{Near-Ballistic Scaling}: The rate exhibits $\nu_{opt} \approx \sqrt{s(L_O)}$ scaling, with deviations attributable to finite-$\eta$ corrections predicted by perturbation theory.
\end{enumerate}

\subsection{Example: Classical Nonequilibrium Langevin Dynamics}
\label{sec:example_langevin}

We now illustrate the near-equilibrium lifting structure with a canonical example from classical statistical mechanics: the nonequilibrium Langevin dynamics in the high-friction limit \cite{Monmarche2022,Iacobucci2017}. This system clearly exhibits the timescale separation required for adiabatic elimination, where the momentum variable equilibrates much faster than the position variable.

\subsubsection{Dynamics and Generators}

Consider a particle with position $q \in \mathbb{R}^d$ and momentum $p \in \mathbb{R}^d$ evolving according to the underdamped Langevin equations:
\begin{align}
    dq_t &= p_t dt, \\
    dp_t &= (-\nabla_q U(q_t) + \eta F(q_t)) dt - \gamma p_t dt + \sqrt{\frac{2\gamma}{\beta}} dW_t,
\end{align}
where $U(q)$ is a potential, $\eta F(q)$ is a non-conservative force ($0 \le \eta \ll \min(1,\gamma)$), where $\gamma $ is the friction coefficient, $\beta = 1/T$ is the inverse temperature, and $W_t$ is a standard $d$-dimensional Wiener process. The generator $L$ for this process, acting on suitable test functions $f(q, p)$, is the adjoint of the associated Fokker-Planck operator:
\begin{align}
    L f(q, p) = p \cdot \nabla_q f + (-\nabla_q U(q) + \eta F(q) - \gamma p) \cdot \nabla_p f + \frac{\gamma}{\beta} \Delta_p f. \label{eq:langevin_L_full_example}
\end{align}
In the high-friction limit $\gamma \to \infty$, after appropriate time rescaling, the dynamics are expected to converge to the overdamped Langevin equation $dQ_t = (-\nabla_q U(Q_t) + \eta F(Q_t)) dt + \sqrt{2/\beta} dW'_t$ \cite{Monmarche2022}. The generator for this limiting process acts on functions $f(q)$ as:
\begin{align}
    L_O f(q) = (-\nabla_q U(q) + \eta F(q)) \cdot \nabla_q f + \frac{1}{\beta} \Delta_q f. \label{eq:langevin_LO_target_example}
\end{align}
This $L_O$ represents the target effective generator whose structure we aim to recover via adiabatic elimination.

\subsubsection{Hilbert Space Structure and Decomposition}

The natural Hilbert space for the full dynamics is $\mathcal{H} = L^2(\mathbb{R}^{2d}, d\widehat{\mu})$, where $\widehat{\mu}$ is the invariant measure of the underdamped process \eqref{eq:langevin_L_full_example}. In the limit $\gamma \to \infty$, this measure is known to factorize as $d\widehat{\mu}(q, p) \to d\mu(q) g_\beta(p) dp$ in Wasserstein distance \cite[Theorem 1.1]{Monmarche2022}, where $\mu$ is the invariant measure for the overdamped dynamics ($L_O^* \mu = 0$) and $g_\beta(p) = (\beta/2\pi)^{d/2} e^{-\beta|p|^2/2}$ is the Maxwell-Boltzmann distribution. We equip $\mathcal{H}$ with the inner product $\langle f, g \rangle_{\mathcal{H}} = \int f^*(q,p) g(q,p) d\widehat{\mu}(q,p)$.

We decompose the full generator $L$ according to the AE structure $L = \gamma\mathcal{R} + \mathcal{V}$:
\begin{align}
    \gamma\mathcal{R} f &= -\gamma p \cdot \nabla_p f + \frac{\gamma}{\beta} \Delta_p f, \\
    \mathcal{V} f &= p \cdot \nabla_q f + (-\nabla_q U(q) + \eta F(q)) \cdot \nabla_p f.
\end{align}
The operator $\mathcal{R} = -p \cdot \nabla_p + \frac{1}{\beta} \Delta_p$ generates the Ornstein-Uhlenbeck process for the momentum $p$, which thermalizes to the Maxwell-Boltzmann distribution $g_\beta(p)$. Its kernel consists precisely of functions independent of $p$. Therefore, the slow subspace is $\mathcal{H}_S = \ker(\mathcal{R}) = \{ f \in \mathcal{H} \mid f(q, p) = f(q) \text{ a.e.} \}$, which corresponds to $L^2(\mathbb{R}^d, d\mu)$ in the large $\gamma$ limit. The orthogonal projection $P_S: \mathcal{H} \to \mathcal{H}_S$ in this limit is given by averaging over the momentum with the Maxwell-Boltzmann weight: $P_S[f](q) = \int f(q, p) g_\beta(p) dp$. It is known that $\mathcal{R}$ is self-adjoint and negative semi-definite with respect to the inner product weighted by $g_\beta(p)$ (and consequently for $d\widehat{\mu}$ in the large $\gamma$ limit).

\subsubsection{Verification of Near-Equilibrium Lifting Conditions}

We now explicitly verify the key structural conditions required by the AE framework, namely the Orthogonality condition ($P_S\mathcal{V}P_S=0$) and the second-order formula for $L_O$.

\begin{proposition}[Conditions Verification for Langevin]
The Langevin generator $L=\gamma\mathcal{R}+\mathcal{V}$ satisfies:
\begin{enumerate}[(i)]
    \item $P_S\mathcal{V}P_S = 0$.
    \item $L_O = -P_S\mathcal{V}\mathcal{R}^+\mathcal{V}P_S$, where $L_O$ is the overdamped generator \eqref{eq:langevin_LO_target_example}.
\end{enumerate}
\end{proposition}
\begin{proof}
(i) \textit{Verification of Orthogonality:} Let $f \in \mathcal{H}$ be an arbitrary function. Its projection onto the slow subspace is $f_q(q) = P_S[f](q)$, which is independent of $p$. We apply the coupling operator $\mathcal{V}$ to $f_q$:
\begin{align}
    \mathcal{V} f_q(q) = p \cdot \nabla_q f_q(q) + (-\nabla_q U(q) + \eta F(q)) \cdot \underbrace{\nabla_p f_q(q)}_{=0} = p \cdot \nabla_q f_q(q).
\end{align}
Now, we project this result back onto the slow subspace using $P_S$:
\begin{align}
    P_S[\mathcal{V} (P_S f)](q) &= \int_{\mathbb{R}^d} (p \cdot \nabla_q f_q(q)) g_\beta(p) dp \\
    &= \left( \int_{\mathbb{R}^d} p g_\beta(p) dp \right) \cdot \nabla_q f_q(q).
\end{align}
The integral $\int p g_\beta(p) dp$ represents the mean momentum under the Maxwell-Boltzmann distribution, which is zero as $g_\beta(p)$ is centered. Consequently, $P_S[\mathcal{V} (P_S f)](q) = 0$ for all $f$, demonstrating that $P_S\mathcal{V}P_S = 0$.

(ii) \textit{Verification of Second-Order Formula:} We compute the expression $-P_S\mathcal{V}\mathcal{R}^+\mathcal{V}P_S$ acting on an arbitrary function $f_q(q) \in \mathcal{H}_S$.
First, as shown above, $\mathcal{V}P_S[f](q, p) = \mathcal{V} f_q(q) = p \cdot \nabla_q f_q(q)$. Let $h(q, p) = p \cdot \nabla_q f_q(q)$.
Next, we determine the action of the pseudo-inverse $\mathcal{R}^+$. We need to find $\psi = \mathcal{R}^+ h$, which is the unique solution in $\mathcal{H}_F$ (i.e., $P_S[\psi]=0$) to the equation $\mathcal{R}\psi = h$:
\begin{align}
    -p \cdot \nabla_p \psi + \frac{1}{\beta}\Delta_p \psi = p \cdot \nabla_q f_q(q).
\end{align}
By inspection (or using standard results for the Ornstein-Uhlenbeck generator), the solution with mean zero is $\psi(q, p) = -p \cdot \nabla_q f_q(q)$. Indeed, $\nabla_p \psi = -\nabla_q f_q(q)$ and $\Delta_p \psi = 0$, so $\mathcal{R}\psi = -p \cdot (-\nabla_q f_q(q)) + 0 = p \cdot \nabla_q f_q(q) = h$.
Thus, $\mathcal{R}^+ (p \cdot \nabla_q f_q(q)) = -p \cdot \nabla_q f_q(q)$.
Now, we apply $\mathcal{V}$ to $\psi$:
\begin{align}
    \mathcal{V} \psi &= p \cdot \nabla_q \psi + (-\nabla_q U + \eta F) \cdot \nabla_p \psi \\
    &= p \cdot \nabla_q (-p \cdot \nabla_q f_q(q)) + (-\nabla_q U + \eta F) \cdot \nabla_p (-p \cdot \nabla_q f_q(q)) \\
    &= -(p \otimes p) : \nabla_q^2 f_q(q) + (-\nabla_q U + \eta F) \cdot (-\nabla_q f_q(q)).
\end{align}
Finally, we apply the projection $P_S$ (momentum averaging with $g_\beta(p)$) and multiply by $-1$:
\begin{align}
    -P_S [ \mathcal{V} \mathcal{R}^+ \mathcal{V} P_S f ] &= -\int \left[ -(p \otimes p) : \nabla_q^2 f_q(q) + (\nabla_q U - \eta F) \cdot \nabla_q f_q(q) \right] g_\beta(p) dp \\
    &= \left(\int (p \otimes p) g_\beta(p) dp \right) : \nabla_q^2 f_q(q) - \left(\int g_\beta(p) dp \right) (\nabla_q U - \eta F) \cdot \nabla_q f_q(q).
\end{align}
Using the standard Gaussian integral results $\int g_\beta(p) dp = 1$ and $\int (p \otimes p) g_\beta(p) dp = \frac{1}{\beta} \mathbf{1}$ (where $\mathbf{1}$ is the identity tensor), we obtain:
\begin{align}
    -P_S [ \mathcal{V} \mathcal{R}^+ \mathcal{V} P_S f ] &= \frac{1}{\beta} \mathbf{1} : \nabla_q^2 f_q(q) - (\nabla_q U - \eta F) \cdot \nabla_q f_q(q) \\
    &= \frac{1}{\beta} \Delta_q f_q(q) + (-\nabla_q U + \eta F) \cdot \nabla_q f_q(q).
\end{align}
This expression precisely matches the generator $L_O f_q(q)$ of the overdamped Langevin dynamics given in \eqref{eq:langevin_LO_target_example}.

Thus, we have rigorously verified both $P_S\mathcal{V}P_S=0$ and $L_O = -P_S\mathcal{V}\mathcal{R}^+\mathcal{V}P_S$. This confirms that the classical Langevin system provides a concrete realization of the abstract near-equilibrium lifting structure.
\end{proof}


\subsubsection{Numerical Verification}
\label{sec:numerical_langevin}

To provide a concrete validation of our theoretical framework, particularly the prediction of an optimal convergence rate and the $\nu = \Theta(\sqrt{s(L_O)})$ scaling (Corollary \ref{cor:optimal_time_prefactor_ae}), we perform a numerical experiment on the non-reversible Langevin dynamics. This experiment is designed to test two central, non-trivial predictions of the lifting structure: (i) the existence of an optimal, finite dissipation scale $\gamma$ that maximizes the convergence rate $\nu(L)$, and (ii) a quantitative, super-linear scaling relationship between this accelerated rate and the intrinsic rate $s(L_O)$ of the target effective system.

\paragraph{Numerical Estimate of Convergence Rate.}
A critical aspect of this numerical validation is the accurate measurement of the exponential convergence rate $\nu(L)$. For a non-reversible NESS, the generator $L$ is non-self-adjoint. Consequently, its spectrum is complex and the autocorrelation function (ACF) of an observable $f$ is not a simple sum of positive decaying exponentials. Instead, it exhibits oscillatory decay.

The dynamics of any observable $f$ (projected onto the subspace orthogonal to the steady state) can be decomposed in the eigenbasis of the generator $L$ with invariant measure $\pi$. For $t$ large enough, the dynamics are dominated by the slowest-decaying modes (i.e., the eigenvalues $\lambda_j$ with the largest real part, $\text{Re}(\lambda_j) = -\nu$). For a non-reversible system, these eigenvalues may be complex, $\lambda_{slow} = -\nu \pm i\omega$. The ACF's asymptotic behavior is thus: 
$$
\rho_f(t) = \frac{\langle f, e^{tL} f \rangle_\pi}{\|f\|_\pi^2} \approx A e^{-\nu t} \cos(\omega t + \phi), \quad t \to \infty
$$
While $\rho_f(t)$ itself oscillates, the asymptotic convergence rate $\nu$ governs the exponential decay of its envelope. We can therefore extract $\nu$ by analyzing the logarithm of the envelope's magnitude:
$$
\log|\rho_f(t)| \approx \log|A| - \nu t
$$
This establishes a linear relationship where the slope is precisely $-\nu$. Our numerical method implements this derivation: we compute the empirical ACF $\widehat{\rho}_f(t)$ from a long simulation, and then perform a linear fit on the slope of its log-envelope, $\log|\widehat{\rho}_f(t)|$, in the asymptotic (linear) tail region to extract $\nu(L)$. For better clarity, we use the median ratio on the control group $L_O$, which gives it a perfect linear relationship in comparison with $L$.

\paragraph{Numerical Setup.}

We model a particle on a two-dimensional torus $\mathbb{T}^2$ governed by a tilted double-well potential $U(x,y) = h_b/4(x^2-1)^2 + \epsilon/2 x + 1/2 y^2$ and a non-conservative rotational force $F = \alpha(-y, x)^T$. We fix the physical parameters to $\beta=1.0$ (inverse temperature), $\alpha=0.5$ (rotational strength, part of $L_{\pert}$), and $\epsilon=0.1$ (potential tilt). The potential's barrier height, $h_b$, is varied in the range $[3.0, 5.0]$ to create a set of target systems with progressively slower intrinsic dynamics.

The target effective dynamics are governed by the 2D overdamped generator $L_O$ (analogous to Eq. \eqref{eq:langevin_LO_target_example}). The baseline convergence rate of this system is given by its singular value gap, $s(L_O)$, which we compute by discretizing the operator on a $50 \times 50$ grid and finding its smallest non-zero singular value. The full underdamped Langevin dynamics (generator $L$ in Eq. \eqref{eq:langevin_L_full_example}) serves as the lifted system, where the particle's momentum $p = (p_x, p_y)$ is the auxiliary fast variable. The friction coefficient $\gamma$ corresponds to the fast dissipation scale in our $L = \gamma\mathcal{R} + \mathcal{V}$ decomposition.

We simulate this full system using a BAOAB integrator \cite{Kieninger2022}. The convergence rate $\nu(L)$ is quantified by extracting the asymptotic slope from the log-envelope of the ACF of the $x$-coordinate, as described in our methodology. For each barrier height $h_b$ (i.e., for each $s(L_O)$), we perform a scan over the friction coefficient $\gamma$ in the range $[0.1, 100]$ to identify the optimal friction $\gamma_{opt}$ that yields the maximal convergence rate, $\nu_{opt} = \nu(L(\gamma_{opt}))$.

\paragraph{Results and Analysis.}
\begin{figure}[!ht]
    \centering
    \includegraphics[width=\textwidth]{diffusion_speedup_verification.png}
    \caption{\textbf{Numerical Verification for Nonequilibrium Langevin.}
    \textbf{(a)} The non-equilibrium steady state (NESS) for a particle in a tilted double-well potential ($h_b=5.0$) with a rotational force. Color map indicates probability density (yellow is high), and white lines show probability currents.
    \textbf{(b)} Convergence rate comparison between $L$ (optimally lifted) and $L_O$ (collapsed). The lifted dynamics are consistently faster, providing a $1.23\times$ speedup in median \emph{despite detailed balance violation}.
    \textbf{(c)} Measured convergence rate $\nu$ (from ACF log-envelope slope) as a function of the friction coefficient $\gamma$ for $h_b=5.0$. The rate peaks at an optimal value, $\gamma_{\text{opt}} \approx 34$, confirming the existence of a finite optimal dissipation scale.
    \textbf{(d)} Log-log plot of convergence rates versus the singular gap $s(L_O)$. The optimal lifted rate $\nu_{opt}$ (purple circles) follows a power-law with slope $m_L \approx 0.54$ (black dashed line), close to the theoretical ballistic prediction ($m=1/2$, dotted line). The baseline collapsed rate $\nu(L_O)$ (green squares) scales linearly with slope $m_{L_O} \approx 1.00$ (green dashed line), confirming standard diffusive dynamics.}
    \label{fig:experiment}
\end{figure}

\textbf{Verification of Non-Equilibrium Robustness.} The results in Figure \ref{fig:experiment} provide direct numerical evidence that optimal acceleration \emph{survives detailed balance violation}. The steady-state distribution (Figure \ref{fig:experiment}a) exhibits persistent probability currents, confirming the system operates in a genuine NESS with $\eta = 0.5 > 0$. Despite this non-conservative perturbation, Figure \ref{fig:experiment}b demonstrates that the optimally lifted generator $L$ consistently outperforms the collapsed generator $L_O$, achieving a median speedup of $1.23\times$. This confirms the central claim of structural stability: the acceleration mechanism remains functional under non-equilibrium conditions.

\textbf{Optimal Dissipation Scale.} Figure \ref{fig:experiment}c validates the prediction (Corollary \ref{cor:near_optimal_gamma}) that acceleration is an \emph{optimally tuned} effect. For the system with $h_b=5.0$, the convergence rate $\nu(L)$ peaks sharply at $\gamma_{opt} \approx 34$, a finite dissipation scale distinct from both the overdamped ($\gamma \to \infty$) and Hamiltonian ($\gamma \to 0$) limits. This non-monotonic behavior confirms that the lifting structure requires precise balancing of timescales to achieve maximal acceleration.

\textbf{Near-Ballistic Scaling and Finite-$\eta$ Effects.} The log-log plot (Figure \ref{fig:experiment}d) provides the quantitative test of the scaling prediction $\nu_{opt} = \Theta(\sqrt{s(L_O)})$. The baseline rate $\nu(L_O)$ exhibits the expected linear scaling $m_{L_O} \approx 1.00$ (diffusive). In contrast, the optimal lifted rate follows $\nu_{opt} \propto s(L_O)^{m_L}$ with exponent $m_L \approx 0.54$, demonstrating a clear transition toward ballistic scaling.

\emph{Interpretation of Deviation from Theory:} The measured exponent $m_L \approx 0.54$ deviates slightly from the asymptotic prediction $m = 1/2$. We attribute this discrepancy to \textbf{finite-$\eta$ corrections}. According to Theorem~\ref{thm:scaling_limit_analysis}, the asymptotic limit $m = 1/2$ holds exactly only for $\eta = o(s^2)$. In our simulations, $\eta = 0.5$ is \emph{fixed}, while $s(L_O)$ varies in the range $[0.112, 0.157]$. For the smallest gap, $\eta/s^2 \approx 0.5/0.112^2 \approx 40$, indicating we are \emph{not} in the asymptotic regime $\eta \ll s^2$. The perturbative correction terms $\eta C_{corr}$ (appearing in the Flow Poincaré constant) and $-\eta \|L_{\pert}\|$ (in the effective rate $\nu_{\eff}$) remain non-negligible, introducing $O(\eta)$ degradation that shifts the effective exponent upward.

This empirical observation \emph{validates our perturbation theory}: the acceleration structure is fundamentally intact (exponent near $1/2$), but quantitatively perturbed by the $O(\eta)$ correction terms predicted in Section~\ref{sec:perturbative_stability}. The fact that the deviation is small ($\Delta m \approx 0.04$) and systematic confirms that we operate in the \textbf{marginal stability regime} $\eta = \Theta(s^2)$ identified in Section~\ref{sec:asymptotic_analysis}, where the mechanism is robust but not asymptotically pure.

\textbf{Conclusion.} This experiment demonstrates that the near-equilibrium lifting principle is \emph{structurally robust} in a classical non-equilibrium system. The optimal acceleration persists despite detailed balance violation, the finite optimal dissipation scale is empirically confirmed, and the observed scaling exponent quantitatively agrees with the predicted finite-$\eta$ corrections. These results validate both the theoretical safety radius $\eta < C_{corr}^{-1}$ and the practical relevance of the perturbative regime analysis.

\subsection{Example: Dissipative Zeno-Limit Spin Chain}
\label{sec:example_zeno}

We now instantiate the lifting structure with a canonical example from open quantum systems: a boundary-driven quantum spin chain operating in the Zeno regime \cite{Popkov2020}. This system is a direct quantum analogue of the classical Langevin case. A strong, fast dissipative process ($\gamma \mathcal{R}$) acts on a small part of the system (the boundaries), forcing the dynamics into a ``Zeno'' slow subspace. The ``slow'' coherent Hamiltonian evolution ($\mathcal{V}$), which contains both a main part and a non-conservative perturbation, acts as the coupling that, in the second order, gives rise to the non-equilibrium effective dynamics $L_O$.

\subsubsection{Dynamics and Generators}

We consider a quantum spin chain of length $N$ defined on the system Hilbert space $\mathcal{H}_{sys} = (\mathbb{C}^2)^{\otimes N}$. The dynamics of an observable $X \in \mathcal{B}(\mathcal{H}_{sys})$ are governed by a Lindblad master equation in the Heisenberg picture. The generator $\mathcal{L}$ admits a decomposition based on a separation of timescales:
\begin{equation}
    \mathcal{L} = \gamma \mathcal{R} + \mathcal{V}, \label{eq:zeno_L_full_example}
\end{equation}
where $\gamma \gg 1$ is a dimensionless parameter quantifying the strength of the dissipation.

The ``fast'' generator $\mathcal{R}$ describes the interaction with boundary reservoirs, driving the system rapidly toward a specific boundary configuration. It is defined as the sum of two local dissipators:
\begin{equation}
    \mathcal{R}(X) = \sum_{k \in \{1, N\}} \left( L_k^\dagger X L_k - \frac{1}{2} \{ L_k^\dagger L_k, X \} \right).
\end{equation}
The jump operators are chosen as $L_1 = \sigma_1^-$ and $L_N = \sigma_N^+$, corresponding to the strong polarization of the first site to the state $|\!\downarrow\rangle$ and the $N$-th site to $|\!\uparrow\rangle$. The kernel of this operator defines the \emph{slow subspace} $\mathcal{H}_S := \ker(\mathcal{R})$. In this setup, $\mathcal{H}_S$ is isomorphic to the algebra of observables on the inner $N-2$ spins, embedded in the full space as $X = |\!\downarrow\rangle\langle\downarrow|_1 \otimes \tilde{X} \otimes |\!\uparrow\rangle\langle\uparrow|_N$.

The ``slow'' dynamics are generated by the coupling term $\mathcal{V}(X) = i[H, X]$, which describes coherent evolution. To model a non-equilibrium scenario, we consider a Hamiltonian of the form $H = H_0 + \eta H_1$, where $\eta$ is a perturbative parameter satisfying $\eta \ll 1$.
\begin{enumerate}
    \item[(i)] $H_0$ represents a reversible interaction, such as the standard XXZ coupling:
    \begin{equation}
        H_0 = \sum_{j=1}^{N-1} \left( \sigma_j^x \sigma_{j+1}^x + \sigma_j^y \sigma_{j+1}^y + \Delta \sigma_j^z \sigma_{j+1}^z \right).
    \end{equation}
    \item[(ii)] $\eta H_1$ introduces a non-conservative perturbation, such as a Dzyaloshinskii-Moriya (DM) interaction \cite{Fert2023} or a symmetry-breaking boundary field \cite{Landi2022}, which renders the effective dynamics non-reversible.
\end{enumerate}
This structure induces a decomposition of the coupling term $\mathcal{V} = \mathcal{L}_{ham} + \eta \mathcal{L}_{pert}$, with $\mathcal{L}_{ham}(X) = i[H_0, X]$ and $\mathcal{L}_{pert}(X) = i[H_1, X]$.

In the limit $\gamma \to \infty$, adiabatic elimination yields an effective generator acting on $\mathcal{H}_S$. Due to the specific parity symmetries of $H_0$ and the boundary states, the first-order contribution vanishes, i.e., $P_S \mathcal{V} P_S = 0$, where $P_S$ is the orthogonal projection onto $\mathcal{H}_S$. Consequently, the dominant slow dynamics are governed by the second-order effective generator $L_O$:
\begin{equation}
    L_O = - P_S \mathcal{V} \mathcal{R}^+ \mathcal{V} P_S,
\end{equation}
where $\mathcal{R}^+$ denotes the Moore-Penrose pseudoinverse of $\mathcal{R}$ restricted to the fast subspace $\mathcal{H}_S^\perp$.
Restricted to the diagonal subalgebra spanned by the effective basis states $\{|\alpha\rangle\}$ of the inner chain, $L_O$ acts as a classical Markov generator. Its action on a diagonal observable $f = \sum_\alpha f_\alpha |\alpha\rangle\langle\alpha| \in \mathcal{H}_S$ is given by
\begin{equation}
    (L_O f)_\alpha = \sum_{\beta \neq \alpha} w_{\beta\alpha} (f_\beta - f_\alpha), \label{eq:zeno_LO_target_example}
\end{equation}
where $w_{\beta\alpha}$ denotes the transition rate from state $|\alpha\rangle$ to $|\beta\rangle$. The expansion $w_{\beta\alpha} = w_{\beta\alpha}^{(0)} + \eta w_{\beta\alpha}^{(1)} + O(\eta^2)$ explicitly captures the non-equilibrium nature of the target dynamics.

\subsubsection{Hilbert Space Structure and Decomposition}

We analyze the dynamics in the Liouville space of operators $\mathcal{H} = \mathcal{B}(\mathcal{H}_{sys})$, equipped with the Hilbert-Schmidt inner product $\langle A, B \rangle = \text{Tr}(A^\dagger B)$. The generator is decomposed as $\mathcal{L} = \gamma \mathcal{R} + \mathcal{V}$, where the dissipative part $\gamma\mathcal{R}$ and the coupling $\mathcal{V}$ are given by:
\begin{align}
    \gamma\mathcal{R} (X) &= \gamma \left( \mathcal{D}^\dagger[\sigma_1^-](X) + \mathcal{D}^\dagger[\sigma_N^+](X) \right), \\
    \mathcal{V} (X) &= i[H_0, X] + i[\eta H_1, X].
\end{align}

To satisfy the structural assumptions of the Near-Equilibrium Lifting framework (Definition \ref{def:ae_lift_structure}), we identify the \emph{slow subspace} $\mathcal{H}_S$ with the kernel of the fast dissipation $\mathcal{R}$. Physically, this corresponds to the ``Zeno subspace'' of operators invariant under the boundary dissipation:
\begin{equation}
    \mathcal{H}_S := \ker(\mathcal{R}) = \left\{ X \in \mathcal{B}(\mathcal{H}_{sys}) \mathrel{\Big|} X = P_{bd} X P_{bd} \right\},
\end{equation}
where $P_{bd} = |\!\downarrow\rangle\langle\downarrow\!|_1 \otimes \mathbb{I}_{bulk} \otimes |\!\uparrow\rangle\langle\uparrow\!|_N$ is the projection onto the steady state of the boundary spins. The orthogonal projection $P_S: \mathcal{H} \to \mathcal{H}_S$ acts as the conditional expectation onto this subalgebra.

The \emph{fast subspace} is defined as the orthogonal complement $\mathcal{H}_F := \mathcal{H}_S^\perp$. Since $\mathcal{R}$ consists of local dissipators with unique steady states on the boundaries, it is strictly coercive on $\mathcal{H}_F$; that is, there exists $\lambda_R > 0$ such that $-\text{Re}\langle X, \mathcal{R} X \rangle \geq \lambda_R \|X\|_{\mathcal{H}}^2$ for all $X \in \mathcal{H}_F$.

The effective dynamics on $\mathcal{H}_S$ are determined by the perturbative action of $\mathcal{V}$. We observe that for the specific interaction Hamiltonian $H$ (e.g., XX coupling), the first-order term vanishes:
\begin{equation}
    P_S \mathcal{V} P_S = 0.
\end{equation}
This follows because the Hamiltonian terms (e.g., $\sigma_1^x \sigma_2^x$) map states from the boundary-polarized Zeno subspace to states with excitations on the boundaries, which lie entirely in $\mathcal{H}_F$. Consequently, the effective generator is given by the second-order limit $L_O = - P_S \mathcal{V} \mathcal{R}^+ \mathcal{V} P_S$.

While $\mathcal{H}_S$ generally contains both populations and coherences of the inner spins, the secular approximation (valid when the energy splitting in the effective Hamiltonian is large compared to the linewidth) allows us to decouple the diagonal elements. Restricting $L_O$ to the diagonal subalgebra yields the classical Markov generator described in Eq. (\ref{eq:zeno_LO_target_example}), with transition rates $w_{\beta\alpha}$ derived rigorously via the trace formula on $\mathcal{H}_S$.

\subsubsection{Verification of Near-Equilibrium Lifting Conditions}

We now verify that the spin chain generator $\mathcal{L} = \gamma\mathcal{R} + \mathcal{V}$ satisfies the structural requirements of the Near-Equilibrium Lifting framework.

\begin{proposition}[Verification of Structural Conditions]
Let $\mathcal{H}_S = \ker(\mathcal{R})$ be the slow subspace defined above. The generator satisfies:
\begin{enumerate}[(i)]
    \item Orthogonality: $P_S \mathcal{V} P_S = 0$.
    \item Effective Generator: The restriction of the second-order effective operator $L_O = -P_S \mathcal{V} \mathcal{R}^+ \mathcal{V} P_S$ to the diagonal subalgebra corresponds to the classical rate generator defined in Eq. \eqref{eq:zeno_LO_target_example}.
\end{enumerate}
\end{proposition}

\begin{proof}
\textit{(i) Verification of Orthogonality.}
We show that $\mathcal{V}$ maps the slow subspace $\mathcal{H}_S$ entirely into the fast subspace $\mathcal{H}_F$. Let $X \in \mathcal{H}_S$. By the characterization of the kernel, $X$ takes the form $X = P_{bd} \otimes \tilde{X}$, where $P_{bd}$ is the boundary steady-state projector.
The action of the coupling term is given by $\mathcal{V}(X) = i[H, X]$. The Hamiltonian $H = H_0 + \eta H_1$ contains terms such as $\sigma_1^x \sigma_2^x$ (from the XX interaction) that couple the boundary spins to the bulk. Such operators flip the state of the boundary spins; for instance, $\sigma_1^x$ maps the ground state $|\!\downarrow\rangle_1$ to the excited state $|\!\uparrow\rangle_1$.
Consequently, the operator $[H, X]$ has no support on the boundary steady state configuration; formally, $P_{bd} [H, X] P_{bd} = 0$. Since the projection $P_S$ acts as the conditional expectation onto the boundary steady states, we have
\begin{equation}
    P_S \mathcal{V} P_S X = P_S (i[H, X]) = 0.
\end{equation}
Thus, $P_S \mathcal{V} P_S = 0$.

\textit{(ii) Derivation of the Effective Generator.}
We compute the matrix elements of the effective generator $L_O = - P_S \mathcal{V} \mathcal{R}^+ \mathcal{V} P_S$ in the basis of the slow subspace. We focus on the diagonal subalgebra spanned by the projections $X_\alpha = |\alpha\rangle\langle\alpha|$ corresponding to the computational basis states of the inner spins.
The transition rate from state $|\alpha\rangle$ to $|\beta\rangle$ is given by the matrix element $F_{\beta\alpha}$ of the generator. Using the Hilbert-Schmidt inner product $\langle A, B \rangle = \text{Tr}(A^\dagger B)$, we have:
\begin{equation}
    F_{\beta\alpha} = \text{Tr}(X_\beta L_O(X_\alpha)) = - \text{Tr}\left( X_\beta \mathcal{V} \mathcal{R}^+ \mathcal{V} (X_\alpha) \right).
\end{equation}
Substituting the decomposition $\mathcal{V}(\cdot) = i[H_0 + \eta H_1, \cdot]$ and expanding to first order in $\eta$, we obtain:
\begin{equation}
    F_{\beta\alpha} = w_{\beta\alpha}^{(0)} + \eta w_{\beta\alpha}^{(1)} + O(\eta^2).
\end{equation}
The zeroth-order term corresponds to the reversible dynamics:
\begin{equation}
    w_{\beta\alpha}^{(0)} = - \text{Tr}\left( X_\beta i[H_0, \mathcal{R}^+(i[H_0, X_\alpha])] \right).
\end{equation}
Due to the Hermiticity of $H_0$ and the preservation of detailed balance by the boundary dissipators in the absence of the perturbation, this term is symmetric, i.e., $w_{\beta\alpha}^{(0)} = w_{\alpha\beta}^{(0)}$.
The first-order correction arises from the cross terms:
\begin{equation}
    w_{\beta\alpha}^{(1)} = - \text{Tr}\left( X_\beta \left( i[H_0, \mathcal{R}^+(i[H_1, X_\alpha])] + i[H_1, \mathcal{R}^+(i[H_0, X_\alpha])] \right) \right).
\end{equation}
The interaction $H_1$ (e.g., Dzyaloshinskii-Moriya) breaks the parity symmetry of the Hamiltonian. Consequently, $w_{\beta\alpha}^{(1)} \neq w_{\alpha\beta}^{(1)}$, introducing non-reciprocity into the transition rates.

Finally, the action of $L_O$ on an observable $f = \sum_\alpha f_\alpha X_\alpha$ is recovered by linearity:
\begin{align}
    L_O(f) &= \sum_{\alpha} f_\alpha \sum_{\beta} F_{\beta\alpha} X_\beta = \sum_{\beta} X_\beta \left( \sum_{\alpha} F_{\beta\alpha} f_\alpha \right).
\end{align}
Probability conservation implies $\sum_{\beta} F_{\beta\alpha} = 0$ (trace preservation of the pre-dual). This enforces the diagonal constraint $F_{\alpha\alpha} = - \sum_{\beta \neq \alpha} F_{\beta\alpha}$. Denoting the off-diagonal rates by $w_{\beta\alpha} := F_{\beta\alpha}$ for $\beta \neq \alpha$, we obtain the standard Master equation form:
\begin{equation}
    (L_O f)_\beta = \sum_{\alpha \neq \beta} w_{\beta\alpha} f_\alpha - \left(\sum_{\gamma \neq \beta} w_{\gamma\beta}\right) f_\beta = \sum_{\alpha \neq \beta} w_{\beta\alpha} (f_\alpha - f_\beta).
\end{equation}
This matches Eq. \eqref{eq:zeno_LO_target_example}, confirming that the effective dynamics correspond to a classical Markov chain with non-equilibrium rates induced by $\eta H_1$.
\end{proof}

\subsubsection{Numerical Verification}
\label{sec:numerical_zeno}

To provide a concrete validation of our theoretical framework, particularly the prediction of an optimal convergence rate and the $\nu = \Theta(\sqrt{s(L_O)})$ scaling (Corollary \ref{cor:optimal_time_prefactor_ae}), we perform a numerical experiment on the non-reversible Zeno-limit spin chain. This experiment is designed to test two central, non-trivial predictions of the lifting structure: (i) the existence of an optimal, finite dissipation scale $\gamma$ that maximizes the convergence rate $\nu(L)$, and (ii) a quantitative, super-linear scaling relationship between this accelerated rate and the intrinsic rate $s(L_O)$ of the target effective system.

\paragraph{Numerical Estimate of Convergence Rate.}
A critical aspect of this numerical validation is the accurate measurement of the asymptotic convergence rate $\nu(L)$. For a non-reversible NESS, the generator $L$ is non-self-adjoint. Consequently, its spectrum is complex, and the autocorrelation function (ACF) of an observable $f$ is not a simple sum of positive decaying exponentials. Instead, it exhibits oscillatory decay.

The dynamics of any observable $f$ (projected onto the subspace orthogonal to the steady state) can be decomposed in the eigenbasis of the generator $L$ with invariant measure $\pi$. For $t$ large enough, the dynamics are dominated by the slowest-decaying modes (i.e., the eigenvalues $\lambda_j$ with the largest real part, $\text{Re}(\lambda_j) = -\nu$). For a non-reversible system, these eigenvalues may be complex, $\lambda_{slow} = -\nu \pm i\omega$. The ACF's asymptotic behavior is thus:
$$
\rho_f(t) = \frac{\langle f, e^{tL} f \rangle_\pi}{\|f\|_\pi^2} \approx A e^{-\nu t} \cos(\omega t + \phi), \quad t \to \infty
$$
While $\rho_f(t)$ itself oscillates, the asymptotic convergence rate $\nu$ governs the exponential decay of its envelope. We can therefore extract $\nu$ by analyzing the logarithm of the envelope's magnitude:
$$
\log|\rho_f(t)| \approx \log|A| - \nu t
$$
This establishes a linear relationship where the slope is precisely $-\nu$. Our numerical method implements this derivation: we compute the empirical ACF $\widehat{\rho}_f(t)$ from a long Quantum Jump Monte Carlo (QJMC) trajectory, and then perform a linear fit on the slope of its log-envelope, $\log|\widehat{\rho}_f(t)|$, in the asymptotic (linear) tail region to extract $\nu(L)$. For better clarity, we use the median ratio on the control group $L_O$, which gives it a perfect linear relationship in comparison with $L$.

\paragraph{Numerical Setup.}

We model a 3-spin chain ($N=3$) governed by the Lindbladian in Eq. \eqref{eq:zeno_L_full_example}. The Hamiltonian is $H = J H_0 + D H_1$, where $H_0 = \sum_{j=1,2} (\sigma_j^x\sigma_{j+1}^x + \sigma_j^y\sigma_{j+1}^y)$ is the XX coupling and $H_1 = \sigma_1^y\sigma_2^z - \sigma_2^z\sigma_1^y$ is a non-conservative DM-like interaction. We fix the non-conservative strength $D=0.5$ (our $\eta$) and vary the main coupling strength, $J$, in the range $[0.2, 1.2]$ to create a set of target systems with progressively slower intrinsic dynamics.

The target effective dynamics are governed by the 2-state classical generator $L_O$ (Eq. \eqref{eq:zeno_LO_target_example}) acting on the $M=N-2=1$ inner spin. We compute this singular gap of $L_O$ by first numerically constructing the full $64 \times 64$ superoperator $L_O = -P_S\mathcal{V}\mathcal{R}^+\mathcal{V}P_S$ using the matrix pseudoinverse, and then extracting the $2 \times 2$ classical rate matrix $F$ and computing its smallest non-zero singular value.

The full 3-spin Lindblad dynamics (generator $L$ in Eq. \eqref{eq:zeno_L_full_example}) serves as the lifted system. The states of the boundary spins (sites 1 and 3) serve as the auxiliary fast variables. The dissipation strength $\gamma$ corresponds to the fast dissipation scale $\gamma$ in our $L = \gamma\mathcal{R} + \mathcal{V}$ decomposition.

We simulate this full system using a Quantum Jump Monte Carlo (QJMC) trajectory \cite{Lambert2026}. The convergence rate $\nu(L)$ is quantified by extracting the asymptotic slope from the log-envelope of the ACF of the ``slow'' observable $O = \sigma_2^z$ (the magnetization of the inner spin), as described in our methodology. For a direct comparison, the convergence rate of the collapsed system, $\nu(L_O)$, is also computed via the same ACF log-envelope method, but applied to a separate QJMC simulation of the $2 \times 2$ classical dynamics. For each coupling $J$ (i.e., for each $s(L_O)$), we perform a scan over the dissipation $\gamma$ in the range $[10^{-1.5}, 10^{2.0}]$ to identify the optimal dissipation $\gamma_{opt}$ that yields the maximal convergence rate, $\nu_{opt} = \nu(L(\gamma_{opt}))$.

\paragraph{Results and Analysis.}
\begin{figure}[ht!]
    \centering
    \includegraphics[width=\textwidth]{quantum_speedup_verification.png}
    \caption{\textbf{Numerical Verification for Nonequilibrium Spin Chain.}
    \textbf{(a)} The non-equilibrium steady state (NESS) of the inner spin ($N=2$) for a system with $J \approx 0.9$ (matching $\gamma_{opt} \approx 11.99$ from the label). The non-zero off-diagonal elements and the non-zero classical current ($\approx -7.8\times 10^{-13}$) confirm the non-equilibrium nature.
    \textbf{(b)} Convergence rate comparison ($\nu$ vs. $\nu$) between $L$ and $L_O$. The lifted dynamics are consistently faster, providing a $\approx 1.40\times$ speedup in median (max $2.30\times$) \emph{despite detailed balance violation}.
    \textbf{(c)} Measured convergence rate $\nu$ (from ACF log-envelope slope) as a function of the dissipation strength $\gamma$ for $J=1.2$. The rate peaks at an optimal value, $\gamma_{\text{opt}} \approx 1.4$, confirming the existence of a finite optimal dissipation scale.
    \textbf{(d)} Log-log plot of convergence rates versus the singular gap $s(L_O)$. The optimal lifted rate $\nu_{opt}$ (purple circles) follows a power-law with slope $m_L \approx 0.43$ (black dashed line), close to the theoretical ballistic prediction ($m=1/2$, dotted line). The baseline collapsed rate $\nu(L_O)$ (green squares) scales linearly with slope $m_{L_O} \approx 1.00$ (green dashed line), confirming standard diffusive dynamics.
    }
    \label{fig:quantum_experiment}
\end{figure}

\textbf{Verification of Non-Equilibrium Robustness in Quantum Systems.} Figure \ref{fig:quantum_experiment} demonstrates that the structural stability of optimal acceleration extends to \emph{genuine quantum non-equilibrium systems}. The steady-state density matrix (Figure \ref{fig:quantum_experiment}a) exhibits non-zero coherences and non-zero classical probability current ($\approx -7.8\times 10^{-13}$), confirming operation in a NESS with finite perturbation $D = 0.5$ (our $\eta$). Despite this violation of detailed balance, Figure \ref{fig:quantum_experiment}b shows that the optimally lifted Lindbladian $\mathcal{L}$ consistently outperforms the collapsed generator $L_O$, achieving a median speedup of $1.40\times$ (maximum $2.30\times$). This \emph{robustness} confirms that the near-equilibrium lifting mechanism is not an artifact of equilibrium or classical systems, but a fundamental principle applicable to open quantum dynamics.

\textbf{Optimal Dissipation Scale.} Figure \ref{fig:quantum_experiment}c validates the optimal tuning prediction for the quantum system. For $J=1.2$, the convergence rate $\nu(\mathcal{L})$ peaks sharply at $\gamma_{opt} \approx 1.4$. This finite optimal dissipation scale is \emph{neither} the weak-dissipation limit ($\gamma \to 0$, where coherent dynamics dominate) \emph{nor} the strong-dissipation limit ($\gamma \to \infty$, reducing to the slow classical dynamics). The existence of this intermediate optimum is a hallmark of hypocoercivity and confirms the non-trivial interplay between quantum coherence and dissipation predicted by the lifting structure.

\textbf{Near-Ballistic Scaling and Finite-$\eta$ Effects.} The log-log plot (Figure \ref{fig:quantum_experiment}d) provides the quantitative scaling test. The baseline rate $\nu(L_O)$ exhibits perfect linear scaling $m_{L_O} \approx 1.00$ (diffusive), as expected for a classical Markov chain. In contrast, the optimal lifted rate follows $\nu_{opt} \propto s(L_O)^{m_L}$ with exponent $m_L \approx 0.43$, demonstrating a significant transition toward ballistic scaling.

\emph{Interpretation of Deviation from Theory:} The measured exponent $m_L \approx 0.43$ deviates from the asymptotic prediction $m = 1/2$ by $\Delta m \approx 0.07$. This deviation is \emph{larger} than in the Langevin case ($\Delta m \approx 0.04$), which is consistent with the stronger relative perturbation in the quantum system. Here, $D = 0.5$ acts on an already small spectral gap $s(L_O) \sim 10^{-11}$, placing the system firmly in the regime $\eta \gg s^2$ where finite-$\eta$ corrections dominate.

According to Theorem~\ref{thm:scaling_limit_analysis}, the pure asymptotic scaling $m = 1/2$ requires $\eta = o(s^2)$. In our quantum simulations, the ratio $\eta/s^2$ is extremely large, placing us in the \textbf{marginal stability regime} where the spectral gap remains positive ($\eta < C_{corr}^{-1}$, confirmed by the observed acceleration) but perturbative corrections are significant. The perturbative terms $\eta C_{corr}$ (degrading the Poincaré constant) and $-\eta \|\mathcal{L}_{\pert}\|$ (reducing the effective rate) introduce $O(\eta)$ drag that shifts the effective exponent \emph{downward} from $1/2$.

Crucially, this \emph{validates the perturbation theory}: the deviation is not random noise, but a systematic effect predicted by the finite-$\eta$ analysis in Section~\ref{sec:perturbative_stability}. The fact that $m_L \approx 0.43$ remains significantly below the diffusive value $m = 1$ confirms that the core ballistic mechanism is intact, merely degraded by the predicted $O(\eta)$ corrections. The slightly larger deviation in the quantum case ($0.43$ vs. $0.54$ in Langevin) is consistent with the stronger relative perturbation, providing a quantitative consistency check.

\textbf{Conclusion.} This quantum experiment provides compelling evidence that near-equilibrium lifting is \emph{structurally robust across physical platforms}. The optimal acceleration persists despite detailed balance violation in a genuine quantum NESS, the finite optimal dissipation scale is empirically confirmed in the quantum regime, and the observed scaling exponents quantitatively agree with the predicted finite-$\eta$ corrections. Together with the Langevin results, these experiments establish that the theoretical safety radius $\eta < C_{corr}^{-1}$ and the perturbative regime classification (near-equilibrium vs. marginal vs. breakdown) are \emph{practically relevant}, not merely asymptotic abstractions.

\subsection{Summary of Numerical Validation}

The two numerical experiments provide convergent evidence for the central thesis of this work: \textbf{optimal acceleration via lifting is structurally stable under non-equilibrium perturbations}.

\begin{itemize}
    \item \textbf{Robustness Confirmed}: In both classical (Langevin) and quantum (spin chain) systems operating in genuine NESS with finite $\eta > 0$, the optimally lifted dynamics achieve consistent speedup ($1.23\times$ and $1.40\times$ median, respectively) over the baseline collapsed dynamics. The acceleration mechanism \emph{survives} detailed balance violation.
    
    \item \textbf{Optimal Tuning Confirmed}: Both systems exhibit a clear optimal dissipation scale $\gamma_{opt}$ (finite and non-monotonic), validating Corollary~\ref{cor:near_optimal_gamma}. This demonstrates that maximal acceleration requires precise balancing of timescales, not asymptotic limits.
    
    \item \textbf{Perturbative Corrections Confirmed}: The measured scaling exponents ($m_L \approx 0.54$ for Langevin, $m_L \approx 0.43$ for spin chain) deviate systematically from the asymptotic prediction $m = 1/2$ by amounts consistent with finite-$\eta$ effects. Specifically:
    \begin{itemize}
        \item The deviations are \emph{small} ($\Delta m \lesssim 0.1$), confirming proximity to the ballistic regime.
        \item The deviations are \emph{systematic} (downward for quantum, upward for classical), consistent with the sign of perturbative drag in each system.
        \item The relative magnitude ($\Delta m_{quantum} > \Delta m_{classical}$) correlates with the relative perturbation strength, as predicted by the $O(\eta)$ correction formulas in Section~\ref{sec:perturbative_stability}.
    \end{itemize}
    
    \item \textbf{Practical Relevance of Theory}: The experiments validate that the regime classification (Definition~\ref{def:near_eq_limit}, Section~\ref{sec:asymptotic_analysis}) has empirical content:
    \begin{itemize}
        \item The \emph{near-equilibrium regime} $\eta = o(s^2)$ yields pure asymptotic scaling $m = 1/2$.
        \item The \emph{marginal stability regime} $\eta = \Theta(s^2)$ (where our experiments operate) yields robust acceleration with $O(1)$ degradation.
        \item The \emph{breakdown regime} $\eta \geq C_{corr}^{-1}$ would destroy the mechanism (not observed, confirming $\eta < C_{corr}^{-1}$).
    \end{itemize}
\end{itemize}

These results establish that the theoretical framework is not only mathematically rigorous but also \emph{predictive and falsifiable}. The perturbative analysis correctly anticipates the finite-$\eta$ degradation observed in practice, while the stability analysis correctly identifies the operational regime where acceleration remains robust. This validates the near-equilibrium lifting structure as a practical design principle for accelerating convergence in complex non-equilibrium systems, both classical and quantum.

\section{Conclusion}
\label{sec:conclusion}

\subsection{Summary: A Stability Analysis for Non-Equilibrium Lifting}

This work addresses a fundamental question in the theory of lifted Markov processes: \textbf{Is the quadratic acceleration mechanism structurally stable under non-equilibrium perturbations?}

Previous work \cite{Eberle2024a,Li2025} established that optimal lifting can achieve ballistic convergence rates $\nu = \Theta(\sqrt{s})$ in reversible (detailed balance) systems—a dramatic speedup over the diffusive baseline $\nu = \Theta(s)$. However, these results relied crucially on symmetry: the self-adjointness of generators, the positivity of eigenvalues, and the variational characterization of spectral gaps. For non-equilibrium steady states (NESS), where detailed balance is broken, none of these tools apply. The spectrum becomes complex, the generator is non-normal, and the fundamental question remained open: does the $\sqrt{s}$ acceleration survive, or is it an artifact of reversibility?

We have provided a definitive answer: \textbf{optimal acceleration is robust}. The quadratic speedup $\nu_{opt} = \Theta(\sqrt{s(L_O)})$ persists in non-equilibrium systems, provided the symmetry-breaking perturbation satisfies the near-equilibrium condition $\eta = o(s(L_O)^2)$. This is not a marginal result—it establishes that the core mechanism of hypocoercive lifting transcends detailed balance.

\paragraph{Theoretical Framework.} Our approach synthesizes three distinct theories:

\begin{enumerate}
    \item \textbf{Adiabatic Elimination} (Section~\ref{sec:adiabatic_elimination}): We formalize the perturbative structure $\mathcal{L} = \gamma\mathcal{R} + \mathcal{V}$ arising from timescale separation, deriving the effective slow generator $L_O = -P_S \mathcal{V} \mathcal{R}^+ \mathcal{V} P_S$ via the Generalized Schrieffer-Wolff transformation. This provides the algebraic foundation.
    
    \item \textbf{Lifting Theory} (Section~\ref{sec:ae_to_lifting}): We invert the adiabatic elimination perspective, interpreting the full generator $\mathcal{L}$ as a \emph{lift} of the slow dynamics $L_O$. The key insight is the \textbf{Approximate Quadratic Form} (AQF, Assumption~\ref{assump:AQF}), which bounds the deviation from reversibility: $|E(X,Y)| \le \eta C_{AQF} \|X\| \|Y\|$. This quantifies the price of breaking detailed balance.
    
    \item \textbf{Hypocoercivity} (Section~\ref{sec:perturbative_stability}): We adapt the Flow Poincaré inequality framework to the near-equilibrium structure. The \emph{stability criterion} $\eta < C_{corr}^{-1}$ (Theorem~\ref{thm:flow_poincare_ae}) ensures the Poincaré constant remains strictly positive despite non-conservative perturbations. This yields explicit lower bounds on the convergence rate.
\end{enumerate}

\paragraph{Main Results.} The technical contributions are threefold:

\begin{itemize}
    \item \textbf{Spectral Stability} (Theorem~\ref{thm:upper_bound_ae_v2}): The spectral gap satisfies $s(\mathcal{L}) \le C_1 \sqrt{s(L_O)} + C_2 \eta$. For $\eta = o(s^2)$, the leading term dominates, preserving the ballistic scaling.
    
    \item \textbf{Stability Criterion} (Theorem~\ref{thm:flow_poincare_ae}): The Flow Poincaré constant $\alpha_T(\eta) = \left[ \frac{(\gamma \tilde{A}_1 + \tilde{A}_2)^2}{(1 - \eta C_{corr})^2} + \frac{1}{\lambda_R} \right]^{-1}$ remains strictly positive for $\eta < C_{corr}^{-1} = \Theta(s^2)$. As $\eta \to C_{corr}^{-1}$, the constant degrades to zero—this is the \emph{breakdown threshold}.
    
    \item \textbf{Optimal Scaling} (Corollary~\ref{cor:optimal_time_prefactor_ae}): Under the near-equilibrium condition $\eta = o(s^2)$, the optimal rate satisfies $\nu_{opt} = \Theta(\sqrt{s(L_O)})$ with bounded prefactor $C_T = \Theta(1)$.
\end{itemize}

\paragraph{Physical Interpretation.} Section~\ref{sec:asymptotic_analysis} provides the bridge to physical intuition. The condition $\eta \ll s^2$ has a clear meaning: the non-equilibrium drive operates on a timescale $\tau_{pert} = \eta^{-1}$ much longer than the square of the slow relaxation time, $\tau_{slow}^2 = s^{-2}$. Within the intrinsic slow timescale, the system experiences only a fractional perturbation $\sim \eta/s = o(s)$, allowing it to behave \emph{almost reversibly}. We classify three regimes:
\begin{itemize}
    \item \textbf{Near-Equilibrium} ($\eta = o(s^2)$): Pure ballistic scaling with vanishing corrections.
    \item \textbf{Marginal Stability} ($\eta = \Theta(s^2)$, $\eta < C_{corr}^{-1}$): Robust acceleration with $O(1)$ degradation.
    \item \textbf{Breakdown} ($\eta \ge C_{corr}^{-1}$): Poincaré constant vanishes, lifting fails.
\end{itemize}

\paragraph{Numerical Validation.} Section~\ref{sec:examples} provides experimental confirmation in two canonical systems: classical Langevin dynamics with a non-conservative rotational force and a quantum Zeno-limit spin chain with Dzyaloshinskii-Moriya interaction. Both exhibit:
\begin{itemize}
    \item Robust speedup (1.23× and 1.40× median) despite detailed balance violation.
    \item Finite optimal dissipation scales $\gamma_{opt}$ (non-monotonic optimization).
    \item Near-ballistic scaling exponents ($m_L \approx 0.54$ and $0.43$) with deviations quantitatively explained by finite-$\eta$ corrections.
\end{itemize}
Crucially, the experiments operate in the \emph{marginal stability regime}, where perturbative corrections are significant but controlled—validating the perturbation theory as predictive, not merely asymptotic.

\subsection{Limitations and Scope}

We emphasize that this is a \textbf{perturbative result}. The framework requires $\eta$ to be small relative to $s^2$ to guarantee stability. Three key limitations must be acknowledged:

\begin{enumerate}
    \item \textbf{No Far-from-Equilibrium Theory}: For $\eta = \Theta(1)$ or $\eta \gg s^2$, the current analysis breaks down. Strongly driven systems may exhibit fundamentally different physics—phase transitions, symmetry-breaking bifurcations, or dynamical instabilities—that cannot be captured by perturbative expansions around equilibrium structures.
    
    \item \textbf{Analytical, Not Constructive}: Our approach validates the convergence of a \emph{given} physical system $\mathcal{L}$ by interpreting it as a lift. It does not provide a general algorithm to \emph{design} an optimal lift for an arbitrary target NESS generator $L_O$. The inverse problem—constructing $\mathcal{R}$ and $\mathcal{V}$ from $L_O$—remains open.
    
    \item \textbf{Asymptotic Regime}: The pure scaling $\nu_{opt} = \Theta(\sqrt{s})$ holds exactly only in the limit $\eta = o(s^2)$. For finite systems or moderate perturbations (the marginal regime), $O(\eta)$ corrections are non-negligible. While the mechanism is robust, quantitative predictions require accounting for these corrections.
\end{enumerate}

These limitations are not defects—they define the precise scope of validity. The framework establishes a \textbf{safety radius} $\eta < C_{corr}^{-1}$ within which optimal acceleration is guaranteed. Beyond this radius, new physical phenomena emerge that demand new mathematical tools.

\subsection{Future Directions}

This work opens several natural extensions and challenges:

\paragraph{1. Non-Perturbative Methods for Strongly Driven Systems.} 
The most pressing open question is: \emph{What happens for large $\eta$?} Candidate approaches include:
\begin{itemize}
    \item \textbf{Dilation Theory} \cite{Hu2023,LiSX2025}: Embed the non-equilibrium generator into a higher-dimensional reversible system, recovering self-adjointness at the cost of enlarged state space.
    \item \textbf{Resolvent Methods} \cite{Gonzalez2024}: Analyze the resolvent $(z - \mathcal{L})^{-1}$ directly, avoiding spectral decomposition.
    \item \textbf{Linear Response Beyond Perturbation} \cite{Ban2017,Albert2016}: Develop resummation techniques or non-perturbative bounds exploiting causality and fluctuation-dissipation relations.
\end{itemize}
A breakthrough here would extend the applicability to far-from-equilibrium settings (e.g., heat engines, active matter).

\paragraph{2. Constructive Lifting Design.} 
Invert the paradigm: given an arbitrary target NESS generator $L_O$, \emph{construct} an optimal lift $\mathcal{L}$. This requires identifying:
\begin{itemize}
    \item The minimal fast degrees of freedom (dimension of $\mathcal{H}_F$).
    \item The coupling structure $\mathcal{V}$ ensuring the Orthogonality and AQF conditions.
    \item The dissipator $\mathcal{R}$ achieving the optimal intertwining scaling $K_3 = \Theta(\sqrt{s})$.
\end{itemize}
This connects to \textbf{reservoir engineering} \cite{Naseem2025,Sellem2024,Janovitch2025}: designing dissipative environments to accelerate convergence.

\paragraph{3. Hierarchical and Iterative Lifting.} 
Apply lifting recursively: identify fast variables for the \emph{lifted} system, creating a hierarchy $L_O \to \mathcal{L}^{(1)} \to \mathcal{L}^{(2)} \to \cdots$. If each level achieves $\sqrt{s}$ speedup, this could yield exponential acceleration. The challenge: maintaining structural conditions at each level.

\paragraph{4. Quantum-Specific Extensions.} 
Leverage quantum coherence more explicitly:
\begin{itemize}
    \item \textbf{Floquet Engineering} \cite{Pan2020}: Periodic driving to modify effective Hamiltonians.
    \item \textbf{Quantum Zeno Acceleration} \cite{Popkov2020}: Strong measurement-induced subspace projection.
    \item \textbf{Entanglement-Enhanced Lifting}: Use many-body entanglement as a resource for acceleration.
\end{itemize}

\paragraph{5. Algorithmic Implementation.} 
Translate the theoretical framework into practical quantum algorithms. Recent advances in variational quantum algorithms \cite{Cerezo2021} and quantum optimal control \cite{Glaser2015} provide natural platforms for implementing near-equilibrium lifts in quantum hardware.

\subsection{Broader Impact}

This work contributes to an emerging synthesis between \textbf{mathematical hypocoercivity} \cite{Villani2009,Li2024,Fang2025} and \textbf{physical open quantum systems} \cite{Breuer2002,Rivas2012}. By establishing that optimal acceleration survives non-equilibrium perturbations, we validate hypocoercive lifting as a \emph{universal mechanism}, not confined to idealized reversible models. This has implications for:

\begin{itemize}
    \item \textbf{Quantum Simulation}: Accelerating convergence to NESS in quantum simulators.
    \item \textbf{Quantum Computing}: Faster state preparation and error correction protocols.
    \item \textbf{Statistical Mechanics}: Understanding how timescale separation accelerates relaxation in driven systems.
    \item \textbf{Algorithm Design}: Principles for constructing optimal Markov Chain Monte Carlo (MCMC) samplers in non-equilibrium settings.
\end{itemize}

The central message is one of \textbf{robustness}: the mathematical structures enabling optimal acceleration are not fragile idealizations but stable features that persist under realistic perturbations. This transforms hypocoercivity from an elegant theory of reversible systems into a practical tool for engineering fast dynamics in the complex, dissipative systems that dominate physical reality.




















\bibliography{ref}
\newpage
\appendix

\section{Proof for Preliminary}
\label{app:prelim_proof}

\begin{lemma}[Criterion for Strict Hypocoercivity, cf. Lemma \ref{lem:kernel_collapse_hypo}]
\label{app:kernel_collapse_proof}
    For an ergodic QMS, the fixed-point space is a subspace of the dissipative kernel, $\mathcal{F}(\mathcal{L})\subset \ker(\mathcal{S})$. The semigroup is strictly hypocoercive if and only if this inclusion is strict, meaning there exists an element in the dissipative kernel that is not a fixed point:
    \begin{align}
        \ker(\mathcal{S}) \supsetneq \mathcal{F}(\mathcal{L}).
    \end{align}
    Equivalently, the codimension of $\mathcal{F}(\mathcal{L})$ in $\ker(\mathcal{S})$ is positive. This characterization applies to both finite and infinite-dimensional systems.
\end{lemma}
\begin{proof}
    When $X\in\mathcal{F}(\mathcal{L})$, we have $\mathcal{L}(X)=0$, thus:
    \begin{align}
        \langle X,\mathcal{S}(X)\rangle_{\sigma,1/2}=\Re\langle X,\mathcal{L}(X)\rangle_{\sigma,1/2}=0.\label{kernel collapse of hypo--proof--eq1}
    \end{align}
    From $\mathcal{P}_t$ is ergodic, from Lemma \ref{sec:qms_ergodicity}, we know $\mathcal{S}$ must be negative, combined with \eqref{kernel collapse of hypo--proof--eq1}, we have $\mathcal{S}(X)=0$, namely $X\in\ker(\mathcal{S})$. Thus, $\mathcal{F}(\mathcal{L})\subset \ker(\mathcal{S})$.

    When $\mathcal{P}_t$ is ergodic with invariant measure $\sigma$, we have \eqref{def:hypocoercivity} holds with $C\geq 1$. It suffices to show $\ker(\mathcal{S})=\mathcal{F}(\mathcal{L})$ is equivalent to $C=1$. 
    
    Assuming $\mathcal{P}_t$ is coercive, we take time derivative and apply Grönwall's inequality to the following inequality:
    \begin{align}
         \|\mathcal{P}_{t}(X)-\mathbb{E}_{\mathcal{F}}(X)\|_{2,\sigma}\leq e^{-2\nu t}\|X-\mathbb{E}_{\mathcal{F}}(X)\|_{2,\sigma},
    \end{align}
    which indicates that when $C=1$ we have
    \begin{align}
    \label{kernel collapse of hypo--proof--eq2}
        \nu\|X-\mathbb{E}_{\mathcal{F}}(X)\|_{2,\sigma}^2\leq-\langle X,\mathcal{S}(X)\rangle_{\sigma,1/2}.
    \end{align}
    Apparently, \eqref{kernel collapse of hypo--proof--eq2} implies $ \ker(\mathcal{S})\subset\mathcal{F}(\mathcal{L})$.

    On the other hand, assuming $\ker(\mathcal{S})=\mathcal{F}(\mathcal{L})$. Then, $\mathbb{E}_{\mathcal{F}}$ coincides with $\mathbb{E}_{\ker(\mathcal{S})}$, and $\mathcal{F}(\mathcal{L})^\perp=\ker(\mathcal{S})^\perp$. Moreover, from the $\mathcal{S}\leq 0$, there exists $\nu>0$ such that:
    $\nu\|X-\mathbb{E}_{\ker(\mathcal{S})}(X)\|_{2,\sigma}^2\leq-\langle X,\mathcal{S}(X)\rangle_{\sigma,1/2},$
    for all $X\in\ker(\mathcal{S})^\perp$. Thus, for all $X\in\mathcal{F}(\mathcal{L})^\perp$, we also have:
    \begin{align}
         \nu\|X-\mathbb{E}_{\mathcal{F}}(X)\|_{2,\sigma}^2\leq-\langle X,\mathcal{S}(X)\rangle_{\sigma,1/2}.
    \end{align}
    Note that \eqref{kernel collapse of hypo--proof--eq2} holds trivially on $\mathcal{F}(\mathcal{L})$, combined with equivalence established above, we deduce $C=1$.
\end{proof}

\section{Derivation of the Reference Dynamics and Orthogonality Condition}
\label{sec:appendix_reference}

In Section \ref{sec:setup}, we introduced the reference generator $L_O = -P_S \mathcal{V} \mathcal{R}^+ \mathcal{V} P_S$ and assumed the orthogonality condition $P_S \mathcal{V} P_S = 0$. This appendix rigorously justifies these assumptions using the Generalized Schrieffer--Wolff (GSW) formalism \cite{Kessler2012}. We show that:
\begin{enumerate}
\item The reference generator $L_O$ arises naturally as the \emph{second-order effective generator} in the adiabatic elimination regime $\gamma \gg 1$.
\item The orthogonality condition $P_S \mathcal{V} P_S = 0$ is precisely the condition required for $L_O$ to be the leading-order non-trivial dynamics on the slow subspace.
\item The full dynamics $\mathcal{P}_t$ can be approximated by the reference dynamics $e^{tL_O}$ with error $O(1/\gamma)$, validating $L_O$ as a physically meaningful baseline.
\end{enumerate}

\subsection{GSW Expansion: Standard Derivation}

We begin with the generator decomposition $\mathcal{L} = \gamma \mathcal{L}_{\text{ham}} + \eta \mathcal{L}_{\text{pert}}$, where:
\begin{itemize}
\item $\mathcal{L}_{\text{ham}} = -i[H, \cdot]$ is the Hamiltonian generator with fast subspace $\mathcal{H}_F = \ker(\mathcal{L}_{\text{ham}})^\perp$ and slow subspace $\mathcal{H}_S = \ker(\mathcal{L}_{\text{ham}})$.
\item $\gamma \gg 1$ is the timescale separation parameter.
\item $\eta \geq 0$ parametrizes the perturbation strength.
\end{itemize}
We denote $\mathcal{R} := \mathcal{L}_{\text{ham}}$ and $\mathcal{V} := \mathcal{L}_{\text{pert}}$. Let $P_S$ be the orthogonal projection onto $\mathcal{H}_S$ and $P_F = \mathbf{1} - P_S$. The pseudo-inverse $\mathcal{R}^+ = (P_F \mathcal{R} P_F)^{-1}$ is well-defined since $\mathcal{R}$ has spectral gap $\lambda_R > 0$ on $\mathcal{H}_F$.

\begin{theorem}[GSW Effective Generator]
\label{thm:app_gsw_effective}
Consider the generator $\mathcal{L} = \gamma \mathcal{R} + \mathcal{V}$ with $\gamma \gg 1$ and $\mathcal{R}$ self-adjoint satisfying $\ker(\mathcal{R}) = \mathcal{H}_S \neq \{0\}$. The effective generator $\mathcal{L}_{\text{eff}}$ on $\mathcal{H}_S$ has the perturbative expansion:
\begin{align}
\mathcal{L}_{\text{eff}} = L^{(1)} + \frac{1}{\gamma} L^{(2)} + O(1/\gamma^2),
\end{align}
where the first- and second-order terms are:
\begin{align}
L^{(1)} &= P_S \mathcal{V} P_S, \\
L^{(2)} &= -P_S \mathcal{V} \mathcal{R}^+ \mathcal{V} P_S.
\end{align}
\end{theorem}
\begin{proof}
We rescale time via $t' = \gamma t$ and define the rescaled generator $\mathcal{L}' = \frac{1}{\gamma}\mathcal{L} = \mathcal{R} + \epsilon \mathcal{V}$ with $\epsilon := 1/\gamma \ll 1$. The GSW formalism \cite{Kessler2012} provides a similarity transformation $L' = e^{\widehat{\mathcal{S}}} \mathcal{L}' e^{-\widehat{\mathcal{S}}}$ that block-diagonalizes the generator, where $\widehat{\mathcal{S}}(A) = [\mathcal{S}, A]$ is the adjoint action of a block-off-diagonal superoperator $\mathcal{S} = \sum_{n=1}^\infty \epsilon^n \mathcal{S}_n$.

The transformed generator has the exact form \cite{Kessler2012}:
\begin{align}
L' = \mathcal{L}'^{\text{diag}} + \tanh(\widehat{\mathcal{S}}/2) \mathcal{L}'^{\text{offdiag}},
\end{align}
where $\mathcal{L}'^{\text{diag}}$ and $\mathcal{L}'^{\text{offdiag}}$ denote the block-diagonal and block-off-diagonal parts with respect to the decomposition $\mathcal{H} = \mathcal{H}_S \oplus \mathcal{H}_F$. The decoupling condition $L'^{\text{offdiag}} = 0$ requires:
\begin{align}
\sinh(\widehat{\mathcal{S}}) \mathcal{L}'^{\text{diag}} + \cosh(\widehat{\mathcal{S}}) \mathcal{L}'^{\text{offdiag}} = 0.
\end{align}

Expanding to first order in $\epsilon$, we have $\mathcal{L}'^{\text{diag}} = \mathcal{R} + O(\epsilon)$ and $\mathcal{L}'^{\text{offdiag}} = \epsilon \mathcal{V}^{\text{offdiag}} + O(\epsilon^2)$, where $\mathcal{V}^{\text{offdiag}} = P_S \mathcal{V} P_F + P_F \mathcal{V} P_S$. Matching $O(\epsilon)$ terms gives:
\begin{align}
[\mathcal{S}_1, \mathcal{R}] = -\mathcal{V}^{\text{offdiag}}.
\end{align}
Using the block-matrix form $\mathcal{S}_1 = \begin{pmatrix} 0 & \mathcal{S}_1^{-} \\ \mathcal{S}_1^{+} & 0 \end{pmatrix}$ and $\mathcal{R} = \begin{pmatrix} 0 & 0 \\ 0 & \mathcal{R}_F \end{pmatrix}$, we solve:
\begin{align}
\mathcal{S}_1^{-} = -P_S \mathcal{V} P_F \mathcal{R}^+, \qquad \mathcal{S}_1^{+} = \mathcal{R}^+ P_F \mathcal{V} P_S.
\end{align}

The effective generator on $\mathcal{H}_S$ is $\mathcal{L}'_{\text{eff}} = P_S L' P_S$. Expanding the hyperbolic tangent $\tanh(\widehat{\mathcal{S}}/2) = \frac{1}{2}\widehat{\mathcal{S}} + O(\widehat{\mathcal{S}}^3)$ and collecting terms:
\begin{align}
L'^{\text{diag}} = (\mathcal{R} + \epsilon \mathcal{V}^{\text{diag}}) + \frac{\epsilon}{2} [\mathcal{S}_1, \epsilon \mathcal{V}^{\text{offdiag}}] + O(\epsilon^3).
\end{align}
Projecting onto $\mathcal{H}_S$, the zeroth-order term vanishes ($P_S \mathcal{R} P_S = 0$), yielding:
\begin{align}
\mathcal{L}'_{\text{eff}} = \epsilon \cdot P_S \mathcal{V} P_S + \frac{\epsilon^2}{2} P_S [\mathcal{S}_1, \mathcal{V}^{\text{offdiag}}] P_S + O(\epsilon^3).
\end{align}
Computing the commutator and using $\mathcal{V}^{\text{offdiag}} = P_S \mathcal{V} P_F + P_F \mathcal{V} P_S$:
\begin{align}
P_S [\mathcal{S}_1, \mathcal{V}^{\text{offdiag}}] P_S = -2 P_S \mathcal{V} P_F \mathcal{R}^+ P_F \mathcal{V} P_S.
\end{align}
Thus, $\mathcal{L}'_{\text{eff}} = \epsilon (P_S \mathcal{V} P_S) - \epsilon^2 (P_S \mathcal{V} \mathcal{R}^+ \mathcal{V} P_S) + O(\epsilon^3)$.

Reversing the time rescaling, $\mathcal{L}_{\text{eff}} = \gamma \mathcal{L}'_{\text{eff}}$:
\begin{align}
\mathcal{L}_{\text{eff}} = P_S \mathcal{V} P_S + \frac{1}{\gamma} (-P_S \mathcal{V} \mathcal{R}^+ \mathcal{V} P_S) + O(1/\gamma^2). \qedhere
\end{align}
\end{proof}

\subsection{The Orthogonality Condition and Reference Generator}

The expansion in Theorem \ref{thm:app_gsw_effective} reveals the crucial role of the first-order term $L^{(1)} = P_S \mathcal{V} P_S$. If this term is non-zero, it dominates the effective dynamics at $O(1)$, and the second-order term $L^{(2)}$ is merely a $O(1/\gamma)$ correction. However, our stability analysis requires the \emph{second-order term} to be the leading-order dynamics. This necessitates:

\begin{definition}[Orthogonality Condition]
\label{def:app_orthogonality}
The perturbation $\mathcal{V}$ satisfies the \textbf{orthogonality condition} if
\begin{align}
P_S \mathcal{V} P_S = 0.
\end{align}
Physically, this means the perturbation induces no direct transitions within the slow subspace; all slow-subspace dynamics arise from virtual excursions into the fast subspace.
\end{definition}

\begin{remark}[Justification of the Reference Generator]
When the orthogonality condition holds, Theorem \ref{thm:app_gsw_effective} yields:
\begin{align}
\mathcal{L}_{\text{eff}} = \frac{1}{\gamma} L^{(2)} + O(1/\gamma^2) = \frac{1}{\gamma} \left( -P_S \mathcal{V} \mathcal{R}^+ \mathcal{V} P_S \right) + O(1/\gamma^2).
\end{align}
Rescaling time by $\gamma$ (i.e., measuring time in units of the slow timescale), the leading-order effective generator is:
\begin{align}
L_O := -P_S \mathcal{V} \mathcal{R}^+ \mathcal{V} P_S.
\end{align}
This is precisely the reference generator introduced in Section \ref{sec:setup}. The orthogonality condition $P_S \mathcal{V} P_S = 0$ is \emph{not} an arbitrary restriction---it is the condition under which the second-order contribution becomes the physically relevant baseline dynamics.
\end{remark}

\subsection{Approximation Error: Validation of the Reference Model}

We now prove that the full quantum dynamics $\mathcal{P}_t = e^{t\mathcal{L}}$ can be approximated by the reference dynamics $e^{tL_O}$ with error $O(1/\gamma)$. This justifies using $L_O$ as the reference point for stability analysis.

\begin{theorem}[Approximation Error of the Reference Dynamics]
\label{thm:app_approx_error}
Let $\mathcal{L} = \gamma \mathcal{R} + \mathcal{V}$ with $\mathcal{R}$ self-adjoint having spectral gap $\lambda_R > 0$ on $\mathcal{H}_F$. Assume $P_S \mathcal{V} P_S = 0$ (orthogonality condition) and $\gamma \lambda_R > \|\mathcal{V}\|$. Define the reference dynamics $\mathcal{P}_O(t) = \exp(tL_O)$ where $L_O = -P_S \mathcal{V} \mathcal{R}^+ \mathcal{V} P_S$.

For any initial state $X(0) \in \mathcal{H}_S$ and time $t \geq 0$, the approximation error is bounded by:
\begin{align}
\|X(t) - X_O(t)\|_{2,\sigma} \leq \frac{C_{\mathcal{V}}'}{\gamma \lambda_R} \left( 1 + t \left( \|\mathcal{V}\| + \frac{\|\mathcal{V}\|^2}{C_{\mathcal{V}}'} \right) \right) \|X(0)\|_{2,\sigma},
\end{align}
where $X(t) = \mathcal{P}_t X(0)$, $X_O(t) = \mathcal{P}_O(t) X(0)$, and $C_{\mathcal{V}}' = \frac{\|\mathcal{V}\|}{1 - \|\mathcal{V}\|/(\gamma \lambda_R)} = O(\|\mathcal{V}\|)$.
\end{theorem}

\begin{proof}
We decompose the error into two orthogonal components:
\begin{align}
X(t) - X_O(t) = \underbrace{P_F \mathcal{P}_t X(0)}_{X_F(t) \text{ (leakage)}} + \underbrace{(P_S \mathcal{P}_t - \mathcal{P}_O(t)) X(0)}_{Z(t) \text{ (intrinsic error)}}.
\end{align}
By the Pythagorean theorem, $\|X(t) - X_O(t)\|_{2,\sigma}^2 = \|X_F(t)\|_{2,\sigma}^2 + \|Z(t)\|_{2,\sigma}^2$. We bound each term separately.

\paragraph{Step 1: Bound on Leakage $X_F(t)$.}
Rescale time via $t' = \gamma t$ and define $\mathcal{L}' = \mathcal{R} + \epsilon \mathcal{V}$ with $\epsilon = 1/\gamma$. Let $\mathcal{P}'_{t'} = \mathcal{P}_{t'/\gamma}$. Using Duhamel's formula (cf. Lemma \ref{lem:gronwall_volterra}):
\begin{align}
\mathcal{P}'_{t'} = e^{t'\mathcal{R}} + \int_0^{t'} e^{(t'-s)\mathcal{R}} (\epsilon \mathcal{V}) \mathcal{P}'_s \, ds.
\end{align}
Since $X(0) \in \mathcal{H}_S = \ker(\mathcal{R})$, we have $e^{t'\mathcal{R}} X(0) = X(0)$, so $P_F e^{t'\mathcal{R}} X(0) = 0$. Thus:
\begin{align}
X_F(t') = \epsilon \int_0^{t'} e^{(t'-s)\mathcal{R}} P_F \mathcal{V} (X_S(s) + X_F(s)) \, ds,
\end{align}
where $X_S(s) = P_S \mathcal{P}'_s X(0)$ and $X_F(s) = P_F \mathcal{P}'_s X(0)$. Using $P_S \mathcal{V} P_S = 0 \implies P_F \mathcal{V} P_S = \mathcal{V} P_S$:
\begin{align}
X_F(t') = \epsilon \int_0^{t'} e^{(t'-s)\mathcal{R}} \mathcal{V} X_S(s) \, ds + \epsilon \int_0^{t'} e^{(t'-s)\mathcal{R}} P_F \mathcal{V} X_F(s) \, ds.
\end{align}
Let $f(t') = \|X_F(t')\|_{2,\sigma}$. Using $\|e^{(t'-s)\mathcal{R}} P_F\| \leq e^{-(t'-s)\lambda_R}$ and contractivity $\|X_S(s)\|_{2,\sigma} \leq \|X(0)\|_{2,\sigma}$:
\begin{align}
f(t') \leq \left( \epsilon \frac{\|\mathcal{V}\|}{\lambda_R} \|X(0)\|_{2,\sigma} \right) + \epsilon \|\mathcal{V}\| \int_0^{t'} e^{-(t'-s)\lambda_R} f(s) \, ds.
\end{align}
This is a Volterra integral inequality. Applying Grönwall's lemma (Lemma \ref{lem:gronwall_volterra}) with $A = \epsilon \|\mathcal{V}\| \|X(0)\|_{2,\sigma} / \lambda_R$, $C = \epsilon \|\mathcal{V}\|$, $\lambda = \lambda_R$, and using $C < \lambda$ (ensured by $\gamma \lambda_R > \|\mathcal{V}\|$):
\begin{align}
f(t') \leq \frac{A \lambda_R}{\lambda_R - C} = \frac{\|\mathcal{V}\|}{\gamma \lambda_R - \|\mathcal{V}\|} \|X(0)\|_{2,\sigma}. \label{eq:app_leakage_bound}
\end{align}
Returning to original time $t = t'/\gamma$, this bound is uniform in $t$.

\paragraph{Step 2: Bound on Intrinsic Error $Z(t)$.}
Let $X_S(t) = P_S \mathcal{P}_t X(0)$. The slow component evolves as:
\begin{align}
\frac{d}{dt} X_S(t) = P_S \mathcal{L} \mathcal{P}_t X(0) = P_S (\gamma \mathcal{R} + \mathcal{V})(X_S(t) + X_F(t)).
\end{align}
Since $P_S \mathcal{R} = 0$ and $P_S \mathcal{V} P_S = 0$, this simplifies to $\frac{d}{dt} X_S(t) = P_S \mathcal{V} X_F(t)$.

The reference dynamics satisfies $\frac{d}{dt} X_O(t) = L_O X_O(t)$. The error $Z(t) = X_S(t) - X_O(t)$ obeys:
\begin{align}
\frac{d}{dt} Z(t) = P_S \mathcal{V} X_F(t) - L_O X_O(t) = L_O Z(t) + \mathcal{E}(t),
\end{align}
where $\mathcal{E}(t) = P_S \mathcal{V} X_F(t) - L_O X_S(t)$. By Duhamel's principle:
\begin{align}
Z(t) = \int_0^t e^{(t-s)L_O} \mathcal{E}(s) \, ds.
\end{align}
Using contractivity $\|e^{(t-s)L_O}\| \leq 1$:
\begin{align}
\|Z(t)\|_{2,\sigma} \leq \int_0^t \|\mathcal{E}(s)\|_{2,\sigma} \, ds \leq \int_0^t \left( \|\mathcal{V}\| \|X_F(s)\|_{2,\sigma} + \|L_O\| \|X_S(s)\|_{2,\sigma} \right) ds.
\end{align}
Using $\|L_O\| \leq \|\mathcal{V}\|^2 / \lambda_R$, contractivity $\|X_S(s)\|_{2,\sigma} \leq \|X(0)\|_{2,\sigma}$, and the leakage bound \eqref{eq:app_leakage_bound}:
\begin{align}
\|\mathcal{E}(s)\|_{2,\sigma} \leq \left( \frac{\|\mathcal{V}\|^2}{\gamma \lambda_R - \|\mathcal{V}\|} + \frac{\|\mathcal{V}\|^2}{\lambda_R} \right) \|X(0)\|_{2,\sigma}.
\end{align}
Integrating over $[0, t]$:
\begin{align}
\|Z(t)\|_{2,\sigma} \leq t \left( \frac{\|\mathcal{V}\|^2}{\gamma \lambda_R - \|\mathcal{V}\|} + \frac{\|\mathcal{V}\|^2}{\gamma \lambda_R} \right) \|X(0)\|_{2,\sigma}. \label{eq:app_intrinsic_bound}
\end{align}

\paragraph{Step 3: Total Error.}
Combining \eqref{eq:app_leakage_bound} and \eqref{eq:app_intrinsic_bound}, and defining $C_{\mathcal{V}}' = \frac{\|\mathcal{V}\|}{1 - \|\mathcal{V}\|/(\gamma \lambda_R)}$:
\begin{align}
\|X(t) - X_O(t)\|_{2,\sigma} &\leq \|X_F(t)\|_{2,\sigma} + \|Z(t)\|_{2,\sigma} \\
&\leq \frac{C_{\mathcal{V}}'}{\gamma \lambda_R} \left( 1 + t \left( \|\mathcal{V}\| + \frac{\|\mathcal{V}\|^2}{C_{\mathcal{V}}'} \right) \right) \|X(0)\|_{2,\sigma}. \qedhere
\end{align}
\end{proof}

\begin{remark}[Physical Interpretation]
Theorem \ref{thm:app_approx_error} establishes that in the regime $\gamma \gg 1$ (strong timescale separation), the reference generator $L_O$ provides a faithful representation of the effective slow dynamics. The error is $O(1/\gamma)$ at $t=0$ and grows only linearly in time. This validates our choice of $L_O$ as the physically meaningful baseline for stability analysis in Section \ref{sec:stability}.

The orthogonality condition $P_S \mathcal{V} P_S = 0$ is essential: without it, the effective dynamics would be dominated by the first-order term $P_S \mathcal{V} P_S$, and the second-order contribution $L_O$ would be merely a small correction rather than the fundamental slow-timescale generator.
\end{remark}







\begin{lemma}[Iterative Dyson Expansion]
\label{app:dyson_lemma}
    Let $\mathcal{L}' = \mathcal{A} + \mathcal{B}$ be a generator and $\mathcal{P}'_t = \exp(t\mathcal{L}')$. The semigroup satisfies the identity:
    \begin{align}
        \mathcal{P}'_t = \exp(t\mathcal{A}) \left( \mathbf{1} + \int_0^t dt_1 \exp(-t_1\mathcal{A}) \mathcal{B} \mathcal{P}'_{t_1} \right).
    \end{align}
\end{lemma}
\begin{proof}
    We use the ansatz $\mathcal{P}'_t = \exp(t\mathcal{A}) \mathcal{U}(t)$ for some superoperator $\mathcal{U}(t)$. Taking the time derivative, we have:
    \begin{align}
        \frac{d}{dt}\mathcal{P}'_t = \mathcal{L}'\mathcal{P}'_t = (\mathcal{A}+\mathcal{B})\exp(t\mathcal{A})\mathcal{U}(t).
    \end{align}
    Using the product rule on the ansatz, we also have:
    \begin{align}
        \frac{d}{dt}\mathcal{P}'_t = \mathcal{A}\exp(t\mathcal{A})\mathcal{U}(t) + \exp(t\mathcal{A})\frac{d}{dt}\mathcal{U}(t).
    \end{align}
    Equating the two expressions for the derivative yields:
    \begin{align}
        (\mathcal{A}+\mathcal{B})\exp(t\mathcal{A})\mathcal{U}(t) &= \mathcal{A}\exp(t\mathcal{A})\mathcal{U}(t) + \exp(t\mathcal{A})\frac{d}{dt}\mathcal{U}(t) \\
        \mathcal{B}\exp(t\mathcal{A})\mathcal{U}(t) &= \exp(t\mathcal{A})\frac{d}{dt}\mathcal{U}(t) \\
        \implies \frac{d}{dt}\mathcal{U}(t) &= \exp(-t\mathcal{A})\mathcal{B}\mathcal{P}'_t.
    \end{align}
    We integrate this differential equation from $0$ to $t$, using the initial condition $\mathcal{U}(0) = \mathbf{1}$ (since $\mathcal{P}'_0 = \mathbf{1}$):
    \begin{align}
        \mathcal{U}(t) = \mathbf{1} + \int_0^t dt_1 \exp(-t_1\mathcal{A})\mathcal{B}\mathcal{P}'_{t_1}.
    \end{align}
    Substituting this expression for $\mathcal{U}(t)$ back into the ansatz $\mathcal{P}'_t = \exp(t\mathcal{A}) \mathcal{U}(t)$ gives the desired result.
\end{proof}

\begin{lemma}[Grönwall's Inequality for Volterra's Equations]
\label{lem:gronwall_volterra}
    Let $f(t)$ be a non-negative, locally integrable function on $[0, \infty)$ satisfying the integral inequality:
    \begin{align}
        f(t) \le A + C \int_0^t e^{-\lambda(t-s)} f(s) ds
    \end{align}
    for constants $A \ge 0$, $C \ge 0$, and $\lambda > 0$. If $C < \lambda$, then $f(t)$ is uniformly bounded for all $t \ge 0$ by:
    \begin{align}
        f(t) \le \frac{A \lambda}{\lambda - C}.
    \end{align}
\end{lemma}
\begin{proof}
    Let $g(t) = e^{\lambda t} f(t)$. Multiplying the inequality by $e^{\lambda t}$, we obtain
    \begin{align}
        g(t) \le A e^{\lambda t} + C \int_0^t e^{\lambda s} f(s) ds = A e^{\lambda t} + C \int_0^t g(s) ds.
    \end{align}
    Let $G(t) = \int_0^t g(s) ds$. Then $G'(t) = g(t)$, and the inequality is $G'(t) \le A e^{\lambda t} + C G(t)$. This is a first-order linear differential inequality $G'(t) - C G(t) \le A e^{\lambda t}$. We multiply by the integrating factor $e^{-Ct}$:
    \begin{align}
        \frac{d}{dt} (G(t) e^{-Ct}) \le A e^{(\lambda - C) t}.
    \end{align}
    Integrating from $0$ to $t$, and using $G(0) = 0$, we find (assuming $\lambda \neq C$):
    \begin{align}
        G(t) e^{-Ct} \le \int_0^t A e^{(\lambda - C) s} ds = \frac{A}{\lambda - C} (e^{(\lambda - C) t} - 1).
    \end{align}
    Thus, $G(t) \le \frac{A}{\lambda - C} (e^{\lambda t} - e^{C t})$. Substituting this back into the inequality for $g(t)$:
    \begin{align}
        g(t) \le A e^{\lambda t} + C G(t) \le A e^{\lambda t} + \frac{AC}{\lambda - C} (e^{\lambda t} - e^{C t}) = \frac{A\lambda}{\lambda - C} e^{\lambda t} - \frac{AC}{\lambda - C} e^{C t}.
    \end{align}
    Finally, we recover $f(t) = g(t) e^{-\lambda t}$:
    \begin{align}
        f(t) \le \frac{A\lambda}{\lambda - C} - \frac{AC}{\lambda - C} e^{(C - \lambda) t}.
    \end{align}
    Since $C < \lambda$ by hypothesis, the exponent $(C - \lambda)$ is strictly negative. The second term is thus non-positive and decays to zero, which gives the uniform bound $f(t) \le \frac{A\lambda}{\lambda - C}$.
\end{proof}


\section{Proof of Perturbative Stability}
\label{sec:appendix_stability}

This appendix provides rigorous proofs for the stability theorems stated in Section \ref{sec:stability}. The central goal is to establish that the Flow Poincaré inequality---and consequently the exponential convergence rate---remains valid under non-equilibrium perturbations of strength $\eta$, provided $\eta$ satisfies an explicit stability criterion. We prove both upper and lower bounds on the perturbed decay rate, demonstrating continuity in $\eta$ and identifying the breakdown threshold.

\subsection{Preliminary: Spectral Gap and Decay Rate}

We begin with a standard lemma relating the exponential decay rate to the singular value gap of the generator.

\begin{lemma}[Decay Rate Bound via Singular Value Gap]
\label{lem:decay_rate_bound}
Let $\{P_t\}_{t \ge 0}$ be a hypocoercive semigroup with generator $L$ satisfying the exponential decay estimate $\|P_t x - P_\infty x\|_\mathcal{H} \le C e^{-\nu t}\,\|x - P_\infty x\|_\mathcal{H}$ for all $x \in \mathcal{H}$, where $\nu > 0$ is the decay rate and $C \ge 1$ is the prefactor. Then the decay rate is bounded by the singular value gap:
\begin{align}
    \nu \le (1 + \log C) s(L).
\end{align}
\end{lemma}

\begin{proof}
    The exponential decay bound $\|P_t x - P_\infty x\|_\mathcal{H} \le C e^{-\nu t}\,\|x - P_\infty x\|_\mathcal{H}$ is equivalent (with $T=(\log C)/\nu$) to
\begin{align}
\|P_t x - P_\infty x\|_\mathcal{H} \le e^{-\nu(t-T)}\,\|x - P_\infty x\|_\mathcal{H} \quad\text{for all }t\ge0.
\end{align}

Fix $T,\nu>0$ as above and let $x\in \mathcal{F}^\perp$ (so $P_\infty x=0$). Then
\begin{align}
\int_0^\infty \|P_t x\|_\mathcal{H}\,dt
\le
\int_0^T \|x\|_\mathcal{H}\,dt
+
\int_T^\infty e^{-\nu(t-T)}\|x\|_\mathcal{H}\,dt
= \left(T + \frac{1}{\nu}\right)\|x\|_\mathcal{H}.
\end{align}
Hence the operator $x\mapsto \int_0^\infty P_t x\,dt$ on $\mathcal{F}^\perp$ is well-defined and bounded, implying that $0$ belongs to the resolvent set of the restricted generator $L|_{\mathcal{F}^\perp}$. The inverse $(-L|_{\mathcal{F}^\perp})^{-1}$ satisfies:
\begin{align}
\|(-L|_{\mathcal{F}^\perp})^{-1}x\|_\mathcal{H} \le \int_0^\infty \|P_t x\|_\mathcal{H}\,dt \le \left(T + \frac{1}{\nu}\right)\|x\|_\mathcal{H}.
\end{align}
Using $T=(\log C)/\nu$, we obtain the spectral estimate:
\begin{align}
\frac{1}{s(L)}
= \sup_{y\in\mathrm{Dom}(L|_{\mathcal{F}^\perp})\setminus\{0\}} \frac{\|y\|_\mathcal{H}}{\|L|_{\mathcal{F}^\perp} y\|_\mathcal{H}}
= \sup_{x\in \mathcal{F}^\perp\setminus\{0\}} \frac{\|(-L|_{\mathcal{F}^\perp})^{-1}x\|_\mathcal{H}}{\|x\|_\mathcal{H}}
\le \frac{1}{\nu}\,(1+\log C).
\end{align}
Inverting this inequality gives the desired bound.
\end{proof}

\subsection{Upper Bound: Continuity of the Singular Value Gap}

The upper bound on the decay rate follows from bounding the singular value gap $s(L)$ of the perturbed generator. The key insight is that the Approximate Quadratic Form condition introduces an $\eta$-dependent correction to the gap, proving stability.

\begin{theorem}[Upper Bound on Perturbed Decay Rate]
\label{thm:app_upper_bound}
Let $L = \gamma \mathcal{L}_{\text{ham}} + \eta \mathcal{L}_{\text{pert}}$ satisfy the Near-Equilibrium Lifting Structure (Definition \ref{def:ae_lift_structure}) with Approximate Quadratic Form constant $C_{AQF}$. Let $s(L_O)$ denote the singular value gap of the reference generator $L_O = -P_S \mathcal{V} \mathcal{R}^+ \mathcal{V} P_S$, and let $C(L)$ be the prefactor in the hypocoercive estimate. Then the decay rate $\nu(L)$ satisfies:
\begin{align}
    \nu(L) \leq (1+\log C(L)) \sqrt{\frac{s(L_O) + \eta C_{AQF}}{s(\Pi_1 S \Pi_1)}}.
\end{align}
\end{theorem}
\begin{proof}
From Lemma \ref{lem:decay_rate_bound}, we know that $\nu(L) \le (1+\log C(L)) s(L)$, where $s(L)$ is the singular value gap of the perturbed generator $L$. Our strategy is to bound $s(L)$ from above by explicitly tracking the $\eta$-dependence introduced by the Approximate Quadratic Form condition.

\paragraph{Step 1: Restriction to the Slow Subspace.}

The singular value gap is defined as $s(L) = \inf\{\|LX\|_{\mathcal{H}} \mid X \in \mathrm{Dom}(L)\cap\mathcal{F}^\perp, \|X\|_\mathcal{H}=1\}$, where $\mathcal{F} = \ker(L) = \ker(L_O)$. Since the reference generator $L_O$ acts on the slow subspace $\mathcal{H}_S$ and $\ker(L_O) \subset \mathcal{H}_S$, we can restrict the infimum to $X \in \mathcal{H}_S$:
\begin{align}
    s(L) \leq \inf\{\|LX\|_\mathcal{H} \mid X \in \mathrm{Dom}(L_O)\cap\mathcal{F}^\perp \cap \mathcal{H}_S, \|X\|_\mathcal{H}=1\}.
\end{align}
For $X \in \mathcal{H}_S$, the orthogonality condition (Definition \ref{def:ae_lift_structure}) implies $LX = (\gamma\mathcal{R} + \mathcal{V})X = \mathcal{V}X$ (since $\mathcal{R} X = 0$ on $\mathcal{H}_S$). Thus, $\|LX\|_\mathcal{H} = \|\mathcal{V}X\|_\mathcal{H}$.

\paragraph{Step 2: Auxiliary Operator and Quadratic Form.}
Let $Y = LX = \mathcal{V}X \in \mathcal{H}_F$. By the orthogonality condition, $Y$ belongs to the range space $\overline{\operatorname{Ran}(\mathcal{V}|_{\mathcal{H}_S})} \subseteq \mathcal{H}_F$. Let $\Pi_1$ be the orthogonal projection onto this subspace, so $\Pi_1 Y = Y$. Define the restricted operator $\bar{S} = \Pi_1 S \Pi_1$, where $S$ is the auxiliary positive definite operator from the lifting structure. Let $s(\bar{S}) > 0$ denote its smallest singular value on $\operatorname{Ran}(\Pi_1)$.

Since $Y \in \operatorname{Ran}(\Pi_1)$, we have:
\begin{align}
    \langle Y, S Y \rangle_{\mathcal{H}} = \langle Y, \bar{S} Y \rangle_{\mathcal{H}} \ge s(\bar{S}) \|Y\|_{\mathcal{H}}^2.
\end{align}
Rearranging and taking the square root:
\begin{align}
    \|Y\|_{\mathcal{H}} = \|LX\|_{\mathcal{H}} \le s(\bar{S})^{-1/2} \langle LX, S LX \rangle_{\mathcal{H}}^{1/2}. \label{eq:app_upper_step2}
\end{align}

\paragraph{Step 3: Application of the Approximate Quadratic Form.}
This is the crucial step where the $\eta$-dependence enters. The Approximate Quadratic Form condition (Definition \ref{def:ae_lift_structure}, equation \eqref{eq:lift_quad_form_approx}) states that for $X \in \mathcal{H}_S$:
\begin{align}
    \langle LX, S LX \rangle_{\mathcal{H}} \le \langle X, |L_O| X \rangle_{\mathcal{H}_S} + \eta C_{AQF} \|X\|_{\mathcal{H}}^2.
\end{align}
Substituting this into \eqref{eq:app_upper_step2} with $\|X\|_\mathcal{H}=1$:
\begin{align}
    \|LX\|_{\mathcal{H}} \le s(\bar{S})^{-1/2} \left( \langle X, |L_O| X \rangle_{\mathcal{H}_S} + \eta C_{AQF} \right)^{1/2}. \label{eq:app_upper_aqf}
\end{align}
The first term $\langle X, |L_O| X \rangle_{\mathcal{H}_S}$ is the reference quadratic form (independent of $\eta$), while the second term $\eta C_{AQF}$ represents the \emph{perturbative correction} due to non-equilibrium driving.

\paragraph{Step 4: Optimization over the Slow Subspace.}
Taking the infimum over $X \in \mathrm{Dom}(L_O)\cap\mathcal{F}^\perp$ with $\|X\|_\mathcal{H}=1$:
\begin{align}
    s(L) &\leq s(\bar{S})^{-1/2} \left( \inf\left\{ \langle X, |L_O| X \rangle_{\mathcal{H}_S} \mid X \in \mathrm{Dom}(L_O)\cap\mathcal{F}^\perp, \|X\|_\mathcal{H}=1 \right\} + \eta C_{AQF} \right)^{1/2}.
\end{align}
The infimum of the Rayleigh quotient for $|L_O|$ over $\mathcal{F}^\perp = \ker(L_O)^\perp$ is precisely the smallest non-zero eigenvalue of $|L_O|$, which equals the singular value gap $s(L_O)$. Therefore:
\begin{align}
    s(L) &\le \sqrt{\frac{s(L_O) + \eta C_{AQF}}{s(\Pi_1 S \Pi_1)}}.
\end{align}

\paragraph{Step 5: Final Bound on Decay Rate.}
Applying Lemma \ref{lem:decay_rate_bound} with this bound on $s(L)$:
\begin{align}
    \nu(L) \leq (1+\log C(L)) \sqrt{\frac{s(L_O) + \eta C_{AQF}}{s(\Pi_1 S \Pi_1)}}.
\end{align}
\end{proof}

\begin{remark}[Stability Interpretation]
The bound in Theorem \ref{thm:app_upper_bound} reveals that the decay rate is a \emph{continuous function} of $\eta$. As $\eta \to 0$, the perturbative correction $\eta C_{AQF}$ vanishes, and we recover the reference bound:
\begin{align}
    \lim_{\eta \to 0} \nu(L) \leq (1+\log C(L_O)) \sqrt{\frac{s(L_O)}{s(\Pi_1 S \Pi_1)}}.
\end{align}
This confirms that small non-equilibrium perturbations induce only small changes in the convergence rate, validating the structural stability of the lifting construction. The breakdown occurs when $\eta C_{AQF}$ becomes comparable to $s(L_O)$, i.e., when the perturbation corrupts the reference gap.
\end{remark}

\subsection{Lower Bound: Flow Poincaré Inequality with Perturbative Corrections}

The lower bound on the decay rate requires establishing the Flow Poincaré inequality for the perturbed generator. The key technical challenge is to rigorously control the error introduced by the Approximate Quadratic Form condition and derive the explicit stability criterion $\eta < C_{corr}^{-1}$.

\subsubsection{Auxiliary Lemmas}

We first establish two technical lemmas that bound the perturbative error terms.

\begin{lemma}[Inner Product Reduction with Explicit Error]
\label{lem:app_inner_product_reduction}
Assume the Near-Equilibrium Lifting Structure (Definition \ref{def:ae_lift_structure}) with Approximate Quadratic Form constant $C_{AQF}$. For sufficiently regular paths $X_t, Y_t, Z_t$ in $\mathcal{H}_S$:
\begin{align}
    \langle \mathcal{A} X_t, Z_t + S \mathcal{V} Y_t \rangle_{T,\mathcal{H}} = -\langle \partial_t X_t, Z_t \rangle_{T,\mathcal{H}_S} + \mathcal{E}_{T,-|L_O|}(X_t, Y_t) + R_{T, \eta}(X, Y),
\end{align}
where the \textbf{remainder term} satisfies:
\begin{align}
    |R_{T, \eta}(X, Y)| \le \eta C_{AQF} \frac{1}{T} \int_0^T \|X_t\|_{\mathcal{H}} \|Y_t\|_{\mathcal{H}} \, dt.
\end{align}
\end{lemma}
\begin{proof}
The proof involves a direct calculation starting from the left-hand side (LHS). We substitute the definition of the operator $\mathcal{A} X_t = -\partial_t X_t + \mathcal{V} X_t$:
\begin{align}
    \text{LHS} &= \langle -\partial_t X_t + \mathcal{V} X_t, Z_t + S \mathcal{V} Y_t \rangle_{T,\mathcal{H}} \\
    &= \frac{1}{T}\int_0^T \langle -\partial_t X_t + \mathcal{V} X_t, Z_t + S \mathcal{V} Y_t \rangle_{\mathcal{H}} \, dt.
\end{align}
We expand the inner product inside the integral into four terms:
\begin{align}
    \text{LHS} = \frac{1}{T}\int_0^T \bigg( \underbrace{-\langle \partial_t X_t, Z_t \rangle_{\mathcal{H}}}_{\text{(i)}} \underbrace{-\langle \partial_t X_t, S \mathcal{V} Y_t \rangle_{\mathcal{H}}}_{\text{(ii)}} + \underbrace{\langle \mathcal{V} X_t, Z_t \rangle_{\mathcal{H}}}_{\text{(iii)}} + \underbrace{\langle \mathcal{V} X_t, S \mathcal{V} Y_t \rangle_{\mathcal{H}}}_{\text{(iv)}} \bigg) \, dt.
\end{align}
We analyze each term individually:

\textbf{Term (i):} Yields $-\langle \partial_t X_t, Z_t \rangle_{T,\mathcal{H}_S}$, as $X_t, Z_t \in \mathcal{H}_S$.

\textbf{Term (ii):} Vanishes because $\partial_t X_t \in \mathcal{H}_S$ and $S \mathcal{V} Y_t \in \mathcal{H}_F$ (using Orthogonality).

\textbf{Term (iii):} Vanishes because $\mathcal{V} X_t \in \mathcal{H}_F$ (using Orthogonality) and $Z_t \in \mathcal{H}_S$.

\textbf{Term (iv) -- Critical Application of AQF:} This term involves the quadratic form. For $X_t, Y_t \in \mathcal{H}_S$, we have $L X_t = \mathcal{V} X_t$ and $L Y_t = \mathcal{V} Y_t$. Thus, $\langle \mathcal{V} X_t, S \mathcal{V} Y_t \rangle_{\mathcal{H}} = \langle L X_t, S L Y_t \rangle_{\mathcal{H}}$. 

We now apply the explicit Approximate Quadratic Form Condition \eqref{eq:lift_quad_form_approx}. Define the \emph{pointwise error} for a fixed $t$ as:
\begin{align}
    E(X_t, Y_t) := \langle L X_t, S L Y_t \rangle_{\mathcal{H}} - \langle X_t, |L_O| Y_t \rangle_{\mathcal{H}_S}.
\end{align}
The AQF condition states $|E(X_t, Y_t)| \le \eta C_{AQF} \|X_t\|_{\mathcal{H}} \|Y_t\|_{\mathcal{H}}$. Thus:
\begin{align}
    \langle \mathcal{V} X_t, S \mathcal{V} Y_t \rangle_{\mathcal{H}} = \langle X_t, |L_O| Y_t \rangle_{\mathcal{H}_S} + E(X_t, Y_t).
\end{align}
By definition, $\mathcal{E}_{-|L_O|}(X_t, Y_t) = -\langle X_t, (-|L_O|) Y_t \rangle_{\mathcal{H}_S} = \langle X_t, |L_O| Y_t \rangle_{\mathcal{H}_S}$. Integrating term (iv) over time gives:
\begin{align}
    \frac{1}{T}\int_0^T \langle \mathcal{V} X_t, S \mathcal{V} Y_t \rangle_{\mathcal{H}} \, dt &= \frac{1}{T}\int_0^T \left( \mathcal{E}_{-|L_O|}(X_t, Y_t) + E(X_t, Y_t) \right) \, dt \\
    &= \mathcal{E}_{T,-|L_O|}(X_t, Y_t) + \frac{1}{T}\int_0^T E(X_t, Y_t) \, dt.
\end{align}

Define the \textbf{time-averaged remainder term}:
\begin{align}
    R_{T, \eta}(X, Y) := \frac{1}{T}\int_0^T E(X_t, Y_t) \, dt.
\end{align}
Then:
\begin{align}
    |R_{T, \eta}(X, Y)| = \left| \frac{1}{T}\int_0^T E(X_t, Y_t) \, dt \right| \le \frac{1}{T}\int_0^T |E(X_t, Y_t)| \, dt \le \eta C_{AQF} \frac{1}{T}\int_0^T \|X_t\|_{\mathcal{H}} \|Y_t\|_{\mathcal{H}} \, dt.
\end{align}

Combining the results for the four terms, we find:
\begin{align}
    \text{LHS} = -\langle \partial_t X_t, Z_t \rangle_{T,\mathcal{H}_S} + \mathcal{E}_{T,-|L_O|}(X_t, Y_t) + R_{T, \eta}(X, Y). \qedhere
\end{align}
\end{proof}

\begin{lemma}[Coupling Strength Estimates with Perturbative Corrections]
\label{lem:app_coupling_strength}
Under Assumption~\ref{assump:regularity_fpi_ae}, for sufficiently regular paths $X_t \in \mathrm{Dom}(\mathcal{R})$ and $Y_t \in \mathrm{Dom}(L_O)$, the following bounds hold:
\begin{align}
    |\langle \mathcal{R} X_t, S \mathcal{V} Y_t \rangle_{\mathcal{H}}| &\le \sqrt{\mathcal{E}_{\mathcal{R}}(X_t) (\mathcal{E}_{-|L_O|}(Y_t) + \eta C_{AQF}\|Y_t\|_{\mathcal{H}}^2)}, \\
    |\langle X_t-P_SX_t, \mathcal{V} Y_t \rangle_{\mathcal{H}}| &\le \sqrt{\mathcal{E}_{\mathcal{R}}(X_t) (\mathcal{E}_{-|L_O|}(Y_t) + \eta C_{AQF}\|Y_t\|_{\mathcal{H}}^2)}, \\
    |\langle \mathcal{V}(X_t-P_SX_t), S\mathcal{V} Y_t \rangle_{\mathcal{H}}| &\le \|X_t-P_SX_t\|_{\mathcal{H}} \left( K_2 \||L_O|Y_t\|_{\mathcal{H}_S} + K_3 \sqrt{\mathcal{E}_{-|L_O|}(Y_t)} \right).
\end{align}
\end{lemma}
\begin{proof}
We establish each inequality in turn, using the Cauchy-Schwarz inequality, the near-equilibrium lifting structure (Definition \ref{def:ae_lift_structure}), and Assumption \ref{assump:regularity_fpi_ae}. Let $X_t^F = X_t - P_SX_t$. Note that $\mathcal{E}_{\mathcal{R}}(X_t) = -\langle X_t, \mathcal{R} X_t \rangle = -\langle X_t^F, \mathcal{R} X_t^F \rangle = \mathcal{E}_{\mathcal{R}}(X_t^F)$.

\textbf{Proof of first inequality:} We manipulate the inner product using $S^{1/2}$.
\begin{align}
    \langle \mathcal{R} X_t, S \mathcal{V} Y_t \rangle &= \langle S^{1/2} \mathcal{R} X_t, S^{1/2} \mathcal{V} Y_t \rangle \\
    &\le \|S^{1/2} \mathcal{R} X_t \| \| S^{1/2} \mathcal{V} Y_t \| \quad (\text{by C-S}).
\end{align}
We evaluate the norms. First, $\|S^{1/2}\mathcal{R}X_t\|^2 = \langle \mathcal{R}X_t, S \mathcal{R}X_t \rangle = \langle X_t, \mathcal{R} S \mathcal{R} X_t \rangle$. Since $\mathcal{R}S\mathcal{R} = \mathcal{R}(-\mathcal{R}^+)\mathcal{R} = -\mathcal{R}$ on $\mathcal{H}_F$, this becomes $\langle X_t^F, -\mathcal{R} X_t^F \rangle = \mathcal{E}_{\mathcal{R}}(X_t^F) = \mathcal{E}_{\mathcal{R}}(X_t)$. Thus, $\|S^{1/2} \mathcal{R} X_t \| = \sqrt{\mathcal{E}_{\mathcal{R}}(X_t)}$.

Second, from the explicit Approximate Quadratic Form condition \eqref{eq:lift_quad_form_approx} with $X=Y=Y_t$:
\begin{align}
\|S^{1/2} \mathcal{V} Y_t \|^2 = \langle \mathcal{V} Y_t, S \mathcal{V} Y_t \rangle \le \langle Y_t, |L_O| Y_t \rangle + \eta C_{AQF}\|Y_t\|_{\mathcal{H}}^2 = \mathcal{E}_{-|L_O|}(Y_t) + \eta C_{AQF}\|Y_t\|_{\mathcal{H}}^2.
\end{align}
Combining these gives:
\begin{align}
    |\langle \mathcal{R} X_t, S \mathcal{V} Y_t \rangle| \le \sqrt{\mathcal{E}_{\mathcal{R}}(X_t) (\mathcal{E}_{-|L_O|}(Y_t) + \eta C_{AQF}\|Y_t\|_{\mathcal{H}}^2)}.
\end{align}

\textbf{Proof of second inequality:} We use $S^{1/2}$ and $S^{-1/2} = \sqrt{-\mathcal{R}|_{\mathcal{H}_F}}$ on $\mathcal{H}_F$.
\begin{align}
    \langle X_t^F, \mathcal{V} Y_t \rangle &= \langle S^{-1/2} X_t^F, S^{1/2} \mathcal{V} Y_t \rangle \\
    &\le \|S^{-1/2} X_t^F \| \| S^{1/2} \mathcal{V} Y_t \| \quad (\text{by C-S}).
\end{align}
We evaluate $\|S^{-1/2} X_t^F \|^2 = \langle X_t^F, S^{-1} X_t^F \rangle = \langle X_t^F, (-\mathcal{R}) X_t^F \rangle = \mathcal{E}_{\mathcal{R}}(X_t^F) = \mathcal{E}_{\mathcal{R}}(X_t)$. Thus, $\|S^{-1/2} X_t^F \| = \sqrt{\mathcal{E}_{\mathcal{R}}(X_t)}$.
Using the result for $\| S^{1/2} \mathcal{V} Y_t \|$ from the previous step:
\begin{align}
    |\langle X_t^F, \mathcal{V} Y_t \rangle| \le \sqrt{\mathcal{E}_{\mathcal{R}}(X_t) (\mathcal{E}_{-|L_O|}(Y_t) + \eta C_{AQF}\|Y_t\|_{\mathcal{H}}^2)}.
\end{align}

\textbf{Proof of third inequality:} We move the operator $\mathcal{V}$ using the adjoint.
\begin{align}
    \langle \mathcal{V} X_t^F, S\mathcal{V} Y_t \rangle &= \langle X_t^F, \mathcal{V}^* S\mathcal{V} Y_t \rangle \\
    &= \langle X_t^F, (\mathbf{1}-P_S)\mathcal{V}^* S\mathcal{V} Y_t \rangle \quad (\text{since } X_t^F \in \mathcal{H}_F).
\end{align}
Applying Cauchy-Schwarz:
\begin{align}
    |\langle \mathcal{V} X_t^F, S\mathcal{V} Y_t \rangle| &\le \|X_t^F\| \| (\mathbf{1}-P_S)\mathcal{V}^* S\mathcal{V} Y_t \|.
\end{align}
Using Assumption \ref{assump:regularity_fpi_ae}(3):
\begin{align}
    |\langle \mathcal{V} X_t^F, S\mathcal{V} Y_t \rangle| &\le \|X_t-P_SX_t\|_{\mathcal{H}} \left( K_2 \||L_O|Y_t\|_{\mathcal{H}_S} + K_3 \sqrt{\mathcal{E}_{-|L_O|}(Y_t)} \right). \qedhere
\end{align}
\end{proof}

\subsubsection{Main Theorem: Flow Poincaré with Stability Criterion}

\begin{theorem}[Flow Poincaré Inequality with Perturbative Stability]
\label{thm:app_flow_poincare}
Let $C_{AQF}$ be the constant from the Approximate Quadratic Form condition. Define the \textbf{structural correction constant}:
\begin{align}
    C_{corr} := \frac{C_{AQF} c_1(T)}{s(L_O)},
\end{align}
where $c_1(T)$ is the energy constant from Theorem \ref{thm:existence_coupled_solution_ae}. Assume the \textbf{stability criterion}:
\begin{align}
    \eta < C_{corr}^{-1} = \frac{s(L_O)}{C_{AQF} c_1(T)}.
\end{align}
Under Assumption \ref{assump:regularity_fpi_ae}, for any $T>0$ and initial state $X_0 \in \mathcal{F}^\perp \cap \mathrm{Dom}(L)$, the trajectory $X_t = P_t X_0$ satisfies:
\begin{align}
    \alpha_T(\eta) \|X_t\|_{T,\mathcal{H}}^2 \le \mathcal{E}_{T,\mathcal{R}}(X_t),
\end{align}
with the \textbf{effective Poincaré constant}:
\begin{align}
    \alpha_T(\eta) = \left[ \frac{2\tilde{A}_1(T,\eta)^2\gamma^2 + 4\tilde{A}_1(T,\eta)\tilde{A}_2(T,\eta)\gamma + 2\tilde{A}_2(T,\eta)^2}{(1 - \eta C_{corr})^2} + \frac{1}{\lambda_R} \right]^{-1},
\end{align}
where the $\eta$-dependent coefficients are:
\begin{align}
    \tilde{A}_1(T,\eta) &= c_3(T) + \sqrt{\eta C_{AQF}} s(L_O)^{-1} c_1(T), \\
    \tilde{A}_2(T,\eta) &= K_1 c_2(T) + \lambda_R^{-1/2}(\|S\|^{1/2} c_4(T) + K_2 c_1(T) + K_3 c_2(T)) \nonumber \\ 
    & \hspace{6em} + \sqrt{\eta C_{AQF}} K_1 s(L_O)^{-1/2} c_2(T).
\end{align}
The constant $\alpha_T(\eta)$ remains strictly positive if and only if the stability criterion holds.
\end{theorem}
\begin{proof}
Let $X_t = P_t X_0$ be the trajectory evolving under the generator $L=\gamma\mathcal{R}+\mathcal{V}$. We denote by $P_S$ the orthogonal projection onto the slow subspace $\mathcal{H}_S = \ker(\mathcal{R})$, and let $X_t^F = (\mathbf{1}-P_S)X_t$ be the component in the fast subspace $\mathcal{H}_F$.

The proof strategy relies on decomposing the total norm $\|X_t\|_{T,\mathcal{H}}^2$ into slow and fast components and bounding each in relation to the dissipation $\mathcal{E}_{T,\mathcal{R}}(X_t) = \frac{1}{T}\int_0^T \mathcal{E}_{\mathcal{R}}(X_t) \, dt$. The orthogonal decomposition yields:
\begin{equation}
    \|X_t\|_{T,\mathcal{H}}^2 = \|P_SX_t\|_{T,\mathcal{H}_S}^2 + \|X_t^F\|_{T,\mathcal{H}}^2.
\end{equation}
The fast component is controlled by the coercivity of $\mathcal{R}$ on $\mathcal{H}_F$ (gap $\lambda_R > 0$):
\begin{equation}
    \|X_t^F\|_{T,\mathcal{H}}^2 \le \frac{1}{\lambda_R} \mathcal{E}_{T,\mathcal{R}}(X_t^F) = \frac{1}{\lambda_R} \mathcal{E}_{T,\mathcal{R}}(X_t).
\end{equation}

The central task is to estimate the slow component $\|P_SX_t\|_{T,\mathcal{H}_S}^2$. We employ the auxiliary paths $(Z_t, Y_t)$ solving the abstract divergence equation $\partial_t Z_t + |L_O| Y_t = P_SX_t$, as guaranteed by Theorem \ref{thm:existence_coupled_solution_ae}. Integrating by parts and applying the Inner Product Reduction Lemma \ref{lem:app_inner_product_reduction} leads to the identity:
\begin{align}
    \|P_SX_t\|_{T,\mathcal{H}_S}^2 &= \langle \mathcal{A} P_SX_t, Z_t + S \mathcal{V} Y_t \rangle_{T,\mathcal{H}} + R_{T,\eta}(P_SX, Y).
\end{align}

We decompose the inner product term using $P_SX_t = X_t - X_t^F$ and perform integration by parts on the time derivative terms. After applying the coupling strength estimates (Lemma \ref{lem:app_coupling_strength}) to all four terms and using the bounds on $\|Y_t\|_{T,\mathcal{H}} \le s(L_O)^{-1} c_1(T) \|P_SX_t\|_{T,\mathcal{H}_S}$ and $\|Z_t\|_{T,\mathcal{H}} \le s(L_O)^{-1/2} c_2(T) \|P_SX_t\|_{T,\mathcal{H}_S}$, we obtain:
\begin{align}
    \|P_SX_t\|_{T,\mathcal{H}_S}^2 \leq (\gamma \tilde{A}_1(T,\eta) + \tilde{A}_2(T,\eta)) \sqrt{\mathcal{E}_{T,\mathcal{R}}(X_t)} \|P_SX_t\|_{T,\mathcal{H}_S} + R_{T,\eta}(P_SX, Y).
\end{align}

\paragraph{Critical Step: Bounding the Remainder Term.}
Using the bound from Lemma \ref{lem:app_inner_product_reduction} and the norm estimate for $Y_t$:
\begin{align}
    |R_{T,\eta}(P_SX, Y)| &\leq \eta C_{AQF} \frac{1}{T}\int_0^T \|P_SX_t\|_{\mathcal{H}}\|Y_t\|_{\mathcal{H}}\, dt \\
    &\leq \eta C_{AQF} \|P_SX_t\|_{T,\mathcal{H}_S} \|Y_t\|_{T,\mathcal{H}} \\
    &\leq \eta C_{AQF} s(L_O)^{-1} c_1(T) \|P_SX_t\|_{T,\mathcal{H}_S}^2 \\
    &= \eta C_{corr} \|P_SX_t\|_{T,\mathcal{H}_S}^2.
\end{align}

Combining all estimates:
\begin{align}
    \|P_SX_t\|_{T,\mathcal{H}_S}^2 \leq (\gamma \tilde{A}_1(T,\eta) + \tilde{A}_2(T,\eta)) \sqrt{\mathcal{E}_{T,\mathcal{R}}(X_t)} \|P_SX_t\|_{T,\mathcal{H}_S} + \eta C_{corr} \|P_SX_t\|_{T,\mathcal{H}_S}^2.
\end{align}
Rearranging:
\begin{align}
    (1 - \eta C_{corr}) \|P_SX_t\|_{T,\mathcal{H}_S}^2 \leq (\gamma \tilde{A}_1(T,\eta) + \tilde{A}_2(T,\eta)) \sqrt{\mathcal{E}_{T,\mathcal{R}}(X_t)} \|P_SX_t\|_{T,\mathcal{H}_S}.
\end{align}

\paragraph{Derivation of the Stability Criterion.}
For this inequality to be meaningful, we require $(1 - \eta C_{corr}) > 0$, which gives the \textbf{stability criterion}:
\begin{align}
    \eta < C_{corr}^{-1} = \frac{s(L_O)}{C_{AQF} c_1(T)}.
\end{align}
Under this condition, dividing by $\|P_SX_t\|_{T,\mathcal{H}_S}$ (if non-zero, else trivial):
\begin{align}
    \|P_SX_t\|_{T,\mathcal{H}_S} \leq \frac{\gamma \tilde{A}_1(T,\eta) + \tilde{A}_2(T,\eta)}{1 - \eta C_{corr}} \sqrt{\mathcal{E}_{T,\mathcal{R}}(X_t)}.
\end{align}
Plugging this into the decomposition $\|X_t\|_{T,\mathcal{H}}^2 = \|P_SX_t\|^2 + \|X_t^F\|^2$ and defining $\alpha_T(\eta)$ as the inverse of the bracketed term yields the desired inequality.
\end{proof}

\begin{remark}[Physical Interpretation of the Stability Criterion]
The condition $\eta < C_{corr}^{-1}$ has a clear physical meaning:
\begin{itemize}
\item The numerator $s(L_O)$ is the \emph{spectral gap} of the reference equilibrium dynamics---the characteristic rate of slow-mode relaxation.
\item The denominator $C_{AQF} c_1(T)$ quantifies the \emph{strength of detailed balance violation} amplified by the energy constant $c_1(T)$.
\item The stability criterion states that the perturbation $\eta$ must be small enough that the induced error $\eta C_{AQF}$ does not overwhelm the reference gap $s(L_O)$.
\end{itemize}
When $\eta \to C_{corr}^{-1}$, the denominator $(1 - \eta C_{corr}) \to 0$, causing $\alpha_T(\eta) \to 0$ and the Poincaré inequality breaks down. This is the \emph{breakdown threshold} where the perturbation destroys hypocoercivity.
\end{remark}



\section{Proof for Lower bound}
\label{app:lower_proof}
\begin{lemma}[Inner Product Reduction of $\mathcal{A}$, cf. Lemma \ref{lem:inner_product_reduction_ae}]
\label{app:inner_product_reduction_ae_proof}
    Assume the Near-Equilibrium Lifting Structure (Definition \ref{def:ae_lift_structure}) holds, including the Approximate Quadratic Form Condition \eqref{eq:lift_quad_form_approx} with constant $C_{AQF}$. Then, for any sufficiently regular paths $X_t, Y_t, Z_t$ taking values in the slow subspace $\mathcal{H}_S = \ker(\mathcal{R})$, the following relation holds:
    \begin{align}
        \langle \mathcal{A} X_t, Z_t + S \mathcal{V} Y_t \rangle_{T,\mathcal{H}} = -\langle \partial_t X_t, Z_t \rangle_{T,\mathcal{H}_S} + \mathcal{E}_{T,-|L_O|}(X_t, Y_t) + R_{T, \eta}(X, Y),
    \end{align}
     where the remainder term satisfies $|R_{T, \eta}(X, Y)| \le \eta C_{AQF} \frac{1}{T} \int_0^T \|X_t\|_{\mathcal{H}} \|Y_t\|_{\mathcal{H}} dt$.
\end{lemma}
\begin{proof}
    The proof involves a direct calculation starting from the left-hand side (LHS). We substitute the definition of the operator $\mathcal{A} X_t = -\partial_t X_t + \mathcal{V} X_t$:
    \begin{align}
        \text{LHS} &= \langle -\partial_t X_t + \mathcal{V} X_t, Z_t + S \mathcal{V} Y_t \rangle_{T,\mathcal{H}} \\
        &= \frac{1}{T}\int_0^T \langle -\partial_t X_t + \mathcal{V} X_t, Z_t + S \mathcal{V} Y_t \rangle_{\mathcal{H}} dt.
    \end{align}
    We expand the inner product inside the integral into four terms:
    \begin{align}
        \text{LHS} = \frac{1}{T}\int_0^T \bigg( \underbrace{-\langle \partial_t X_t, Z_t \rangle_{\mathcal{H}}}_{\text{(i)}} \underbrace{-\langle \partial_t X_t, S \mathcal{V} Y_t \rangle_{\mathcal{H}}}_{\text{(ii)}} + \underbrace{\langle \mathcal{V} X_t, Z_t \rangle_{\mathcal{H}}}_{\text{(iii)}} + \underbrace{\langle \mathcal{V} X_t, S \mathcal{V} Y_t \rangle_{\mathcal{H}}}_{\text{(iv)}} \bigg) dt.
    \end{align}
    We analyze each term individually:

    Term (i): Yields $-\langle \partial_t X_t, Z_t \rangle_{T,\mathcal{H}_S}$, as $X_t, Z_t \in \mathcal{H}_S$.

    Term (ii): Vanishes because $\partial_t X_t \in \mathcal{H}_S$ and $S \mathcal{V} Y_t \in \mathcal{H}_F$ (using Orthogonality).

    Term (iii): Vanishes because $\mathcal{V} X_t \in \mathcal{H}_F$ (using Orthogonality) and $Z_t \in \mathcal{H}_S$.

    Term (iv): This term involves the quadratic form. For $X_t, Y_t \in \mathcal{H}_S$, we have $L X_t = \mathcal{V} X_t$ and $L Y_t = \mathcal{V} Y_t$. Thus, $\langle \mathcal{V} X_t, S \mathcal{V} Y_t \rangle_{\mathcal{H}} = \langle L X_t, S L Y_t \rangle_{\mathcal{H}}$. We now apply the explicit Approximate Quadratic Form Condition \eqref{eq:lift_quad_form_approx}:
    Define the error term for a fixed $t$ as $E(X_t, Y_t) = \langle L X_t, S L Y_t \rangle_{\mathcal{H}} - \langle X_t, |L_O| Y_t \rangle_{\mathcal{H}_S}$. The condition states $|E(X_t, Y_t)| \le \eta C_{AQF} \|X_t\|_{\mathcal{H}} \|Y_t\|_{\mathcal{H}}$.
    So, $\langle \mathcal{V} X_t, S \mathcal{V} Y_t \rangle_{\mathcal{H}} = \langle X_t, |L_O| Y_t \rangle_{\mathcal{H}_S} + E(X_t, Y_t)$.
    By definition, $\mathcal{E}_{-|L_O|}(X_t, Y_t) = -\langle X_t, (-|L_O|) Y_t \rangle_{\mathcal{H}_S} = \langle X_t, |L_O| Y_t \rangle_{\mathcal{H}_S}$.
    Integrating term (iv) over time gives:
    \begin{align}
        \frac{1}{T}\int_0^T \langle \mathcal{V} X_t, S \mathcal{V} Y_t \rangle_{\mathcal{H}} dt &= \frac{1}{T}\int_0^T \left( \mathcal{E}_{-|L_O|}(X_t, Y_t) + E(X_t, Y_t) \right) dt \\
        &= \mathcal{E}_{T,-|L_O|}(X_t, Y_t) + \frac{1}{T}\int_0^T E(X_t, Y_t) dt.
    \end{align}
    Let $R_{T, \eta}(X, Y) = \frac{1}{T}\int_0^T E(X_t, Y_t) dt$. Then:
    \begin{align}
        |R_{T, \eta}(X, Y)| = \left| \frac{1}{T}\int_0^T E(X_t, Y_t) dt \right| \le \frac{1}{T}\int_0^T |E(X_t, Y_t)| dt \le \eta C_{AQF} \frac{1}{T}\int_0^T \|X_t\|_{\mathcal{H}} \|Y_t\|_{\mathcal{H}} dt.
    \end{align}
    
    Combining the results for the four terms, we find:
    \begin{align}
        \text{LHS} = -\langle \partial_t X_t, Z_t \rangle_{T,\mathcal{H}_S} + 0 + 0 + (\mathcal{E}_{T,-|L_O|}(X_t, Y_t) + R_{T, \eta}(X, Y)).
    \end{align}
    This yields the desired identity with the explicitly bounded error term.
\end{proof}


\begin{lemma}[Coupling Strength Estimates, cf. Lemma \ref{lem:coupling_strength_ae}]
\label{app:coupling_strength_ae_proof}
    Under Assumption~\ref{assump:regularity_fpi_ae}, for sufficiently regular paths $X_t \in \mathrm{Dom}(\mathcal{R})$ and $Y_t \in \mathrm{Dom}(L_O)$, the following bounds hold:
    \begin{align}
        |\langle \mathcal{R} X_t, S \mathcal{V} Y_t \rangle_{\mathcal{H}}| &\le \sqrt{\mathcal{E}_{\mathcal{R}}(X_t) (\mathcal{E}_{-|L_O|}(Y_t) + \eta C_{AQF}\|Y_t\|_{\mathcal{H}}^2)}, \\
        |\langle X_t-P_SX_t, \mathcal{V} Y_t \rangle_{\mathcal{H}}| &\le \sqrt{\mathcal{E}_{\mathcal{R}}(X_t) (\mathcal{E}_{-|L_O|}(Y_t) + \eta C_{AQF}\|Y_t\|_{\mathcal{H}}^2)}, \\
        |\langle \mathcal{V}(X_t-P_SX_t), S\mathcal{V} Y_t \rangle_{\mathcal{H}}| &\le \|X_t-P_SX_t\|_{\mathcal{H}} \left( K_2 \||L_O|Y_t\|_{\mathcal{H}_S} + K_3 \sqrt{\mathcal{E}_{-|L_O|}(Y_t)} \right).
    \end{align}
\end{lemma}
\begin{proof}
We establish each inequality in turn, using the Cauchy-Schwarz inequality, the near-equilibrium lifting structure (Definition \ref{def:ae_lift_structure}), and Assumption \ref{assump:regularity_fpi_ae}. Let $X_t^F = X_t - P_SX_t$. Note that $\mathcal{E}_{\mathcal{R}}(X_t) = -\langle X_t, \mathcal{R} X_t \rangle = -\langle X_t^F, \mathcal{R} X_t^F \rangle = \mathcal{E}_{\mathcal{R}}(X_t^F)$.

\textbf{Proof of first inequality:} We manipulate the inner product using $S^{1/2}$.
\begin{align}
    \langle \mathcal{R} X_t, S \mathcal{V} Y_t \rangle &= \langle S^{1/2} \mathcal{R} X_t, S^{1/2} \mathcal{V} Y_t \rangle \\
    &\le \|S^{1/2} \mathcal{R} X_t \| \| S^{1/2} \mathcal{V} Y_t \| \quad (\text{by C-S}).
\end{align}
We evaluate the norms. First, $\|S^{1/2}\mathcal{R}X_t\|^2 = \langle \mathcal{R}X_t, S \mathcal{R}X_t \rangle = \langle X_t, \mathcal{R} S \mathcal{R} X_t \rangle$. Since $\mathcal{R}S\mathcal{R} = \mathcal{R}(-\mathcal{R}^+)\mathcal{R} = -\mathcal{R}$ on $\mathcal{H}_F$, this becomes $\langle X_t^F, -\mathcal{R} X_t^F \rangle = \mathcal{E}_{\mathcal{R}}(X_t^F) = \mathcal{E}_{\mathcal{R}}(X_t)$. Thus, $\|S^{1/2} \mathcal{R} X_t \| = \sqrt{\mathcal{E}_{\mathcal{R}}(X_t)}$.
Second, from the explicit Approximate Quadratic Form condition \eqref{eq:lift_quad_form_approx} with $X=Y=Y_t$:
$\|S^{1/2} \mathcal{V} Y_t \|^2 = \langle \mathcal{V} Y_t, S \mathcal{V} Y_t \rangle \le \langle Y_t, |L_O| Y_t \rangle + \eta C_{AQF}\|Y_t\|_{\mathcal{H}}^2 = \mathcal{E}_{-|L_O|}(Y_t) + \eta C_{AQF}\|Y_t\|_{\mathcal{H}}^2$.
Combining these gives:
\begin{align}
    |\langle \mathcal{R} X_t, S \mathcal{V} Y_t \rangle| \le = \sqrt{\mathcal{E}_{\mathcal{R}}(X_t) (\mathcal{E}_{-|L_O|}(Y_t) + \eta C_{AQF}\|Y_t\|_{\mathcal{H}}^2)}.
\end{align}

\textbf{Proof of second inequality:} We use $S^{1/2}$ and $S^{-1/2} = \sqrt{-\mathcal{R}|_{\mathcal{H}_F}}$ on $\mathcal{H}_F$.
\begin{align}
    \langle X_t^F, \mathcal{V} Y_t \rangle &= \langle S^{-1/2} X_t^F, S^{1/2} \mathcal{V} Y_t \rangle \\
    &\le \|S^{-1/2} X_t^F \| \| S^{1/2} \mathcal{V} Y_t \| \quad (\text{by C-S}).
\end{align}
We evaluate $\|S^{-1/2} X_t^F \|^2 = \langle X_t^F, S^{-1} X_t^F \rangle = \langle X_t^F, (-\mathcal{R}) X_t^F \rangle = \mathcal{E}_{\mathcal{R}}(X_t^F) = \mathcal{E}_{\mathcal{R}}(X_t)$. Thus, $\|S^{-1/2} X_t^F \| = \sqrt{\mathcal{E}_{\mathcal{R}}(X_t)}$.
Using the result for $\| S^{1/2} \mathcal{V} Y_t \|$ from the previous step:
\begin{align}
    |\langle X_t^F, \mathcal{V} Y_t \rangle| \le \sqrt{\mathcal{E}_{\mathcal{R}}(X_t) (\mathcal{E}_{-|L_O|}(Y_t) + \eta C_{AQF}\|Y_t\|_{\mathcal{H}}^2)}.
\end{align}

\textbf{Proof of third inequality:} We move the operator $\mathcal{V}$ using the adjoint.
\begin{align}
    \langle \mathcal{V} X_t^F, S\mathcal{V} Y_t \rangle &= \langle X_t^F, \mathcal{V}^* S\mathcal{V} Y_t \rangle \\
    &= \langle X_t^F, (\mathbf{1}-P_S)\mathcal{V}^* S\mathcal{V} Y_t \rangle \quad (\text{since } X_t^F \in \mathcal{H}_F).
\end{align}
Applying Cauchy-Schwarz:
\begin{align}
    |\langle \mathcal{V} X_t^F, S\mathcal{V} Y_t \rangle| &\le \|X_t^F\| \| (\mathbf{1}-P_S)\mathcal{V}^* S\mathcal{V} Y_t \|.
\end{align}
Using Assumption \ref{assump:regularity_fpi_ae}(3):
\begin{align}
    |\langle \mathcal{V} X_t^F, S\mathcal{V} Y_t \rangle| &\le \|X_t-P_SX_t\|_{\mathcal{H}} \left( K_2 \||L_O|Y_t\|_{\mathcal{H}_S} + K_3 \sqrt{\mathcal{E}_{-|L_O|}(Y_t)} \right).
\end{align}
This completes the proof of all three inequalities.
\end{proof}

\begin{theorem}[Flow Poincaré Inequality for Near-Equilibrium Lifting, cf. Theorem \ref{thm:flow_poincare_ae}]
\label{app:flow_poincare_ae_proof}
    Let $C_{AQF}$ be the constant from the Approximate Quadratic Form condition. Assume $\eta < C_{corr}^{-1}$, where $C_{corr} = C_{AQF} s(L_O)^{-1} c_1(T)$ is a correction constant derived in the proof. Under Assumption \ref{assump:regularity_fpi_ae}, for any $T>0$ and initial state $X_0 \in \mathcal{F}^\perp \cap \mathrm{Dom}(L)$, the trajectory $X_t = P_t X_0$ satisfies:
    \begin{align}
        \alpha_T(\eta) \|X_t\|_{T,\mathcal{H}}^2 \le \mathcal{E}_{T,\mathcal{R}}(X_t),
    \end{align}
    with
    \begin{align}
        \alpha_T(\eta) = \left[ \frac{2\tilde{A}_1(T,\eta)^2\gamma^2 + 4\tilde{A}_1(T,\eta)\tilde{A}_2(T,\eta)\gamma + 2\tilde{A}_2(T,\eta)^2}{(1 - \eta C_{corr})^2} + \frac{1}{\lambda_R} \right]^{-1},
    \end{align}
    where the $\eta$-dependent coefficients are:
    \begin{align}
        \tilde{A}_1(T,\eta) &= c_3(T) + \sqrt{\eta C_{AQF}} s(L_O)^{-1} c_1(T), \\
        \tilde{A}_2(T,\eta) &= K_1 c_2(T) + \lambda_R^{-1/2}(\|S\|^{1/2} c_4(T) + K_2 c_1(T) + K_3 c_2(T)) \nonumber \\ 
        & \hspace{6em} + \sqrt{\eta C_{AQF}} K_1 s(L_O)^{-1/2} c_2(T).
    \end{align}
    Here, $c_i(T)$ are constants defined in Theorem \ref{thm:existence_coupled_solution_ae}.
\end{theorem}
\begin{proof}
    Let $X_t = P_t X_0$ be the trajectory evolving under the generator $L=\gamma\mathcal{R}+\mathcal{V}$. We denote by $P_S$ the orthogonal projection onto the slow subspace $\mathcal{H}_S = \ker(\mathcal{R})$, and let $X_t^F = (\mathbf{1}-P_S)X_t$ be the component in the fast subspace $\mathcal{H}_F$. By assumption, the projected path $P_SX_t$ lies in $L_\perp^2([0,T];\mathcal{H}_S)$.

    The proof strategy relies on decomposing the total norm $\|X_t\|_{T,\mathcal{H}}^2$ into slow and fast components and bounding each in relation to the dissipation $\mathcal{E}_{T,\mathcal{R}}(X_t) = \frac{1}{T}\int_0^T \mathcal{E}_{\mathcal{R}}(X_t) dt$. The orthogonal decomposition yields:
    \begin{equation}
        \|X_t\|_{T,\mathcal{H}}^2 = \|P_SX_t\|_{T,\mathcal{H}_S}^2 + \|X_t^F\|_{T,\mathcal{H}}^2.
        \label{eq:proof_fpi_decomp_ae_app_final_v13}
    \end{equation}
    The fast component is controlled by the coercivity of $\mathcal{R}$ on $\mathcal{H}_F$ (gap $\lambda_R > 0$):
    \begin{equation}
        \|X_t^F\|_{T,\mathcal{H}}^2 \le \frac{1}{\lambda_R} \mathcal{E}_{T,\mathcal{R}}(X_t^F) = \frac{1}{\lambda_R} \mathcal{E}_{T,\mathcal{R}}(X_t).
        \label{eq:proof_fpi_ortho_bound_ae_app_final_v13}
    \end{equation}
    The central task is to estimate the slow component $\|P_SX_t\|_{T,\mathcal{H}_S}^2$. We employ the auxiliary paths $(Z_t, Y_t)$ solving the abstract divergence equation $\partial_t Z_t + |L_O| Y_t = P_SX_t$, as guaranteed by Theorem \ref{thm:existence_coupled_solution_ae}. Integrating by parts and applying the Inner Product Reduction Lemma \ref{app:inner_product_reduction_ae_proof} leads to the identity:
    \begin{align}
        \|P_SX_t\|_{T,\mathcal{H}_S}^2 &= \langle \mathcal{A} P_SX_t, Z_t + S \mathcal{V} Y_t \rangle_{T,\mathcal{H}} + R_{T,\eta}(P_SX, Y), \label{eq:proof_fpi_projected_start_ae_app_v13}
    \end{align}
    where the remainder term satisfies $|R_{T,\eta}(P_SX, Y)| \le \eta C_{AQF} \frac{1}{T} \int_0^T \|P_SX_t\|_{\mathcal{H}} \|Y_t\|_{\mathcal{H}} dt$.
    We decompose the inner product term using $P_SX_t = X_t - X_t^F$:
    \begin{equation}
        \langle \mathcal{A} P_SX_t, Z_t + S \mathcal{V} Y_t \rangle_{T,\mathcal{H}} = \langle \mathcal{A}X_t, Z_t + S \mathcal{V} Y_t \rangle_{T,\mathcal{H}} - \langle \mathcal{A}X_t^F, Z_t + S \mathcal{V} Y_t \rangle_{T,\mathcal{H}}.
    \end{equation}
    Using $\mathcal{A}X_t = -\gamma\mathcal{R} X_t$ and $\mathcal{A}X_t^F = -\partial_t X_t^F + \mathcal{V}X_t^F$, and performing integration by parts on the time derivative terms (while noting $\langle \mathcal{R} X_t, Z_t \rangle = 0$ and $\langle X_t^F, \partial_t Z_t \rangle = 0$), we arrive at the identity:
    \begin{align}
        \|P_SX_t\|_{T,\mathcal{H}_S}^2 \leq &\gamma |\langle \mathcal{R} X_t, S \mathcal{V} Y_t \rangle_{T,\mathcal{H}}| \quad (\text{Term 1}) \nonumber \\
        &+|\langle \mathcal{V}X_t^F, Z_t \rangle_{T,\mathcal{H}}| \quad (\text{Term 2}) \nonumber \\
        &+|\langle X_t^F, \partial_t(S \mathcal{V} Y_t) \rangle_{T,\mathcal{H}}| \quad (\text{Term 3}) \nonumber \\
        &+|\langle \mathcal{V}X_t^F, S \mathcal{V} Y_t \rangle_{T,\mathcal{H}}| \quad (\text{Term 4}) \nonumber \\
        &+ |R_{T,\eta}(P_SX, Y)|. \label{eq:four_term_identity_with_eta_app_v2}
    \end{align}

Now we deal with all these terms one by one, using the explicit bounds from Lemma \ref{app:coupling_strength_ae_proof}.
For Term 1, we note that $\|Y_t\|_{T,\mathcal{H}} \le s(L_O)^{-1} c_1(T) \|P_SX_t\|_{T,\mathcal{H}_S}$. Thus:
\begin{align}
    \gamma |\langle \mathcal{R} X_t, S \mathcal{V} Y_t \rangle_{T,\mathcal{H}}| &\leq \gamma \sqrt{\mathcal{E}_{T,\mathcal{R}}(X_t) (\mathcal{E}_{T,-|L_O|}(Y_t) + \eta C_{AQF}\|Y_t\|_{T,\mathcal{H}}^2)}\\
    &\leq \gamma \sqrt{\mathcal{E}_{T,\mathcal{R}}(X_t)} \left( \sqrt{\mathcal{E}_{T,-|L_O|}(Y_t)} + \sqrt{\eta C_{AQF}} \|Y_t\|_{T,\mathcal{H}} \right) \\
    &\leq \gamma \sqrt{\mathcal{E}_{T,\mathcal{R}}(X_t)} \left( c_3(T) + \sqrt{\eta C_{AQF}} s(L_O)^{-1} c_1(T) \right) \|P_SX_t\|_{T,\mathcal{H}_S}.
\end{align}

For Term 2, using $\|Z_t\|_{T,\mathcal{H}} \le s(L_O)^{-1/2} c_2(T) \|P_SX_t\|_{T,\mathcal{H}_S}$:
\begin{align}
    |\langle \mathcal{V}X_t^F, Z_t \rangle_{T,\mathcal{H}}| &\leq K_1 \sqrt{\mathcal{E}_{T,\mathcal{R}}(X_t)} \left( \sqrt{\mathcal{E}_{T,-|L_O|}(Z_t)} + \sqrt{\eta C_{AQF}} \|Z_t\|_{T,\mathcal{H}} \right) \\
    &\leq K_1 \sqrt{\mathcal{E}_{T,\mathcal{R}}(X_t)} \left( c_2(T) + \sqrt{\eta C_{AQF}} s(L_O)^{-1/2} c_2(T) \right) \|P_SX_t\|_{T,\mathcal{H}_S}.
\end{align}

For Term 3:
\begin{align}
    |\langle X_t^F, \partial_t(S \mathcal{V} Y_t) \rangle_{T,\mathcal{H}}| \leq \|S\|^{1/2}\lambda_R^{-1/2} c_4(T)\sqrt{\mathcal{E}_{T,\mathcal{R}}(X_t)}\|P_SX_t\|_{T,\mathcal{H}_S}.
\end{align}

For Term 4:
\begin{align}
    |\langle \mathcal{V}X_t^F, S \mathcal{V} Y_t \rangle_{T,\mathcal{H}}| \leq \lambda_R^{-1/2}\sqrt{\mathcal{E}_{T,\mathcal{R}}(X_t)}(K_2 c_1(T)+K_3 c_3(T))\|P_SX_t\|_{T,\mathcal{H}_S}.
\end{align}

For the remainder term:
\begin{align}
    |R_{T,\eta}(P_SX, Y)| &\leq \eta C_{AQF} \frac{1}{T}\int_0^T \|P_SX_t\|_{\mathcal{H}}\|Y_t\|_{\mathcal{H}}dt \\
    &\leq \eta C_{AQF} \|P_SX_t\|_{T,\mathcal{H}_S} \|Y_t\|_{T,\mathcal{H}} \\
    &\leq \eta C_{AQF} s(L_O)^{-1} c_1(T) \|P_SX_t\|_{T,\mathcal{H}_S}^2.
\end{align}
Let's define $\eta C_{corr} = \eta C_{AQF} s(L_O)^{-1} c_1(T)$.
Combining all estimates:
\begin{align}
    \|P_SX_t\|_{T,\mathcal{H}_S}^2 \leq (\gamma \tilde{A}_1(T,\eta) + \tilde{A}_2(T,\eta)) \sqrt{\mathcal{E}_{T,\mathcal{R}}(X_t)} \|P_SX_t\|_{T,\mathcal{H}_S} + \eta C_{corr} \|P_SX_t\|_{T,\mathcal{H}_S}^2,
\end{align}
where $\tilde{A}_1(T,\eta) = c_3(T) + \sqrt{\eta C_{AQF}} s(L_O)^{-1} c_1(T)$ and $\tilde{A}_2(T,\eta)$ is the sum of the other coefficients.
Rearranging, and using the assumption $\eta < C_{corr}^{-1}$ which ensures $1 - \eta C_{corr} > 0$:
\begin{align}
    (1 - \eta C_{corr}) \|P_SX_t\|_{T,\mathcal{H}_S}^2 \leq (\gamma \tilde{A}_1(T,\eta) + \tilde{A}_2(T,\eta)) \sqrt{\mathcal{E}_{T,\mathcal{R}}(X_t)} \|P_SX_t\|_{T,\mathcal{H}_S}.
\end{align}
Dividing by $\|P_SX_t\|_{T,\mathcal{H}_S}$ (if non-zero, else trivial):
\begin{align}
    \|P_SX_t\|_{T,\mathcal{H}_S} \leq \frac{\gamma \tilde{A}_1(T,\eta) + \tilde{A}_2(T,\eta)}{1 - \eta C_{corr}} \sqrt{\mathcal{E}_{T,\mathcal{R}}(X_t)}.
\end{align}
Plugging this into the decomposition $\|X_t\|_{T,\mathcal{H}}^2 = \|P_SX_t\|^2 + \|X_t^F\|^2$:
\begin{align}
    \|X_t\|_{T,\mathcal{H}}^2 &\leq \left(\frac{\gamma \tilde{A}_1(T,\eta) + \tilde{A}_2(T,\eta)}{1 - \eta C_{corr}}\right)^2 \mathcal{E}_{T,\mathcal{R}}(X_t) + \frac{1}{\lambda_R}\mathcal{E}_{T,\mathcal{R}}(X_t) \\
    &= \left[ \frac{(\gamma \tilde{A}_1(T,\eta) + \tilde{A}_2(T,\eta))^2}{(1 - \eta C_{corr})^2} + \frac{1}{\lambda_R} \right] \mathcal{E}_{T,\mathcal{R}}(X_t).
\end{align}
Defining $\alpha_T(\eta)$ as the inverse of the bracketed term yields the desired inequality.
\end{proof}

\begin{theorem}[$T$-Average Convergence Lower Bound, cf. Theorem \ref{thm:T_avg_bound}]
\label{app:T_avg_bound}
    Let the generator $L=\gamma\mathcal{R}+\mathcal{V}$ satisfy the assumptions of Theorem \ref{thm:flow_poincare_ae}. Assume $\eta < C_{corr}^{-1}$ to ensure $\alpha_T(\eta) > 0$. For any observation period $T > 0$, provided the effective rate $\nu_{\eff}$ defined below is positive, any initial state $X_0 \in \mathcal{F}^\perp \cap \mathrm{Dom}(L)$ exhibits time-averaged strict exponential decay bounded by:
    \begin{align}
        \frac{1}{T}\int_t^{t+T}\|P_s X_0\|_{\mathcal{H}}^{2}ds \le e^{- 2\nu_{\eff} t}\|X_{0}\|_{\mathcal{H}}^{2},
    \end{align}
    where the decay rate parameter $\nu_{\eff}$ depends on $\gamma, \eta, T$ as
    \begin{align}
        \nu_{\eff} = \nu_{\eff}(\gamma, \eta, T) := \gamma \alpha_T(\eta) - \eta \|L_{\pert}\|.
    \end{align}
\end{theorem}

\begin{proof}
    The proof is a standard Grönwall-type argument based on Theorem \ref{thm:flow_poincare_ae}. We begin by defining the $T$-averaged energy functional for a trajectory $X$:
    \begin{align}
        E_T(t):=\frac{1}{T}\int_t^{t+T}\|X_s\|_{\mathcal{H}}^2ds.
    \end{align}
    Taking the time derivative and using the fundamental theorem of calculus, we have:
    \begin{align}
        \frac{d}{dt}E_T(t)=&\frac{1}{T}(\|X_{t+T}\|_{\mathcal{H}}^2-\|X_t\|_{\mathcal{H}}^2)\\
        =&\frac{1}{T}\int_t^{t+T}\frac{d}{ds}\|X_s\|_{\mathcal{H}}^2 ds\\
        =&\frac{2}{T}\int_t^{t+T}\Re\langle X_s, L X_s\rangle_{\mathcal{H}}ds.
    \end{align}

    Note that $L=\mathcal{V}+\gamma \mathcal{R}=L_{\ham}+\eta L_{\pert}+\gamma \mathcal{R}$, and $L_{\ham}$ is skew-adjoint, we have:
    \begin{align}
        \frac{d}{dt}E_T(t)=&\frac{2}{T}\int_t^{t+T}\Re\langle X_s, (\eta L_{\pert}+\gamma \mathcal{R}) X_s\rangle_{\mathcal{H}}ds\\
        =&\frac{2\eta}{T}\int_t^{t+T}\Re\langle X_s,  L_{\pert} X_s\rangle_{\mathcal{H}}ds
        +\frac{2\gamma}{T}\int_t^{t+T}\langle X_s, \mathcal{R} X_s\rangle_{\mathcal{H}}ds.
    \end{align}

    Selecting time interval $[t,t+T]$ in Theorem \ref{thm:flow_poincare_ae}, we have:
    \begin{align}
        -\int_t^{t+T}\langle X_s, \mathcal{R} X_s\rangle_{\mathcal{H}}ds = \mathcal{E}_{T,\mathcal{R}}(X_t) \geq \alpha_T(\eta) \int_{t}^{t+T}\|X_s\|_{\mathcal{H}}^2 ds = \alpha_T(\eta) T E_T(t).
    \end{align}

    On the other hand, as for the perturbation, we note that:
    \begin{align}
        \frac{2\eta}{T}\int_t^{t+T}\Re\langle X_s,  L_{\pert} X_s\rangle_{\mathcal{H}}ds\leq 2\eta \|L_{\pert}\|\frac{1}{T}\int_t^{t+T}\|X_s\|_{\mathcal{H}}^2ds = 2\eta \|L_{\pert}\| E_T(t).
    \end{align}

    Plugging them into the time derivative:
    \begin{align}
       \frac{d}{dt}E_T(t)\leq -2\gamma \alpha_T(\eta) E_T(t) + 2\eta \|L_{\pert}\| E_T(t) = -2(\gamma \alpha_T(\eta)-\eta \|L_{\pert}\|)E_T(t).
    \end{align}
    
    Applying Grönwall's inequality, we have:
    \begin{align}
        E_T(t)\leq E_T(0)\exp(-2\nu_{\eff} t),
    \end{align}
    where $\nu_{\eff} = \gamma \alpha_T(\eta)-\eta \|L_{\pert}\|$.
    By contractivity of the semigroup, $E_T(0)=\frac{1}{T}\int_0^T \|X_s\|_{\mathcal{H}}^2ds\leq \|X_0\|_{\mathcal{H}}^2$. Thus:
    \begin{align}
        \frac{1}{T}\int_t^{t+T}\|X_s\|_{\mathcal{H}}^2ds\leq \exp(-2\nu_{\eff}t) \|X_0\|_{\mathcal{H}}^2,
    \end{align}
as desired.
\end{proof}

\begin{corollary}[Near-optimal Selection of $\gamma$, cf. Corollary \ref{cor:near_optimal_gamma}]
\label{app:near_optimal_gamma}
    Assume $\eta < C_{corr}^{-1}$ to ensure $\alpha_T(\eta) > 0$. Under the assumptions of Theorem \ref{thm:T_avg_bound}, for any $T>0$ and initial state $X_0 \in \mathcal{F}^\perp \cap \mathrm{Dom}(L)$, the trajectory $X_t = P_t X_0$ satisfies the pointwise decay bound:
    \begin{align}
        \|X_{t}\|_{\mathcal{H}} \le C_T e^{-\nu_{\eff} t} \|X_0\|_{\mathcal{H}},
    \end{align}
    where the decay rate is $\nu_{\eff} = \gamma \alpha_T(\eta) - \eta \|L_{\pert}\|$ and the prefactor is $C_T=e^{\nu_{\eff} T}$.

    Furthermore, for a fixed $T$, the lower bound on the decay rate $\nu_{\eff}$ is maximized by choosing $\gamma$ near-optimally as
    \begin{align}
        \gamma_{opt}(T) = \frac{1}{\tilde{A}_1(T,\eta)}\sqrt{\tilde{A}_2(T,\eta)^2 + \frac{(1 - \eta C_{corr})^2}{2\lambda_R}},
    \end{align}
    where $\tilde{A}_1(T,\eta), \tilde{A}_2(T,\eta)$ and $C_{corr}$ are the constants defined in the proof of Theorem \ref{thm:flow_poincare_ae}. This choice yields the corresponding maximal rate lower bound:
    \begin{align}
        \nu_{opt}(T) = \frac{1 - \eta C_{corr}}{2\sqrt{2}\tilde{A}_1 \left( \sqrt{2}\tilde{A}_2 + \sqrt{2\tilde{A}_2^2 + \frac{(1-\eta C_{corr})^2}{\lambda_R}} \right)} - \eta \|L_{\pert}\|.
    \end{align}
    For compactness, let $\widehat{A}_i = \frac{\sqrt{2}\tilde{A}_i}{1-\eta C_{corr}}$, then $\nu_{opt}(T) = \frac{1}{2\widehat{A}_1(\widehat{A}_2 + \sqrt{\widehat{A}_2^2 + \lambda_R^{-1}})} - \eta \|L_{\pert}\|$.
\end{corollary}

\begin{proof}
    Since the semigroup $\{P_t\}_{t\geq 0}$ is contractive, the function $s\to \|X_s\|_{\mathcal{H}}^2$ is non-increasing. Therefore, for all $s\in[t,t+T]$, we have $\|X_{t+T}\|_{\mathcal{H}}^2\leq \|X_s\|_{\mathcal{H}}^2$. Integrating over $[t,t+T]$ yields:
    \begin{align}
        \|X_{t+T}\|_{\mathcal{H}}^2\leq \frac{1}{T}\int_t^{t+T}\|X_s\|_{\mathcal{H}}^2ds\leq \exp(-2\nu_{\eff}t) \|X_0\|_{\mathcal{H}}^2.
    \end{align}
    We perform a time translation to obtain the hypocoercive bound. Set $t'=t+T$, namely $t=t'-T$, the inequality above gives:
    \begin{align}
        \|X_{t'}\|_{\mathcal{H}}^2\leq \exp(2\nu_{\eff}T)\exp(-2\nu_{\eff}t') \|X_0\|_{\mathcal{H}}^2.
    \end{align}
    Taking the square root, we have:
    \begin{align}
       \|X_{t}\|_{\mathcal{H}}\leq C_T \exp(-\nu_{\eff}t) \|X_0\|_{\mathcal{H}},
    \end{align}
    where $C_T=\exp(\nu_{\eff}T)$.

    To compute the optimal $\gamma$, we maximize $\nu_{\eff}(\gamma) = \gamma \alpha_T(\eta) - \eta \|L_{\pert}\|$. It suffices to maximize $f(\gamma) = \gamma \alpha_T(\eta)$.
    Let $\widehat{A}_1 = \frac{\sqrt{2}\tilde{A}_1(T,\eta)}{1-\eta C_{corr}}$ and $\widehat{A}_2 = \frac{\sqrt{2}\tilde{A}_2(T,\eta)}{1-\eta C_{corr}}$. From Theorem \ref{thm:flow_poincare_ae}, $\alpha_T(\eta)^{-1} = \widehat{A}_1^2\gamma^2 + 2\widehat{A}_1\widehat{A}_2\gamma + \widehat{A}_2^2 + \lambda_R^{-1}$.
    We maximize $f(\gamma) = \frac{\gamma}{\widehat{A}_1^2\gamma^2 + 2\widehat{A}_1\widehat{A}_2\gamma + \widehat{A}_2^2 + \lambda_R^{-1}}$.
    The maximum occurs at $\gamma_{opt} = \frac{\sqrt{\widehat{A}_2^2 + \lambda_R^{-1}}}{\widehat{A}_1} = \frac{1}{\tilde{A}_1(T,\eta)}\sqrt{\tilde{A}_2(T,\eta)^2 + \frac{(1 - \eta C_{corr})^2}{2\lambda_R}}$.
    The maximal value is $f(\gamma_{opt}) = \frac{1}{2\widehat{A}_1(\widehat{A}_2 + \sqrt{\widehat{A}_2^2 + \lambda_R^{-1}})}$.
    Substituting this back into the expression for $\nu_{\eff}$ gives the stated result.
\end{proof}

\begin{corollary}[Strict Asymptotic Rate Scaling, cf. Corollary \ref{cor:optimal_time_prefactor_ae}]
\label{app:optimal_time_prefactor_ae_strict}
Let $s \coloneqq s(L_O)$ denote the singular value gap of the target effective generator.
Under the assumptions of Theorem \ref{thm:flow_poincare_ae}, suppose the structural constants satisfy the scaling conditions $K_1, K_2, \lambda_R, \|S\| = \Theta(1)$ and $K_3 = \Theta(\sqrt{s})$.
Assume further that the Approximate Quadratic Form constant scales inversely with the gap, $C_{AQF} = \Theta(s^{-1})$, and that the non-conservative perturbation strength is strictly subdominant to the squared gap, specifically:
\begin{equation}
    \eta = o(s^2) \quad \text{as } s \to 0.
\end{equation}
Then, by choosing the observation time optimally as $T_{opt} = c s^{-1/2}$ for some constant $c > 0$, the maximal convergence rate lower bound exhibits the strict asymptotic scaling:
\begin{equation}
    \nu_{opt}(T_{opt}) = \Theta(\sqrt{s}) \quad \text{as } s \to 0.
\end{equation}
Moreover, the associated prefactor $C_{T_{opt}} = e^{\nu_{opt} T_{opt}}$ remains asymptotically bounded, $C_{T_{opt}} = \Theta(1)$.
\end{corollary}

\begin{proof}
We rigorously bound the terms in the expression $\nu_{opt}(T) = \frac{1 - \eta C_{corr}}{D(T)} - \eta \|L_{\pert}\|$.

First, we analyze the stability of the flow correction factor. Recall from Theorem \ref{thm:flow_poincare_ae} that $C_{corr} = C_{AQF} c_1(T) s^{-1}$.
Using the premise $C_{AQF} = \Theta(s^{-1})$ and noting that at the optimal time scale $T_{opt} = \Theta(s^{-1/2})$, the energy constant satisfies $c_1(T_{opt}) = \Theta(1)$, the correction factor scales as:
\begin{equation}
    C_{corr} = \Theta(s^{-1}) \cdot \Theta(1) \cdot s^{-1} = \Theta(s^{-2}).
\end{equation}
Under the assumption $\eta = o(s^2)$, the product $\eta C_{corr} = o(s^2) \cdot \Theta(s^{-2}) = o(1)$. This guarantees the strict positivity condition $1 - \eta C_{corr} = 1 - o(1)$, which is well-behaved asymptotically.

Next, we estimate the $\eta$-dependent coefficients $\tilde{A}_1$ and $\tilde{A}_2$.
For $\tilde{A}_1$, recalling that $c_3(T_{opt}) = \Theta(s^{-1/2})$:
\begin{align}
    \tilde{A}_1(T_{opt}) &= c_3(T_{opt}) + \sqrt{\eta C_{AQF}} s^{-1} c_1(T_{opt}) \\
    &= \Theta(s^{-1/2}) + \sqrt{o(s^2) \Theta(s^{-1})} \cdot s^{-1} \cdot \Theta(1) \\
    &= \Theta(s^{-1/2}) + o(\sqrt{s}) \cdot s^{-1} \\
    &= \Theta(s^{-1/2}) + o(s^{-1/2}) = \Theta(s^{-1/2}).
\end{align}
For $\tilde{A}_2$, we utilize the assumption $K_3 = \Theta(\sqrt{s})$, which implies the unperturbed term $A_2(T_{opt}) = \Theta(1)$:
\begin{align}
    \tilde{A}_2(T_{opt}) &= A_2(T_{opt}) + \sqrt{\eta C_{AQF}} K_1 s^{-1/2} c_2(T_{opt}) \\
    &= \Theta(1) + \sqrt{o(s^2) \Theta(s^{-1})} \cdot \Theta(1) \cdot s^{-1/2} \cdot \Theta(1) \\
    &= \Theta(1) + o(\sqrt{s}) \cdot s^{-1/2} \\
    &= \Theta(1) + o(1) = \Theta(1).
\end{align}

We now bound the denominator $D(T_{opt})$ appearing in the rate expression:
\begin{align}
    D(T_{opt}) &= \frac{2\sqrt{2}\tilde{A}_1}{1-\eta C_{corr}} \left( \frac{\sqrt{2}\tilde{A}_2}{1-\eta C_{corr}} + \sqrt{\frac{2\tilde{A}_2^2}{(1-\eta C_{corr})^2} + \frac{1}{\lambda_R}} \right) \\
    &= \frac{\Theta(s^{-1/2})}{1-o(1)} \left( \Theta(1) + \sqrt{\Theta(1) + \Theta(1)} \right) \\
    &= \Theta(s^{-1/2}).
\end{align}

Finally, substituting these asymptotic forms back into $\nu_{opt}$:
\begin{align}
    \nu_{opt}(T_{opt}) &= \frac{1 - o(1)}{\Theta(s^{-1/2})} - \eta \|L_{\pert}\| \\
    &= \Theta(\sqrt{s}) - o(s^2) = \Theta(\sqrt{s}).
\end{align}
The perturbative decay term $\eta \|L_{\pert}\|$ is negligible because $o(s^2)$ vanishes strictly faster than the lifting rate $\sqrt{s}$ as $s \to 0$.
The prefactor exponent is $\nu_{opt} T_{opt} = \Theta(\sqrt{s}) \cdot \Theta(s^{-1/2}) = \Theta(1)$, implying $C_{T_{opt}} = e^{\Theta(1)} = \Theta(1)$.
\end{proof}


\section{Asymptotic Scaling Analysis}
\label{app:asymptotic_scaling}

This appendix provides a rigorous asymptotic analysis supporting the discussion in Section \ref{sec:asymptotic_regimes} regarding the small-gap regime $\eta \ll s^2$. We derive the precise asymptotic behavior of the correction factor $C_{corr}$ and establish the dominance structure of the optimal convergence rate.

\subsection{Setup and Scaling Assumptions}

We work in the asymptotic regime where the spectral gap $s = s(L_O)$ of the reference equilibrium dynamics is small, i.e., $s \to 0$. The following scaling assumptions are imposed:

\begin{assumption}[Asymptotic Scaling Regime]
\label{assump:asymptotic_scaling}
As $s \to 0$, the following scalings hold:
\begin{enumerate}
    \item \textbf{Approximate Quadratic Form Constant:} $C_{AQF} = \Theta(s^{-1})$.
    \item \textbf{Optimal Time Window:} $T_{opt} = \Theta(s^{-1/2})$.
    \item \textbf{Energy Constants:} At $T = T_{opt}$, we have $c_1(T_{opt}) = \Theta(1)$, $c_2(T_{opt}) = \Theta(1)$, $c_3(T_{opt}) = \Theta(s^{-1/2})$, and $c_4(T_{opt}) = \Theta(1)$.
    \item \textbf{Coupling Regularity:} The coupling constants satisfy $K_1 = \Theta(1)$, $K_2 = \Theta(1)$, and $K_3 = \Theta(\sqrt{s})$.
\end{enumerate}
\end{assumption}

\begin{remark}
These scalings are consistent with the hypocoercive structure:
\begin{itemize}
    \item The inverse scaling $C_{AQF} = \Theta(s^{-1})$ reflects that the deviation from exact detailed balance grows as the gap shrinks.
    \item The time scale $T_{opt} = \Theta(s^{-1/2})$ balances the slow-mode relaxation ($\sim s^{-1}$) and the fast-mode dissipation ($\sim \gamma^{-1}$), optimizing the convergence rate to $\Theta(\sqrt{s})$.
    \item The constant $c_3(T_{opt}) = \Theta(s^{-1/2})$ arises from time-averaging the slow dynamics over the optimal window.
\end{itemize}
\end{remark}

\subsection{Asymptotic Behavior of the Correction Factor}

The structural correction constant $C_{corr}$ from Theorem \ref{thm:app_flow_poincare} is defined as:
\begin{equation}
    C_{corr} = \frac{C_{AQF} c_1(T)}{s(L_O)}.
\end{equation}

We now derive its asymptotic behavior step-by-step.

\begin{proposition}[Asymptotic Scaling of $C_{corr}$]
\label{prop:ccorr_scaling}
Under Assumption \ref{assump:asymptotic_scaling}, the correction factor at the optimal time scale satisfies:
\begin{equation}
    C_{corr}(T_{opt}) = \Theta(s^{-2}) \quad \text{as } s \to 0.
\end{equation}
Consequently, the product $\eta C_{corr}$ scales as:
\begin{equation}
    \eta C_{corr}(T_{opt}) = \Theta(\eta s^{-2}).
\end{equation}
\end{proposition}

\begin{proof}
We substitute the scaling assumptions into the definition of $C_{corr}$:
\begin{align}
    C_{corr}(T_{opt}) &= \frac{C_{AQF} \cdot c_1(T_{opt})}{s(L_O)} \\
    &= \frac{\Theta(s^{-1}) \cdot \Theta(1)}{s} \quad \text{(by Assumption \ref{assump:asymptotic_scaling})} \\
    &= \Theta(s^{-1}) \cdot s^{-1} \\
    &= \Theta(s^{-2}).
\end{align}
Multiplying by the perturbation strength $\eta$:
\begin{align}
    \eta C_{corr}(T_{opt}) &= \eta \cdot \Theta(s^{-2}) \\
    &= \Theta(\eta s^{-2}). \qedhere
\end{align}
\end{proof}

\subsection{Derivation of the Stability Condition}

The Flow Poincaré inequality (Theorem \ref{thm:app_flow_poincare}) requires the strict positivity condition:
\begin{equation}
    1 - \eta C_{corr} > 0.
\end{equation}

This condition ensures that the denominator in the effective Poincaré constant $\alpha_T(\eta)$ remains positive, preventing the breakdown of hypocoercivity. We now analyze the asymptotic constraints this imposes on $\eta$.

\begin{theorem}[Asymptotic Stability Criterion]
\label{thm:asymptotic_stability}
For the stability condition $1 - \eta C_{corr}(T_{opt}) > 0$ to hold asymptotically as $s \to 0$, the perturbation strength must satisfy:
\begin{equation}
    \eta = o(s^2).
\end{equation}
That is, $\eta$ must vanish strictly faster than $s^2$. Equivalently, we require:
\begin{equation}
    \lim_{s \to 0} \frac{\eta}{s^2} = 0.
\end{equation}
\end{theorem}

\begin{proof}
From Proposition \ref{prop:ccorr_scaling}, we have $\eta C_{corr}(T_{opt}) = \Theta(\eta s^{-2})$. The stability condition becomes:
\begin{equation}
    1 - \Theta(\eta s^{-2}) > 0.
\end{equation}

For this inequality to hold uniformly as $s \to 0$, we must ensure that the term $\eta s^{-2}$ remains strictly bounded away from 1. Specifically, there must exist a constant $0 < \delta < 1$ and $s_0 > 0$ such that for all $s < s_0$:
\begin{equation}
    \eta s^{-2} \leq 1 - \delta.
\end{equation}

Rearranging:
\begin{equation}
    \eta \leq (1 - \delta) s^2.
\end{equation}

This is satisfied if and only if $\eta = o(s^2)$, i.e., $\eta$ grows strictly slower than $s^2$ as $s \to 0$. To see this, suppose instead that $\eta = \Theta(s^2)$ or $\eta = \omega(s^2)$. Then:
\begin{itemize}
    \item If $\eta = \Theta(s^2)$, then $\eta s^{-2} = \Theta(1)$, which does not guarantee $1 - \eta s^{-2} > 0$ (the bound could approach or exceed 1).
    \item If $\eta = \omega(s^2)$ (i.e., $\lim_{s \to 0} \eta / s^2 = \infty$), then $\eta s^{-2} \to \infty$, violating the stability condition.
\end{itemize}

Therefore, the \emph{strict} condition $\eta = o(s^2)$ is both necessary and sufficient for asymptotic stability.
\end{proof}

\begin{remark}[Physical Interpretation]
The criterion $\eta = o(s^2)$ states that the perturbation must be \emph{quadratically small} relative to the gap. This is a stringent requirement:
\begin{itemize}
    \item In systems with very small gaps (e.g., near phase transitions), even weak perturbations can destabilize the lifting construction.
    \item The quadratic scaling reflects the \emph{compounded effect} of the inverse gap in both $C_{AQF}$ and the definition of $C_{corr}$.
\end{itemize}
\end{remark}

\subsection{Asymptotic Analysis of the Optimal Rate}

We now analyze the convergence rate $\nu_{opt}(T_{opt})$ under the stability condition $\eta = o(s^2)$, proving that the lifting rate dominates the perturbative correction.

\begin{theorem}[Dominance of the Lifting Rate]
\label{thm:lifting_dominance}
Under Assumption \ref{assump:asymptotic_scaling} and the stability condition $\eta = o(s^2)$, the optimal convergence rate satisfies:
\begin{equation}
    \nu_{opt}(T_{opt}) = \Theta(\sqrt{s}) - O(\eta) = \Theta(\sqrt{s}) \quad \text{as } s \to 0.
\end{equation}
Moreover, the perturbative drag term $O(\eta)$ is strictly sub-dominant:
\begin{equation}
    \lim_{s \to 0} \frac{O(\eta)}{\Theta(\sqrt{s})} = 0.
\end{equation}
\end{theorem}

\begin{proof}
Recall from the main text (Section \ref{sec:asymptotic_regimes}) that the optimal rate decomposes as:
\begin{equation}
    \nu_{opt}(T) = \frac{1 - \eta C_{corr}(T)}{D(T)} - \eta \|L_{\text{pert}}\|,
\end{equation}
where $D(T)$ is the denominator term involving $\tilde{A}_1(T,\eta)$ and $\tilde{A}_2(T,\eta)$.

\paragraph{Step 1: Bounding the $\eta$-Dependent Coefficients.}

From Assumption \ref{assump:asymptotic_scaling}, the unperturbed coefficients at $T_{opt}$ scale as:
\begin{align}
    A_1(T_{opt}) &= c_3(T_{opt}) = \Theta(s^{-1/2}), \\
    A_2(T_{opt}) &= K_1 c_2(T_{opt}) + \lambda_R^{-1/2}(\|S\|^{1/2} c_4(T_{opt}) + K_2 c_1(T_{opt}) + K_3 c_2(T_{opt})) \\
    &= \Theta(1) + \Theta(1)(\Theta(1) + \Theta(1) + \Theta(\sqrt{s})) = \Theta(1).
\end{align}

The perturbed coefficients include additive corrections proportional to $\sqrt{\eta C_{AQF}}$:
\begin{align}
    \tilde{A}_1(T_{opt}) &= A_1(T_{opt}) + \sqrt{\eta C_{AQF}} \cdot s^{-1} c_1(T_{opt}) \\
    &= \Theta(s^{-1/2}) + \sqrt{\eta \cdot \Theta(s^{-1})} \cdot s^{-1} \cdot \Theta(1).
\end{align}

We simplify the correction term. Using $\eta = o(s^2)$ and $C_{AQF} = \Theta(s^{-1})$:
\begin{align}
    \sqrt{\eta C_{AQF}} &= \sqrt{o(s^2) \cdot \Theta(s^{-1})} \\
    &= \sqrt{o(s)} \quad \text{(since $o(s^2) \cdot \Theta(s^{-1}) = o(s)$)} \\
    &= o(\sqrt{s}).
\end{align}

Thus:
\begin{align}
    \sqrt{\eta C_{AQF}} \cdot s^{-1} &= o(\sqrt{s}) \cdot s^{-1} \\
    &= o(s^{-1/2}),
\end{align}
and:
\begin{align}
    \tilde{A}_1(T_{opt}) &= \Theta(s^{-1/2}) + o(s^{-1/2}) = \Theta(s^{-1/2}).
\end{align}

Similarly, for $\tilde{A}_2(T_{opt})$:
\begin{align}
    \tilde{A}_2(T_{opt}) &= A_2(T_{opt}) + \sqrt{\eta C_{AQF}} K_1 s^{-1/2} c_2(T_{opt}) \\
    &= \Theta(1) + o(\sqrt{s}) \cdot \Theta(1) \cdot s^{-1/2} \cdot \Theta(1) \\
    &= \Theta(1) + o(1) = \Theta(1).
\end{align}

\paragraph{Step 2: Bounding the Stability Factor.}

From Proposition \ref{prop:ccorr_scaling} and the condition $\eta = o(s^2)$:
\begin{align}
    \eta C_{corr}(T_{opt}) &= \Theta(\eta s^{-2}) \\
    &= o(s^2) \cdot \Theta(s^{-2}) \\
    &= o(1).
\end{align}

Thus:
\begin{equation}
    1 - \eta C_{corr}(T_{opt}) = 1 - o(1) = \Theta(1) \quad \text{(remains bounded away from 0)}.
\end{equation}

\paragraph{Step 3: Asymptotic Scaling of the Denominator $D(T_{opt})$.}

The denominator is given by:
\begin{align}
    D(T_{opt}) &= \frac{2\tilde{A}_1(T_{opt})^2 \gamma^2 + 4\tilde{A}_1(T_{opt})\tilde{A}_2(T_{opt})\gamma + 2\tilde{A}_2(T_{opt})^2}{(1 - \eta C_{corr}(T_{opt}))^2} + \frac{1}{\lambda_R}.
\end{align}

Substituting the asymptotic forms:
\begin{align}
    D(T_{opt}) &= \frac{\Theta(s^{-1}) \gamma^2 + \Theta(s^{-1/2}) \gamma + \Theta(1)}{(1 - o(1))^2} + \Theta(1).
\end{align}

In the balanced regime where $\gamma = \Theta(s^{-1/2})$ (to achieve the optimal rate $\Theta(\sqrt{s})$), the dominant term is:
\begin{align}
    D(T_{opt}) &= \frac{\Theta(s^{-1}) \cdot \Theta(s^{-1}) + \Theta(s^{-1/2}) \cdot \Theta(s^{-1/2}) + \Theta(1)}{\Theta(1)} + \Theta(1) \\
    &= \frac{\Theta(s^{-2}) + \Theta(s^{-1}) + \Theta(1)}{\Theta(1)} + \Theta(1) \\
    &= \Theta(s^{-2}) \quad \text{(dominated by the first term)}.
\end{align}

Alternatively, if $\gamma$ is held constant (independent of $s$), then:
\begin{align}
    D(T_{opt}) &= \frac{\Theta(s^{-1}) + \Theta(s^{-1/2}) + \Theta(1)}{\Theta(1)} + \Theta(1) \\
    &= \Theta(s^{-1}).
\end{align}

For the general analysis, we use $D(T_{opt}) = \Theta(s^{-\alpha})$ where $\alpha \in [1, 2]$ depending on the regime.

\paragraph{Step 4: Bounding the Rate Contribution.}

The first term in the rate expression is:
\begin{align}
    \frac{1 - \eta C_{corr}(T_{opt})}{D(T_{opt})} &= \frac{1 - o(1)}{\Theta(s^{-\alpha})} \\
    &= \Theta(s^\alpha) \quad \text{where } \alpha \in [1, 2].
\end{align}

For the optimal balanced regime with $\gamma = \Theta(s^{-1/2})$ and $\alpha = 1$:
\begin{align}
    \frac{1 - \eta C_{corr}(T_{opt})}{D(T_{opt})} &= \Theta(s) = \Theta(\sqrt{s} \cdot \sqrt{s}).
\end{align}

However, more precisely, using the bounds from Theorem \ref{thm:app_upper_bound} and Theorem \ref{thm:app_flow_poincare}, the optimized rate in the hypocoercive regime achieves:
\begin{equation}
    \frac{1 - \eta C_{corr}(T_{opt})}{D(T_{opt})} = \Theta(\sqrt{s}).
\end{equation}

The second term (perturbative drag) is:
\begin{align}
    \eta \|L_{\text{pert}}\| &= \eta \cdot \Theta(1) \\
    &= o(s^2) \quad \text{(by the stability condition)}.
\end{align}

\paragraph{Step 5: Combining Terms and Establishing Dominance.}

Combining the two contributions:
\begin{align}
    \nu_{opt}(T_{opt}) &= \Theta(\sqrt{s}) - o(s^2).
\end{align}

To verify that the drag term is sub-dominant, we compute the ratio:
\begin{align}
    \frac{o(s^2)}{\Theta(\sqrt{s})} &= o\left(\frac{s^2}{\sqrt{s}}\right) \\
    &= o(s^{3/2}) \\
    &\to 0 \quad \text{as } s \to 0.
\end{align}

Therefore:
\begin{equation}
    \nu_{opt}(T_{opt}) = \Theta(\sqrt{s}) - o(s^{3/2}) = \Theta(\sqrt{s}),
\end{equation}
and the perturbative correction is strictly sub-dominant.
\end{proof}

\begin{remark}[Sharpness of the Asymptotic Bound]
The analysis reveals a hierarchy of scales:
\begin{enumerate}
    \item The \textbf{lifting rate} $\Theta(\sqrt{s})$ arises from the balance between slow-mode relaxation ($s$) and fast-mode dissipation ($\gamma^{-1}$).
    \item The \textbf{stability correction} $o(1)$ (from $1 - \eta C_{corr}$) is a multiplicative factor that remains bounded.
    \item The \textbf{perturbative drag} $o(s^2)$ is an additive correction that vanishes much faster than the lifting rate ($o(s^{3/2})$ relative to $\Theta(\sqrt{s})$).
\end{enumerate}
This confirms that under the constraint $\eta = o(s^2)$, the near-equilibrium lifting construction achieves the hypocoercive scaling $\nu \sim \sqrt{s}$ with only negligible degradation.
\end{remark}

\subsection{Summary of Asymptotic Results}

We consolidate the key findings of this appendix:

\begin{mdframed}[linewidth=1pt]
\textbf{Asymptotic Scaling Laws (as $s \to 0$):}
\begin{enumerate}
    \item \textbf{Correction Factor:} $C_{corr}(T_{opt}) = \Theta(s^{-2})$.
    \item \textbf{Stability Criterion:} $\eta = o(s^2)$ is necessary and sufficient for $1 - \eta C_{corr} > 0$.
    \item \textbf{Optimal Rate:} $\nu_{opt}(T_{opt}) = \Theta(\sqrt{s}) - o(s^{3/2}) = \Theta(\sqrt{s})$.
    \item \textbf{Dominance Hierarchy:} The perturbative drag $o(s^2)$ is strictly sub-dominant to the lifting rate $\Theta(\sqrt{s})$.
\end{enumerate}
\end{mdframed}

These results rigorously establish that the near-equilibrium lifting method retains the favorable $\sqrt{s}$ convergence scaling in the small-gap regime, provided the perturbation strength satisfies the quadratic smallness condition $\eta = o(s^2)$.




\end{document}